\chapter{Oriented Graph}\label{chap: general oriented graph}

The properties of semi-complete digraphs basically inherit
the properties of tournaments,
and most of the proofs are almost the same.

We now move on to another family of graphs that are less
similar to tournaments: oriented graphs.
We will investigate general properties of oriented graphs
in this chapter.
In \cref{chap: quasi-transitive}, we will focus on
specific families of oriented graphs.

Recall \cref{def:oriented graph},
an oriented graph is a digraph that doesn't
have self-loops (a vertex beats itself)
and double ties.

\begin{theorem}\label{the: emperor then only king in oriented graph}
  If there is an emperor \(k\) in an oriented graph \(G\),
  then \(k\) is the only king in the graph.
\end{theorem}
\begin{proof}
  The emperor \(k\) beats every vertex in one step,
  therefore it is a king.

  Because there is no double ties in an oriented graph,
  if \(k\) is an emperor,
  then no vertex in \(G\) can beats \(k\),
  hence no vertex can beat \(k\) by one or two steps.
  Therefore, \(k\) is the only king in the graph.
\end{proof}

\begin{corollary}\label{the: if vertex with out-degree n-1 then only one king}
  In an oriented graph with \(n\) vertices,
  if there exists a vertex with out-degree \(n-1\),
  then there is only one king in the graph.
\end{corollary}

\begin{proof}
  A vertex with out-degree \(n-1\) beats every vertex
  in the graph except itself.
  Therefore that vertex is an emperor,
  and because of \cref{the: emperor then only king in oriented graph},
  there will be only one king in this graph.
\end{proof}

\begin{lemma}\label{the: add edge only add king}
  For an oriented graph \(G\),
  if we add a new edge to \(G\) to get \(G'\),
  the kings in \(G\) remain kings in \(G'\).
\end{lemma}

\begin{proof}
  Need to show that for every king \(k\) in \(G\),
  \(k\) is also a king in \(G'\).
  We can see that for every pair of vertices \(a, b\)
  if \(a \to b\) in \(G\), then \(a \to b\) in \(G'\),
  since we are not removing any edges.

  \begin{itemize}
    \item
      If \(k\) beats a vertex \(v\) by one step in \(G\),
      then \(k \to v\) in \(G\),
      therefore \(k \to v\) in \(G'\)
    \item
      If \(k\) beats a vertex \(v\) by 2 steps in \(G\),
      then exists vertex \(a\) such that
      \(k \to a \to v\) in \(G\), then \(k \to a \to v\) in \(G'\)
      and \(k\) beats \(v\) by 2 steps in \(G'\).
  \end{itemize}

  Therefore, \(k\) is also a king in \(G'\),
  then every king in \(G\) is preserved in \(G'\).
\end{proof}

It follows from \cref{the: add edge only add king}
that adding an edge to oriented graph \(G\)
cannot decrease the number of kings.

\begin{lemma}\label{the: no (1 0) oriented graph}
  There does not exist a \((1, 0)\) oriented graph.
\end{lemma}
\begin{proof}
  If any oriented graph has only 1 vertex,
  then that vertex by definition is a king.
  Therefore, there does not exist a \((1, 0)\) oriented graph.
\end{proof}

\begin{lemma}\label{the: no (2 2) oriented graph}
  There does not exist a \((2, 2)\) oriented graph
\end{lemma}

\begin{proof}
  \begin{figure}
    \centering
    \begin{subfigure}{0.45\linewidth}
      \centering
      \tikz\graph[layered layout, math nodes, grow=right, sibling distance=2cm, level sep=0.75cm] {
      a; b;
      };
      \caption{2 vertices with no edge.}
      \label{fig: all oriented graph with 2 vertices: no edge}  % chktex 24
    \end{subfigure}
    \begin{subfigure}{0.45\linewidth}
      \centering
      \tikz\graph[layered layout, math nodes, grow=right, sibling distance=2cm, level sep=0.75cm] {
      a -> b;
      };
      \caption{2 vertices with 1 edge.}
      \label{fig: all oriented graph with 2 vertices: 1 edge}  % chktex 24
    \end{subfigure}
    \caption{all the oriented graph with 2 vertices.}
    \label{fig: all oriented graph with 2 vertices}  % chktex 24
  \end{figure}
  \cref{fig: all oriented graph with 2 vertices}
  shows all the possible oriented graphs with 2 vertices.
  Notice, the oriented graph in
  \cref{fig: all oriented graph with 2 vertices: no edge}
  has 0 king,
  and the oriented graph in
  \cref{fig: all oriented graph with 2 vertices: 1 edge}
  has only 1 king.
  Therefore, there does not exist a \((2,2)\) oriented graph.
\end{proof}
\begin{lemma}\label{the: no (3 2) oriented graph}
  There does not exist a \((3, 2)\) oriented graph
\end{lemma}

\begin{proof}
  We consider all the oriented graphs with 3 vertices.

  First, consider the graph with maximum out-degree 0.
  The graph with maximum out-degree 0 will have no edge,
  therefore there cannot be a king,
  since no vertex can beat other vertices by 2 steps.

  \begin{figure}
    \centering
    \begin{subfigure}{0.3\linewidth}
      \centering
      \tikz\graph[simple necklace layout, math nodes, node sep=1cm] {
      a -> b; c;
      };
      \caption{3 vertices with 1 edge.}
      \label{fig: 3 oriented graph with max out-degree 1: 1 edge}  % chktex 24
    \end{subfigure}
    \begin{subfigure}{0.3\linewidth}
      \centering
      \tikz\graph[simple necklace layout, math nodes, node sep=1cm] {
      a -> b -> c;
      };
      \caption{3 vertices with 2 edges.}
      \label{fig: 3 oriented graph with max out-degree 1: 2 edge}  % chktex 24
    \end{subfigure}
    \begin{subfigure}{0.3\linewidth}
      \centering
      \tikz\graph[simple necklace layout, math nodes, node sep=1cm] {
      a -> b -> c -> a;
      };
      \caption{3 vertices with 3 edges.}
      \label{fig: 3 oriented graph with max out-degree 1: 3 edge}  % chktex 24
    \end{subfigure}
    \caption{all the oriented graphs with 3 vertices and maximum out-degree 1.}
    \label{fig: 3 oriented graph with max out-degree 1}  % chktex 24
  \end{figure}

  Then, consider the graph with maximum out-degree 1.
  In \cref{fig: 3 oriented graph with max out-degree 1},
  We show every oriented graph with 3 vertices and
  maximum out-degree 1.
  In \cref{fig: 3 oriented graph with max out-degree 1: 1 edge},
  there is no king;
  in \cref{fig: 3 oriented graph with max out-degree 1: 2 edge},
  there is 1 king;
  in \cref{fig: 3 oriented graph with max out-degree 1: 3 edge},
  every vertex is a king, therefore it has 3 kings.

  Finally, we consider the oriented graphs
  with maximum out-degree 2.
  By \cref{the: if vertex with out-degree n-1 then only one king},
  because we have 3 vertices in the graph,
  and at least one vertex has out-degree 2,
  there can only be 1 king.

  Therefore, there is no \((3,2)\) oriented graph.
\end{proof}

\begin{lemma}\label{the: no (4 4) oriented graph}
  There does not exist a \((4,4)\) oriented graph.
\end{lemma}
\begin{proof}
  Assume there exists a \((4, 4)\) oriented graph.
  When we add edges to this \((4, 4)\) oriented graph:
  \begin{itemize}
    \item
      the number of kings cannot increase,
      since a graph cannot have more kings than vertices.
    \item
      the number of kings cannot decrease,
      because of \cref{the: add edge only add king}
  \end{itemize}

  Then we can keep adding edges until
  every pair of vertices are adjacent
  and get a \((4,4)\) tournament.
  By \cref{the: (n k) tournament exists},
  there do not exist \((4,4)\) tournaments.

  Therefore, there cannot exist \((4,4)\) oriented graphs.
\end{proof}

We have shown that \((1,0), (2,2), (3,2)\) and \((4,4)\)
oriented graphs do not exists.
These four cases turns out to be the only exceptions.

\begin{lemma}\label{the: (n 2) oriented graph}
  There exists an \((n, 2)\) oriented graph for \(n \geq 4\).
\end{lemma}

\begin{proof}
  \begin{figure}
    \centering
    \tikz\graph[tree layout, grow=down, math nodes, sibling distance=1cm,level sep=1cm] {
    a -> b -> {c_1, c_2, "\ldots", c_{n-3}, c_{n-2}};  %chktex 18
    c_1 -> a;
    };
    \caption{only \(a\) and \(b\) are kings for \(n \geq 4\).}
    \label{fig: (n 2) oriented graph}  %chktex 24
  \end{figure}
  We can see in \cref{fig: (n 2) oriented graph} that
  \(c_1\) cannot dominate \(c_2\) by 1 or 2 steps,
  and other \(c_i (i \neq 1)\) cannot dominate any other vertex,
  because the out-degree of these vertices are 0.

  \(a\) is a king because \(a \to b\) and \(b\) beats every \(c_i\),
  therefore \(a\) beats \(b\) by one step,
  and \(a\) beats every \(c_i\) by 2 steps.
  \(b\) beats every \(c_i\) by 1 step,
  and \(b\) beats \(a\) by 2 steps: \(b \to c_1 \to a\).
\end{proof}

\begin{lemma}\label{the: (n 0) oriented graph}
  There exists \((n, 0)\) oriented graph for \(n \geq 0\)
  except \(n = 1\)
\end{lemma}

\begin{proof}
  For an oriented graph with \(n\) vertices and no edge,
  every vertex cannot beat any other vertex.
  Therefore, the graph has 0 kings
  and the graph is a \((n, 0)\) oriented graph.
\end{proof}

\begin{theorem}\label{the: (n k) oriented graph}
  There exists an \((n, k)\) oriented graph for all \(n \geq k \geq 0\),
  with the exception of \((1, 0)\), \((2, 2)\), \((3, 2)\),
  and \((4, 4)\) oriented graph.
\end{theorem}

\begin{proof}
  \cref{the: (n k) tournament exists} shows that
  there exists an \((n, k)\) tournament for all \(n \geq k \geq 1\)
  with the exception of \((n, 2)\), and \((4, 4)\).

  Because tournaments are also oriented graphs and
  by \cref{the: no (2 2) oriented graph,the: no (3 2) oriented graph,the: no (1 0) oriented graph,the: no (4 4) oriented graph}, \cref{the: (n 2) oriented graph}, and \cref{the: (n 0) oriented graph},
  exceptions listed above are the only exceptions.
\end{proof}

We generalized the result on tournaments from~\cite{maurer_king_1980}
to oriented graphs and show that there are only 4
\((n, k)\) oriented graphs that do not exist.

Following the idea from \cref{chap: semi-complete digraph},
one of the questions to ask is how can we use
ties more ``efficiently''.
The construction method in the proof of
\cref{the: (n 2) oriented graph} is very inefficient.

Here we present a ``better'' way to construct
these oriented graphs that uses at most one tie.

\begin{lemma}\label{the: (n 2) with one tie}
  There exists an \((n, 2)\) oriented graph
  with only one tie, for \(n \geq 4\).
\end{lemma}

\begin{proof}
  \begin{figure}
    \centering
    \tikz\graph[tree layout, math nodes, grow=down, sibling distance=2cm,level sep=0.75cm] {
    a -> b -> {c, T_{n-3} [draw, circle]};
    T_{n-3} <- c;
    T_{n-3} -> [bend right] a;
    };
    \caption{the constructive proof for
    \cref{the: (n 2) with one tie}}
    \label{fig: (n 2) with one tie} %chktex 24
  \end{figure}

  See \cref{fig: (n 2) with one tie},
  \(T_{n - 3}\) is a tournament of \(n - 3\) vertices.
  In this graph, the only tie is between \(a\) and \(c\)
  and the only kings are \(a\) and \(b\).

  \(a\) is a king because, \(a \to b \to c\)
  and \(a \to b \to T_{n - 3}\),
  therefore \(a\) beats \(b\) by 1 step
  and \(a\) beats \(c\) and \(T_{n-3}\) by 2 steps.
  \(b\) is a king because, \(b \to c\)
  and \(b \to T_{n - 3} \to a\)
  (because \(T_{n-3}\) is not empty),
  therefore \(b\) beats \(c\) and \(T_{n - 3}\),
  and \(b\) beats \(a\) by 2 steps.

  \(c\) is not a king, the shortest path from \(c\) to \(b\)
  is \(c \to T _{n-3} \to a \to b\).
  Any vertex \(v\) in \(T_{n-3}\) cannot be a king,
  because the shortest path between \(v\) and \(c\)
  is \(v \to a \to b \to c\) which has length 3.
\end{proof}

\begin{lemma}\label{the: (n 0) oriented graph with 1 tie}
  There exists \((n, 0)\) oriented graph where \(n \neq 1\)
  with at most 1 tie.
\end{lemma}
\begin{proof}
  First, we can see that a \((0, 0)\) oriented graph exists,
  it is just an empty graph with no vertex and edge.

  \begin{figure}
  \centering
    \tikz\graph[layered layout, math nodes, grow=right, sibling distance=2cm, level sep=0.75cm] {
      a; b;
    };
    \caption{\((2, 0)\) oriented graph.}
    \label{fig: (2 0) oriented graph with one tie}  % chktex 24
  \end{figure}
  Then, in \cref{fig: (2 0) oriented graph with one tie}
  we show that a \((2, 0)\) oriented graph exists with one tie:
  it is just 2 vertices and no edge between them.

  \begin{figure}
  \centering
    \tikz\graph[layered layout, grow=right, sibling distance=2cm, level sep=2cm] {
      "\((n, 0)\) oriented graph" [draw, circle]  %chktex 18
      -> "\(s\)";  %chktex 18
    };
    \caption{construct \((n+1, 0)\) oriented graph from \((n, 0)\) oriented graph.}
    \label{fig: (n+1 0) oriented graph with one tie}  % chktex 24
  \end{figure}
  In \cref{fig: (n+1 0) oriented graph with one tie},
  we give a way to construct an \((n+1, 0)\) oriented graph
  from an \((n, 0)\) oriented graph.
  We denote the \((n, 0)\) oriented graph as \(G\).

  We need to show the resulting graph
  in \cref{fig: (n+1 0) oriented graph with one tie}
  has no king:
  first, \(s\) cannot be a king,
  because it does not beat any vertex.
  We then show any vertex \(v \in V(G)\) is not a king.
  Because \(v\) is not a king in \(G\),
  there exists a vertex \(v' \in V(G)\)
  such that \(v\) cannot beat \(v'\) by 1 or 2 steps in \(G\).
  Then \(v\) still cannot beat \(v'\) in this new graph,
  because \(s\) is the only added vertex,
  and the path \(v \to s \to v'\) do not exist.
\end{proof}

With \cref{the: (n 2) with one tie} and
\cref{the: (n 0) oriented graph with 1 tie},
we can prove the following theorem.

\begin{theorem}\label{the: (n k) oriented graph with one tie}
  There exists an \((n, k)\) oriented graph
  with at most one tie for all \(n \geq k \geq 0\),
  with the exception of \((1, 0)\), \((2, 2)\),
  \((3, 2)\), and \((4, 4)\) oriented graphs.
\end{theorem}

\begin{proof}
  Almost the same proof as \cref{the: (n k) oriented graph},
  just substitute \cref{the: (n 2) oriented graph},
  \cref{the: (n 0) oriented graph} with
  \cref{the: (n 2) with one tie}
  and \cref{the: (n 0) oriented graph with 1 tie},
  respectively.
\end{proof}

In other words,
\cref{the: (n k) oriented graph with one tie}
states that for all \((n, k)\) oriented graphs that exist,
they can be constructed with only one tie.

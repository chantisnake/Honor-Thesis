\chapter{Abstract}

This thesis explores how to find and construct kings
in three generalizations of tournament:
semi-compelete digraphs, oriented graphs and
quasi-transitive oriented graphs.

In \cref{chap: semi-complete digraph} and \cref{chap: general oriented graph},
We present a way to interpret semi-compelete digraphs
and oriented graphs as tournaments with ``ties''
(we call the ``ties'' in semi-compelete digraphs ``double ties'',
and the ``ties'' in oriented graphs ``ties'').
In \cref{chap: semi-complete digraph},
we prove there exists an \((n, k)\) semi-complete digraphs
if and only if \(n \geq k \geq 1\),
and all the \((n, k)\) semi-compelete digraphs that exists
can be constructed with at most 1 double tie.
In \cref{chap: general oriented graph},
we prove there exists an \((n, k)\) oriented
for all \(n \geq k \geq 0\) except
\((1,0)\), \((2,2)\), \((3,2)\), and \((4,4)\) oriented graphs
and all the \((n, k)\) oriented graphs that exists
can be constructed with at most 1 tie.

The main focus of this thesis is quasi-transitive oriented graph,
which is discussed in \cref{chap: quasi-transitive}.
We showed a interesting fact that
all the quasi-transitive oriented graphs
can be condensed into tournaments by
``tie component condensations''.
Then, we showed that the tie component condensation
on a quasi-transitive oriented graph
is a most efficient condensation to tournament
in all the condensations to tournaments
defined on all the oriented graph with the same tie structure.
Finally we prove that the kings in
quasi-transitive oriented graph \(Q\) is related to
the kings in the ``underlying tournament of \(Q\)''
(result of \(Q\) after tie component condensation).
This result gives us a way to understand the
properties of kings in quasi-transitive oriented graphs
using the properties of king in tournaments.

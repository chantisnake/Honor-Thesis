\section{Tie Components}

We continue to explore the relationship between tie paths and
arrow direction transmitted by ties.
We define the following structure with
the intuition of ``path'' and ``connected component''
in an undirected graph~\cite{west_introduction_2001}.

\begin{definition}
  In a digraph \(G\), a
  \keyword{tie component of vertex \(a\)}[tie component of a vertex],
  denoted \(C(a)\), is the set of vertices \(v\)
  such that there exists a tie path between \(a\) and \(v\).
\end{definition}

\begin{definition}
  In a digraph \(G\), a tie component \(C(a)\)
  is a \keyword{trivial tie component}
  if \(C(a)\) only contains \(a\) itself
  and no other vertex.
\end{definition}


\begin{figure}
  \centering
  \begin{subfigure}[b]{0.45\linewidth}
    \centering
    \tikz\graph[simple necklace layout, math nodes, node sep=1cm] {
      a -> {c, d, e};
      b -> {c, d, e};
      {c, d, e} -> f;
      {a, b} -> f;
      c -> d --[dashed] e --[dashed] c;
      a --[dashed] b;
    };
    \caption{distinct tie components are
    \(\set{a, b}, \set{c, d, e}, \set{f}\)}
    \label{fig: tie component example} %chktex 24
  \end{subfigure}
  \begin{subfigure}[b]{0.45\linewidth}
    \centering
    \tikz\graph[simple necklace layout, math nodes, node sep=1cm] {
      a;
      c;
      d --e --c;
      a --b;
      f;
    };
    \caption{connected components are
    \(\set{a, b}, \set{c, d, e}, \set{f}\)}
    \label{fig: connected component example} %chktex 24
  \end{subfigure}
  \caption{the tie components in (a) are the connected components in (b).}
  \label{fig: tie components and connected components}  % chktex 24
\end{figure}

In \cref{fig: tie components and connected components},
we first change every tie in
\cref{fig: tie component example} to an undirected edge
and then remove all the directed edges
to obtain \cref{fig: connected component example},
and tie components in \cref{fig: tie component example}
exactly correspond to connected components in
\cref{fig: connected component example}.
Therefore, just like a tie path is similar to a path
in an undirected graph,
a tie component is very similar to a connected component
(called a ``component'' in~\cite{west_introduction_2001})
in an undirected graph,
and we can bring all the properties and intuitions
of connected components to tie components.

We prove results about tie components in digraphs.
Notice that theses results work in all digraphs,
not just in quasi-transitive oriented graphs.

\begin{theorem}\label{the: tie path equivalence relation}
  For a digraph \(G\),
  we define a relation \(\mathrel{R}\) on \(V(G)\)
  for vertices \(a, b \in V(G)\),
  \(a\mathrel{R}b\) if and only if
  there exists a tie path between \(a\) and \(b\).
  \(\mathrel{R}\) is an equivalence relation.
\end{theorem}

\begin{proof}
  Reflexive: given a vertex \(a\),
  \([a]\) is a tie path of length 0.
  therefore there exists a tie path from \(a\) to itself.

  Symmetry: if there exists a tie path from \(a_0\) to \(a_n\):
  \([a_0, a_1, a_2, \ldots, a_n]\),
  then because ties do not have directions,
  \([a_n, a_{n-1}, \ldots, a_1, a_0]\) is a tie path from
  \(a_n\) to \(a_0\).

  Transitive: by \cref{the: tie path connection lemma}.
\end{proof}

\begin{corollary}\label{the: tie component equivalence class}
  For a digraph \(G\), a tie component is an equivalence class
  on \(V(G)\).
\end{corollary}
\begin{proof}
  By \cref{the: tie path equivalence relation},
  and the definition of tie component.
\end{proof}

\begin{corollary}\label{the: tie components partition unique}
  For digraph \(G\), tie components form a unique
  partition of \(V(G)\).
\end{corollary}

\begin{proof}
  By \cref{the: tie component equivalence class},
  tie components form equivalence classes on \(V(G)\),
  and equivalence classes form a unique partition
  of the set \(V(G)\)~\cite{epp_discrete_2011}.
\end{proof}

\begin{corollary}\label{the: adjacent if not in component}
  In a digraph, if vertex \(v\) is not in tie component \(C(a)\),
  then \(v\) is adjacent to \(C(a)\).
\end{corollary}

\begin{proof}
  Suppose \(v\) is not in \(C(a)\).
  Because \(C(a)\) is an equivalence class on the relation
  \(\mathrel{R}\) such that
  \(a\mathrel{R}b\) if there exists a tie path between \(a\) and \(b\).
  two equivalent class \(C(a)\) and \(C(v)\) disjoint~\cite{epp_discrete_2011}.
  Therefore there is no tie path between \(v\) and \(a\),
  thus \(v\) does not tie any vertex in \(C(a)\).

  Therefore, \(v\) is adjacent to every vertex in \(C(a)\).
\end{proof}

Our previous results that ties ``transmit'' arrow directions
in quasi-transitive oriented graphs,
had us see that for a tie component \(C\)
and a vertex \(v \notin C\), \(C \to v\) or \(v \to C\).
See \cref{fig: transmitting arrow direction in tie component},
for a visualization.

\begin{figure}
  \centering
  \begin{subfigure}[b]{0.45\linewidth}
    \centering
    \tikz\graph[simple necklace layout, math nodes, node sep=1cm] {
      b <- a;
      / [label = left:\(C\), draw] // {a};
    };
    \caption{\(a\) beats \(b\).}
  \end{subfigure}
  \begin{subfigure}[b]{0.45\linewidth}
    \centering
    \tikz\graph[simple necklace layout, math nodes, node sep=1cm] {
      a --[dashed] c;
      {a, c} -> b;
      / [label = left:\(C\), draw] // {a, c};
    };
    \caption{direction transmitted from \(a\) to \(c\).}
  \end{subfigure}
  \begin{subfigure}[b]{0.45\linewidth}
    \centering
    \tikz\graph[simple necklace layout, math nodes, node sep=1cm] {
      a --[dashed] c --[dashed] d;
      ""; "";  % chktex 18
      a -> b;
      c -> b;
      d -> b;
      / [label = left:\(C\), draw] // {a, c, d};
    };
    \caption{transmitted from \(a\) to \(c\) to \(d\).}
  \end{subfigure}
  \begin{subfigure}[b]{0.45\linewidth}
    \centering
    \tikz\graph[simple necklace layout, math nodes, node sep=1cm] {
      a --[dashed] c --[dashed] d;
      c --[dashed] e;
      "";  % chktex 18
      a -> b;
      c -> b;
      d -> b;
      e -> b;
      / [label = left:\(C\), draw] // {a, c, d, e};
    };
    \caption{transmitted from \(a\) to \(c\) to \(e\).}
  \end{subfigure}
  \caption{the arrow direction to \(b\) was transmitted in tie component \(C\).}
  \label{fig: transmitting arrow direction in tie component}  % chktex 24
\end{figure}

We then formalize and generalize this idea into two theorems:

\begin{lemma}\label{the: component and a single point}
  In a quasi-transitive oriented graph,
  for any tie component \(C(a)\) and
  any vertex \(v\) such that \(v \notin C(a)\),
  then either \(C(a) \to v\) or \(v \to C(a)\).
\end{lemma}

\begin{proof}
  By \cref{the: adjacent if not in component},
  we know that \(v\) is adjacent to \(C(a)\).
  Let \(x, y\) be two vertices in \(C(a)\).
  There exists a tie path between them,
  so both \(x\) and \(y\) beat \(v\)
  or \(v\) beats both \(x, y\).
  Since this is true for all pairs of vertices on \(C(a)\),
  we get \(v \to C(a)\) or \(C(a) \to v\).
\end{proof}

\begin{corollary}\label{the: vertex force component beating}
  In a quasi-transitive oriented graph,
  for any tie component \(C(a)\),
  If there exists one vertex \(v \in C(a)\)
  that beats (is beaten by) a vertex \(v' \notin C(a)\).
  Then \(C(a) \to v'\) (\(v' \to C(a)\)).
  See \cref{fig: vertex force component beating}
\end{corollary}

\begin{figure}
  \centering
  \begin{subfigure}[b]{0.45\linewidth}
    \centering
    \tikz\graph[simple necklace layout, math nodes, node sep=1cm] {
      a --[dashed] b --[dashed] "\ldots" --[dashed] c --[dashed] d;  % chktex 18
      c --[dashed] e;
      ""; % chktex 18
      b -> f;
      ""; ""; % chktex 18
      / [label = 45:\(C(a)\), draw] // {a, b, "\ldots", c, d, e};  % chktex 18
    };
    \caption{a single vertex in \(C(a)\) beats \(f\).}
  \end{subfigure}
  \begin{subfigure}[b]{0.45\linewidth}
    \centering
    \tikz\graph[simple necklace layout, math nodes, node sep=1cm] {
      a --[dashed] b --[dashed] "\ldots" --[dashed] c --[dashed] d;  % chktex 18
      c --[dashed] e;
      ""; % chktex 18
      {a, b, "\ldots", c, d, e} -> f;  % chktex 18
      ""; ""; % chktex 18
      / [label = 45:\(C(a)\), draw] // {a, b, "\ldots", c, d, e};  % chktex 18
    };
    \caption{component \(C(a)\) has to beat \(f\).}
  \end{subfigure}
  \caption{a vertex \(b\) beats \(f\)
    will force the component \(C(a)\) to beats \(f\).}
  \label{fig: vertex force component beating}  % chktex 24
\end{figure}

\begin{proof}
  Because \(v' \notin C(a)\),
  then either \(v' \to C(a)\) or \(C(a) \to v'\)
  by \cref{the: component and a single point}.

  Case 1, \(v \to v'\): because \(v \in C(a)\),
  then \(v'\) cannot beat any vertex in \(C(a)\).
  Therefore \(C(a) \to v'\).

  Case 2, \(v' \to v\): because \(v \in C(a)\),
  then \(v'\) cannot be beaten by all of \(C(a)\).
  Therefore \(v' \to C(a)\).
\end{proof}

To further generalize \cref{the: component and a single point},
we prove the following theorem:

\begin{theorem}\label{the: tie component beats tie component}
  For any two distinct tie components \(C(a)\) and \(C(b)\)
  in a quasi-transitive oriented graph,
  \(C(a) \to C(b)\) or \(C(b) \to C(a)\).
\end{theorem}

\begin{proof}
  Take any vertex \(a'\) in \(C(a)\).
  By \cref{the: tie components partition unique}, \(a'\) is not in \(C(b)\).
  Then because of \cref{the: component and a single point},
  \(a' \to C(b)\) or \(C(b) \to a'\).

  \begin{itemize}
    \item
      Case 1, \(a' \to C(b)\): take any element \(b' \in C(b)\).
      By \cref{the: vertex force component beating}, \(C(a) \to b'\).
      Therefore \(C(a)\) beats every vertex in \(C(b)\),
      then \(C(a) \to C(b)\).
    \item
      Case 2, \(C(b) \to a'\):
      by the same reasoning, \(C(b) \to C(a)\).
  \end{itemize}
\end{proof}

After several pages of theorems,
we can finally take a pause
and understand what we are saying here.
We combine \cref{the: tie components partition unique} and
\cref{the: tie component beats tie component},
and try to understand it visually.

\begin{figure}
  \centering
  \begin{subfigure}[b]{0.45\linewidth}
    \centering
    \tikz\graph[simple necklace layout, math nodes, node sep=1cm] {
      a; b;
      a -> {c, d, e};
      b -> {c, d, e};
      {c, d, e} -> f;
      {a, b} -> f;
      j -> {c, d, e};
      {a, b} -> j;
      f -> j;
      c -> d --[dashed] e --[dashed] c;
      a --[dashed] b;
    };
    \caption{a quasi-transitive oriented graph.}
  \end{subfigure}
  \begin{subfigure}[b]{0.45\linewidth}
    \centering
    \tikz\graph[simple necklace layout, math nodes, node sep=1cm] {
      a; b; c; d; e; f; j;
      d --[dashed] e --[dashed] c;
      a --[dashed] b;
      / [label = left:\(A\), draw, circle] // {a, b};
      / [label = left:\(B\), draw] // {d, e, c};
      / [label = right:\(C\), draw, circle] // {f};
      / [label = right:\(D\), draw, circle] // {j};
    };
    \caption{find its tie components.}
  \end{subfigure}
  \begin{subfigure}[b]{0.45\linewidth}
    \centering
    \tikz\graph[simple necklace layout, math nodes, node sep=1cm] {
      A [draw, circle, minimum size=1.5cm] -> B [draw, circle, minimum size=1.75cm]
      -> C [draw, circle, minimum size=1cm] -> D [draw, circle, minimum size=1cm];
      A -> D -> B;  % chktex 13
      A -> C;  % chktex 13
    };
    \caption{components beat each other.}
  \end{subfigure}
  \begin{subfigure}[b]{0.45\linewidth}
    \centering
    \tikz\graph[simple necklace layout, math nodes, node sep=1cm] {
      A' -> B' -> C' -> D';
      A' -> D' -> B';
      A' -> C';
    };
    \caption{components are just like vertices.}
  \end{subfigure}
  \caption{condense the tie components and get a tournament.}
  \label{fig: tie components condensation}  % chktex 24
\end{figure}

See \cref{fig: tie components condensation}.
For every quasi-transitive oriented digraph,
we can split it into tie components,
and every tie component either beats another tie component
or is beaten by another tie component.
Thus, tie components behave like vertices.
Because there are no ties between tie components,
The graph formed by tie components is a tournament,
which is the ``most well understood class of
directed graphs''~\cite{bang-jensen_generalizations_1998}.

\chapter{Background}\label{chap: background}

\section{Directed Graph}

  \begin{definition}
    a \keyword{directed graph}, \keyword{digraph} or
    \keyword{graph} consists of a vertex set \(V(G)\),
    and an edge set (ordered pair of vertices) \(E(G)\).
  \end{definition}

  \begin{figure}
    \centering
    \tikz\graph[simple necklace layout, math nodes, node sep=1.75cm] {
        e -> a;
        b -> c;
        a -> b;
        e <-> [bend left] c;
        d;
        a -> [loop above] a;
    };
    \caption{example of a directed graph.}
    \label{fig:digraph example} %chktex 24
  \end{figure}

  For example, in \cref{fig:digraph example},
  if we call this digraph \(G\), then
  the vertex set or \(V(G)\) is \(\set{a, b, c, d, e}\),
  and the edge set or \(E(G)\) is
  \(\set{(a, a), (a, b), (b, c), (c, e), (e, a), (e,c)}\).
  We sometimes refer to ``edges'' as ``arrows''.

\section{Beating Relations}

  To simplify the notation, we think of an edge as a
  \emph{beating relation}:

  \begin{definition}
    In a directed graph \(G\), if \((a, b) \in E(G)\)
    then we say \(a\) \keyword{beats}[beat]
    (\keyword{dominates}[dominate])
    \(b\) or \(a \to b\).
  \end{definition}

  \begin{definition}
    A \keyword{path} or a \keyword{walk}
    from vertex \(a_0\) to vertex \(a_n\)
    is a sequence of vertices \([a_0, a_1, \ldots, a_{n-1}, a_{n}]\)
    such that for all \(0 \leq k < n, ~ a_k \to a_{k+1}\).
    We will sometimes write this path as
    \(a_0 \to a_1 \to \cdots \to a_{n - 1} \to a_n\).
  \end{definition}

  \begin{definition}
    For a given path \(P\), if the sequence has \(n + 1\) vertices,
    then we say the \keyword{length of path} \(P\) is \(n\).
    A special case is a path consisting of one single vertex,
    which is a path of length 0.
  \end{definition}

  The length of a path measures the number of edges on this path.
  For example, in \cref{fig:digraph example},
  there is a path \(P = e \to a \to b \to c\).
  And there are edges \((e, a), (a, b), (b, c)\) on this path,
  therefore the length of this path is 3.
  There is another path \(P' = e \to c\) goes from \(e\) to \(c\).
  This path \(P'\) has length \(1\).
  Therefore, path \(P'\) is shorter than path \(P\),
  and if we go through every possible path from \(e\) to \(c\),
  we can find out that \(P'\) is the \emph{shortest path}
  from \(e\) to \(c\).

  \begin{definition}
    In a directed graph \(G\),
    \keyword{\(a\) beats (dominates) \(b\) by \(n\) steps}[beat by \(n\) steps],
    if the shortest path from \(a\) to \(b\) has length \(n\).
  \end{definition}

  In \cref{fig:digraph example},
  \(e \to a \to b\),
  and \(e\) does not beat \(b\) by 1 step,
  therefore \(e\) beats \(b\) by 2 steps.

  However, although \(e \to a \to b \to c\),
  \(e\) \emph{does not} beat \(c\) by 3 steps,
  because the shortest path from \(e\) to \(c\) is \(e \to c\).
  Therefore, \(e\) beats \(c\) by 1 step.

  \begin{definition}
    In a directed graph, vertex \(a\) is \keyword{adjacent} to vertex \(b\) if
    \(a \to b\) or \(b \to a\) or both.
  \end{definition}

  \begin{definition}
    In a directed graph, for two distinct vertices \(a, b\),
    if \(a\) is not adjacent to \(b\),
    then \(a\) \keyword{ties}[tie] \(b\).
  \end{definition}

  In \cref{fig:digraph example}, \(c\) is adjacent to \(e\);
  \(a\) is adjacent to \(b\); \(b\) is adjacent to \(c\).
  Whereas, \(c\) ties \(a\),
  because there is no edge \((c, a)\) and no edge \((a, c)\)
  in this digraph.

  Since we don't always know the exact structure of the graph,
  it is sometimes useful to look at
  the beating relationships between sets of vertices.
  We then define beating, adjacency, and tie between vertex sets.

  \begin{definition}
    In digraph \(G\), we write \(A \to B\)
    where \(A\) and \(B\) are disjoint subsets of \(V(G)\),
    when every vertex in \(A\) beats every vertex in \(B\).
    We also write \(A \to b\) as a shorthand for \(A \to \set{b}\),
    and \(a \to B\) as a shorthand for \(\set{a} \to B\).
  \end{definition}

  \begin{definition}
    In digraph \(G\), \(A\) is adjacent to \(B\)
    where \(A\) and \(B\) are disjoint subsets of \(V(G)\),
    when every vertex in \(A\) is adjacent to every vertex in \(B\).
    We also write ``\(A\) is adjacent to \(b\)''
    as a shorthand for ``\(A\) is adjacent to \(\set{b}\)'',
    and ``\(a\) is adjacent to \(B\)''
    as a shorthand for ``\(\set{a}\) is adjacent to \(B\)''.
  \end{definition}

  \begin{definition}
    In digraph \(G\), \(A\) ties \(B\)
    where \(A\) and \(B\) are disjoint subsets of \(V(G)\),
    when every vertex in \(A\) ties every vertex in \(B\).
    We also write ``\(A\) ties \(b\)''
    as a shorthand for ``\(A\) ties \(\set{b}\)''
    and ``\(a\) ties \(B\)''
    as a shorthand for ``\(\set{a}\) ties \(B\)''
  \end{definition}

  \begin{figure}
    \centering
    \begin{subfigure}[b]{.45\linewidth}
      \centering
      \tikz\graph[simple necklace layout, math nodes, node sep=1.75cm] {
        {a_1, a_2} -> b;
        b -> {c_1, c_2};
        a_1 -> {a_2, c_2};
        c_1 -> c_2;
      };
      \subcaption{}
    \end{subfigure}
    \begin{subfigure}[b]{.45\linewidth}
      \centering
      \tikz\graph[simple necklace layout, math nodes, node sep=1.75cm] {
        A [draw, circle, minimum size=2cm] -> b -> C [draw, circle, minimum size=2cm];
      };
      \subcaption{}
      \label{fig: condenced beating: condenced subfig} % chktex 24
    \end{subfigure}
    \caption{we can draw graph (a) as graph (b).}
    \label{fig:condenced beating} % chktex 24
  \end{figure}

  In \cref{fig:condenced beating},
  we show how we draw the beating relationship with
  subsets \(A = \set{a_1, a_2}\) and \(C = \set{c_1, c_2}\).
  Notice, in \cref{fig: condenced beating: condenced subfig},
  set \(A\) and set \(C\) are not adjacent.
  This \emph{does not} means that
  set \(A\) ties set \(C\);
  it only means that we have not drawn out
  the relationship between these 2 sets in
  \cref{fig: condenced beating: condenced subfig}.

  \begin{figure}
    \centering
    \begin{subfigure}[b]{.45\linewidth}
      \centering
      \tikz\graph[simple necklace layout, math nodes, node sep=1.75cm] {
        a_1 -> b;
        b -> {c_1, c_2, a_2};
        a_1 -> a_2;
        c_1 -> c_2;
      };
      \caption{}
    \end{subfigure}
    \begin{subfigure}[b]{.45\linewidth}
      \centering
      \tikz\graph[simple necklace layout, math nodes, node sep=1.75cm] {
        A [draw, circle, minimum size=2cm] --  %chktex 8
        b ->
        C [draw, circle, minimum size=2cm] -- [dashed] %chktex 8
        A;  % chktex 13
      };
      \caption{}
    \end{subfigure}
    \caption{we can draw graph (a) as graph (b).}
    \label{fig:condenced tie and adjacency} % chktex 24
  \end{figure}

  We draw adjacency between sets
  (for example, set \(A\) is adjacent to set \(\set{b}\))
  using a solid edge without arrow.
  We draw tie between sets (for example, set \(A\) ties set \(C\))
  using a dashed edge without arrow.
  (see \cref{fig:condenced tie and adjacency})


  \begin{definition}
    The \keyword{submissive set} (\keyword{dominant set})
    of a vertex \(v\) in graph \(G\) is the set of vertices in \(G\)
    that are beaten by \(v\) (beat \(v\)),
    formally it is \(\set{e \in V(G) \mid v \to e}\)
    (or \(\set{e \in V(G) \mid e \to v}|\)).
    We will denote the submissive set of \(v\) as \(S_v\),
    and the dominant set of \(v\) as \(D_v\).
    (See \cref{fig: dominate and submissive set example})
  \end{definition}

  \begin{figure}
    \centering
    \tikz\graph[layered layout, math nodes, grow=right, sibling distance=2cm,level sep=0.75cm] {
        D_v [draw, circle, minimum size=2cm] ->
        v ->
        S_v [draw, circle, minimum size=2cm]
    };
    \caption{\(D_v\) is the dominant set of \(v\);
       \(S_v\) is the submissive set of \(v\)}
    \label{fig: dominate and submissive set example}  % chktex 24
  \end{figure}

  \begin{definition}
    \keyword{Out-degree} (\keyword{in-degree}) of a vertex \(v\) in graph \(G\) is
    the size of the submissive set (dominant set) of \(v\).
  \end{definition}

  For example, in \cref{fig:digraph example},
  \(S_a = \set{a, b}\), therefore vertex \(a\) has out-degree 2;
  \(D_a = \set{e, a}\), therefore vertex \(a\) has in-degree 2.
  Vertex \(d\) has in-degree 0, and out-degree 0
  because \(d\) is not adjacent to any other vertex in the graph.
  Therefore, \(D_d\) and \(S_d\) are both empty sets.

  Notice it is very natural to relate
  out-degree and in-degree with the ``power'' or ``strength''
   of a vertex.
  However, we will discuss in later sections and chapters
  that a vertex with larger out-degree does not necessarily
  have more power, by our definition of ``king'' (defined in \cref{sec:kings}),
  However, the property that ``vertices with largest out-degree
  are kings'' is true in some nice families of graphs
  (see \cref{sec:kings} and
  \cref{sec: semi-complete properties}).


  \begin{definition}
    An \keyword{induced subgraph} or \keyword{vertex-induced subgraph}
    \(H\) of digraph \(G\) is a subgraph of \(G\) such that:
    for all pairs of vertices \(a, b \in V(H)\),
    if \((a, b) \in E(G)\), then \((a, b) \in E(H)\).
  \end{definition}

  \begin{figure}
    \centering
    \begin{subfigure}[b]{.3\linewidth}
      \centering
      \tikz\graph[simple necklace layout, math nodes, node sep=1.75cm] {
        a -> b;
        b -> {c, d, e};
        a -> c;
        d -> e;
      };
      \caption{start with graph \(G\).}
    \end{subfigure}
    \begin{subfigure}[b]{.3\linewidth}
      \centering
      \tikz\graph[simple necklace layout, math nodes, node sep=1.75cm] {
        a; b; c; ""; "";  %chktex 18
      };
      \caption{let \(V(H) = \set{a, b, c}\).} % chktex 44
    \end{subfigure}
    \begin{subfigure}[b]{.3\linewidth}
      \centering
      \tikz\graph[simple necklace layout, math nodes, node sep=1.75cm] {
        a -> {b, c};
        b -> c;
        ""; "";  %chktex 18
      };
      \caption{induced subgraph \(H\).}
    \end{subfigure}
    \caption{the process to create an induced subgraph.}
    \label{fig:induced subgraph example} % chktex 24
  \end{figure}

  In \cref{fig:induced subgraph example},
  we show the process of creating an induced subgraph.
  We redraw the vertices in \(V(H)\), that is \(a, b, c\),
  and then simply copy the edges between those vertices,
  that is, \((a, b), (b, c), (a, c)\) from \(G\) to get \(H\).

  This definition can be viewed in 2 ways:
  \begin{itemize}
    \item
      Take the vertices \(a, b, c\)
      together with the edges between them
      to form subgraph \(H\).
    \item
      Take away vertices \(e, d\)
      together with all the edges
      having \(e\) or \(d\) as an end point,
      the rest of the graph is the subgraph \(H\).
  \end{itemize}

  For all the families of graphs discussed in this paper
  (oriented graph, semi-complete digraph,
  tournament, quasi-transitive oriented graph),
  induced subgraphs preserve the property of the original graph.

\section{Oriented Graph and Tournament}

  \begin{definition}\label{def:oriented graph}
    An \keyword{oriented graph} is a digraph, such that:
    \begin{enumerate}
      \item for all vertices \(a\), \(a\) does not beat itself.
      \item for all pairs of adjacent vertices \(a, b\),
        if \(a\) beats \(b\), then \(b\) does not beat \(a\).
    \end{enumerate}
  \end{definition}

  \begin{figure}
    \centering
    \tikz\graph[layered layout, math nodes, grow=right, sibling distance=2cm,level sep=0.75cm] {
      {a, c}
      -> [bend right] d
      -> b
      -> [bend right] c;
      a -> [bend left] b;
    };
    \caption{example of an oriented graph.}
    \label{fig:oriented graph example} %chktex 24
  \end{figure}

  We show an oriented graph in \cref{fig:oriented graph example}.
  This definition of oriented graph can be viewed as
  a mathematical model of a round-robin tournament with ties allowed.
  A team cannot compete with itself, therefore property 1 holds;
  each pair of teams \(a\) and \(b\) competes exactly once,
  either \(a\) beats \(b\), \(b\) beats \(a\),
  or there is a tie between \(a\) and \(b\).
  therefore property 2 holds.

  Therefore, it is very natural to view an edge
  going from \(a\) to \(b\) as \(a\) ``beats'' \(b\),
  and two non-adjacent vertices \(a\), \(b\) as
  a tie between \(a\) and \(b\).

  \begin{definition}\label{def: tournaments}
    A \keyword{tournament}
    (sometimes called \keyword{semi-complete oriented graph})
    is an oriented graph without ties.
  \end{definition}

  \begin{figure}
    \centering
    \tikz\graph[simple necklace layout, math nodes, node sep=1.75cm] {
        e -> {a, b};
        b -> c -> d -> e;
        a -> d -> b;
        c -> a -> b;
        e -> c
    };
    \caption{example of a tournament.}
    \label{fig:tournament example} % chktex 24
  \end{figure}

  We give an example of a tournament
  in \cref{fig:tournament example}.
  A tournament can be interpreted as,
  what we call in the real world, a round-robin tournament,
  where every pair of vertices compete
  and the competition results in exactly one winner and one loser.

  In~\cite{maurer_king_1980},
  the author uses tournament to model a flock of chicken,
  between every two chickens there is a ``pecking relation''
  (we call it the ``beating relation'').
  Given two chickens,
  one chicken has to peck the other chicken,
  or be pecked by the other chicken, but not both.


\section{Kings}\label{sec:kings}

  \begin{figure}
    \centering
    \tikz\graph[simple necklace layout, math nodes, node sep=1.75cm] {
        e -> {a, b, c};
        b -> c -> d -> e;
        a -> d;
        b -> a;
    };
    \caption{an example with kings \(e\) and \(d\).}
    \label{fig:complicated king example} % chktex 24
  \end{figure}

  \begin{definition}
    a \keyword{king} in a digraph is a vertex that
    beats every other vertex by 1 or 2 steps.
  \end{definition}

  In \cref{fig:complicated king example},
  the kings in this graph are vertices \(e\) and \(d\).
  Vertex \(e\) beats \(a, b, c\) by one step,
  and beats \(d\) by 2 steps, so \(e\) is a king.
  Vertex \(d\) beats \(e\) by one step,
  and \(e\) beats \(a, b, c\),
  so \(d\) beats \(a, b, c\) by 2 steps,
  which makes \(d\) a king.

  The other vertices (\(a, b, c\)) are not kings,
  because \(a\) cannot beat \(c\) by one or two steps
  (\(a \to d \to e \to c\) is a shortest path);
  \(b\) cannot beat \(e\) by one or two steps
  (\(b \to c \to d \to e\) is a shortest path);
  \(c\) cannot beat \(b\) by one or two steps
  (\(c \to d \to e \to b\) is a shortest path).

  There are two points to note in this example:
  \begin{itemize}
    \item
      \cref{fig:complicated king example} shows that
      larger out-degree does not associate with
      more ``power'' or ``strength''.
      Vertex \(d\) only has out-degree 1, and \(d\) is a king.
      Whereas vertex \(b\) has out-degree 2,
      but \(b\) is not a king.
      Vertex \(d\) is more ``powerful'' than \(b\),
      even though \(d\) has smaller out-degree.

    \begin{itemize}
      \item
        A vertex with the \emph{highest} out-degree
        in a digraph may not be a king.
        In \cref{fig: largest out-degree is not king},
        vertex \(a\) has out-degree 3,
        which is the highest out-degree in this graph.
        However, it is not a king,
        since it cannot beat \(k\) by 1 or 2 steps

      \item
        Also, a vertex with \emph{lowest} out-degree
        in a digraph may be a king.
        In \cref{fig: largest out-degree is not king},
        the vertex \(k\) has the smallest out-degree in the graph,
        however it is the only king in this graph.
    \end{itemize}

    \item
      \cref{fig:complicated king example} shows that
      a digraph may have more than one king.
      In fact,~\cite{maurer_king_1980} proved the probability
      that every vertex in a tournament is a king approaches 1
      as the number of vertices in the tournament approaches \(\infty \).
  \end{itemize}

  \begin{figure}
    \centering
    \tikz\graph[tree layout, grow=down, math nodes, sibling distance=2cm,level sep=0.75cm] {
        k -> a;
        a -> {b, c, d};
        c <-> b;
        c <-> d;
        d <-> [bend left] b;
    };
    \caption{the vertex with smallest out-degree is the only king.}
    \label{fig: largest out-degree is not king}  % chktex 24
  \end{figure}

  \begin{definition}
    An \keyword{emperor} in a digraph is a vertex that beats every other vertex.
    (see \cref{fig: emperor example})
  \end{definition}

  \begin{figure}
    \centering
    \tikz\graph[tree layout, math nodes, grow=down, sibling distance=2cm,level sep=0.75cm] {
        e -> {s_1, s_2, s_3, "\ldots", s_n};  %chktex 18
    };
    \caption{vertex \(e\) is the emperor of this graph.}
    \label{fig: emperor example}  % chktex 24
  \end{figure}

  \begin{definition}\label{def: (n k) graphs}
    an \keyword{\((n, k)\) digraph}
    (\keyword{\((n, k)\) oriented graph},
    \keyword{\((n, k)\) tournament}, etc.)
    is a digraph (oriented graph, tournament, etc.)
    with \(n\) vertices and \(k\) kings.
  \end{definition}

  The following work has been done on kings in tournaments.
  These theorems will be useful when we investigate
  properties of kings in generalized tournaments in later chapters.

  \begin{theorem}\label{the: one king iff emperor}
    A tournament has only one king
    if and only if that king is an emperor.~\cite{maurer_king_1980}
  \end{theorem}

  \begin{theorem}\label{the: no 2 kings}
    A tournament cannot have exactly 2 kings.~\cite{maurer_king_1980}
  \end{theorem}

  \begin{theorem}\label{the: largest out-degree is king}
    In a tournament, a vertex with largest out-degree is a king.~\cite{maurer_king_1980}
  \end{theorem}

  \begin{corollary}\label{the: king exists}
    In a tournament, there always exists at least 1 king.~\cite{maurer_king_1980}
  \end{corollary}

  \begin{theorem}\label{the: (n k) tournament exists}
    for all integers \(n \geq k \geq 1\),
    \((n, k)\)-tournaments exist with the following exceptions:
    \((n, 2)\) with any \(n \geq 2\) and  \((4, 4)\).~\cite{maurer_king_1980}
  \end{theorem}







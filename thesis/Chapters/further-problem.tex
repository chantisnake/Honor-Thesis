\chapter{Further Problems}

In \cref{chap: semi-complete digraph} and \cref{chap: general oriented graph},
we see that because of the addition of two types of ties,
we can construct many more \((n, k)\) semi-complete digraphs
or \((n, k)\) oriented graphs than \((n, k)\) tournaments.
Then what can we construct if we limit the number of ties
or double ties?

\begin{definition}
  an \((n, k, t)\) digraph is a digraph with \(n\) vertices,
  \(k\) kings and \(t\) ties (or double ties).
\end{definition}

In \cref{chap: semi-complete digraph}, and \cref{chap: general oriented graph},
we explore the construction of \((n, k)\) semi-complete digraphs
and \((n, k)\) oriented graphs with 1 tie.
In other words, we have solved the problem of constructing
\((n, k, 1)\) oriented graphs and semi-complete digraphs.
What happens when \(t\) gets larger?

One of the useful things to note when approaching the
problem of the existence of \((n, k, t)\) digraphs
(or oriented graphs, semi-complete digraphs, etc.) 
is the construction of a quasi-transitive oriented graph
with given number of kings, which appears at the end of
 \cref{sec: quasi-transitive king}.
This way of constructing quasi-transitive oriented graphs
may be very useful,
since we can manipulate the number of ties in the graph
without changing the number of vertices and kings.
For a tie component with \(n\) vertices,
the number of ties in the component can be any number
between \(n - 1\)
(ties form a spanning tree of the component)
and \(\frac{n(n+1)}{2}\)
(there exists a tie between every two vertices).

In \cref{sec:kings}, we showed that there exist kings with very
low out-degrees.
What is the property of a king with very low out-degree?
What is the property of a king with only out-degree 1?
What is the property of a king that has the lowest out-degree
in the graph?
What is the property of a king that has the lowest out-degree
among all of the kings in the graph?

In \cref{sec: quasi-transitive king},
we showed that a king has a very rich structure.
However, \cref{the: king partitions in quasi-transitive}
and \cref{the: D S of king adjacent in quasi-transitive}
are not used in this thesis.
Can we use these 2 properties to deduce any property
of quasi-transitive oriented graphs with kings?
A interesting path is that
\cref{the: D S of king adjacent in quasi-transitive}
implies that in a quasi-transitive oriented graph with king \(k\),
all the tie components are subsets of \(D_k\) or \(S_k\).

\begin{definition}
  An infinite graph is a graph with an infinite number of vertices.
\end{definition}

Can we extend these result to an infinite graph?
Can we keep the definition of tie components when
we expand the context to infinite graphs?
How will the property of infinite path
and infinite tie path change the results in this thesis?

In~\cite{bang-jensen_kings_1998}
and~\cite{bangjensen_quasitransitive_1995},
the author uses the idea of strong digraph and strong components
(see the definition of strong digraph and strong components
in~\cite{west_introduction_2001})
to deduce the property of quasi-transitive oriented graph.
We can show that given any \emph{non-trivial} condensation,
the uncondensed graph is strong
if and only if its underlying tournament is strong
(\emph{spoiler alert!}:
\cref{the: shortest path different components lemma} will be useful).
Can we combine our results with the result
in~\cite{bang-jensen_kings_1998} and~\cite{bangjensen_quasitransitive_1995}?

In \cref{sec: graph condensation},
we showed that graph condensation is a very useful transformation.
and many graphs have only the identity condensation
and the trivial condensation defined on it.
How hard is a condensation?
How many graphs have only those two condensations on them? 

How useful is graph condensation?
We showed that they preserve the shortest path
(\cref{the: condensation preserves shortest path}),
and we can show that they preserve the ``strong'' property of the graph.
What are other properties that graph condensations preserve?

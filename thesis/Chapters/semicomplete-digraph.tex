\chapter{Semi-complete Digraph}\label{chap: semi-complete digraph}

\section{Definitions}

  In \cref{chap: background},
  we introduced one way to model ``tie'' as
  the non-adjacency between 2 vertices.
  However, if we think of ``beating'' as a weak ordering
  (like ``subset'' relation),
  then we can define ``double tie'' to capture this idea:

  \begin{definition}
    There exists a \keyword{double tie} between
    vertices \(a\) and \(b\) if \(a \to b\) and \(b \to a\).
  \end{definition}

  With the definition of double tie,
  we can then formalize another model of
  round-robin tournament with double ties.
  This kind of graph is called ``semi-complete digraph''.

  \begin{definition}\label{def: semi-complete digraph}
    A \keyword{semi-complete digraph} is a digraph
    where every vertex is adjacent to every other vertex,
    but does not beat itself.
    see \cref{fig:semi-complete digraph example}
  \end{definition}

  This definition says that a semi-complete digraph
  is a digraph such that between each pair of vertices,
  there exists at least one edge between them.

  \begin{figure}
    \centering
    \tikz\graph[simple necklace layout, math nodes, node sep=1.75cm] {
      a -> b;
      a <-> c;
      b -> {c, d};
      c -> e;
      c <-> d;
      d -> {a, e};
      e -> {a, b};
    };
    \caption{an example of semi-complete digraph.}
    \label{fig:semi-complete digraph example} % chktex 24
  \end{figure}

  \begin{definition}
    In a semi-complete digraph,
    if \(a \to b\) and \(b\) does not beat \(a\),
    then vertex \(a\) \keyword{strictly beats}[strictly beat]
    vertex \(b\).
  \end{definition}

  Tournaments are a special case of semi-complete digraphs.
  In tournaments, for each pair of vertices \(\set{a, b}\),
  \(a\) strictly beats \(b\) or \(b\) strictly beaten by \(a\).
  We can view a tournament as a semi-complete digraph
  that only has strict beatings; no double tie allowed.

  \begin{definition}
    In a semi-complete digraph,
    the \keyword{double tie set of vertex \(v\)}[double tie set of vertex]
    is the set of vertices that double ties \(v\).
    We denote the double tie set of \(v\) as \(DT_v\).
  \end{definition}

  For example, in \cref{fig:semi-complete digraph example},
  there is a double tie between \(a\) and \(c\),
  and another double tie between \(c\) and \(d\).
  Every other beating relation between
  any other pair of vertices are ``strict beating'' relations.
  For example, \(a\) strictly beats \(b\),
  \(d\) strictly beats \(e\), and \(b\) strictly beats \(d\).

  For vertex \(c\),
  \(DT_c\) (double tie set of \(c\)) is \(\set{a, d}\).
  The set of vertices that strictly beat \(c\) is \(\set{b}\),
  which can be expressed as \(D_c - DT_c\).
  The set of vertices that are strictly beaten by \(c\) is
  \(\set{e}\),
  which can be expressed as \(S_c - DT_c\).



\section{Properties}\label{sec: semi-complete properties}

  Although tournaments are a special case of semi-complete digraphs,
  many useful properties of tournaments
  are also true for semi-complete digraphs.


  \begin{lemma}\label{the: graph partition lemma}
    For every vertex \(v\) in any semi-complete digraph \(G\),
    \(\set{S_v - DT_v, DT_v, D_v - DT_v, \set{v}}\)
    forms a partition of \(G\).
  \end{lemma}

  \begin{figure}
    \centering
    \tikz\graph[simple necklace layout, math nodes, node sep=2cm] {
      "D_v - DT_v" [draw, circle, minimum size = 2cm] ->  %chktex 18 %chktex 8
      v ->
      "S_v - DT_v"[draw, circle, minimum size = 2cm];  %chktex 18 %chktex 8
      v <-> DT_v [draw, circle, minimum size = 2cm];
    };
    \caption{illustration of \cref{the: graph partition lemma}.}
    \label{fig: graph partition lemma} % chktex 24
  \end{figure}

  \begin{proof}
    \(D_v - DT_v\) is the set of vertices that strictly beat \(v\),
    \(S_v - DT_v\) is the set of vertices that are strictly beaten by \(v\),
    \(DT_v\) is all the vertices that double tie with \(v\).
    They are clearly disjoint by definition,
    and all of them are disjoint to \(\set{v}\) by definition.

    Every vertex in \(V(G)\) needs to be in one of
    the sets \(S_v - DT_v, DT_v, D_v - DT_v, \set{v}\),
    because every other vertex needs to be adjacent to \(v\).
  \end{proof}

  See \cref{fig: graph partition lemma} for a visualization
  for the proof of \cref{the: graph partition lemma}.

  \cref{the: graph partition lemma} and
  \cref{fig: graph partition lemma} are useful
  in proofs of many properties of semi-complete digraphs.

  When we move on to general oriented graphs,
  the lack of this property in oriented graphs will make the
  structure of oriented graphs harder to work with.

  \begin{theorem}\label{the: largest out-degree is a king in semi-complete digraph}
    All vertices with the maximum out-degree
    in a semi-complete digraph are kings.
  \end{theorem}

  \begin{proof}
    Let \(v\) be a vertex with the maximum out-degree
    in a semi-complete digraph \(G\).

    Suppose \(v\) is not a king in \(G\).
    Since \(v\) is not a king and
    \(v\) beats every vertex in \(S_v\) by exactly 1 step,
    then by \cref{the: graph partition lemma},
    there exists a vertex
    \(d \in D_v - DT_v = V(G) - S_v - \set{v}\)
    that is not beaten by \(v\) by 1 or 2 steps.
    Therefore there cannot be any vertex in \(S_v\) that beats \(d\).
    Because \(d\) is adjacent to every vertex in \(S_v\),
    \(d\) strictly dominates \(v\) and \(S_v\).
    Hence \(|S_d| \geq 1 + |S_v|\).
    Therefore \(|S_d| > |S_v|\),
    that is, \(d\) has larger out-degree than \(v\).
    Contradiction.
  \end{proof}

  \begin{figure}
    \centering
    \tikz\graph[simple necklace layout, math nodes, node sep=2cm] {
      a -> {b, c, d};
      b -> {c, e, a};
      c -> {e, d, b};
      d -> {b, a, c};
      e -> {a, d, c};
    };
    \caption{every vertex has the same out-degree.}
    \label{fig: multiple max out-degree} % chktex 24
  \end{figure}

  \cref{fig: multiple max out-degree} shows that
  there can be more than one vertex with maximum out-degree.
  In fact, in this graph,
  every vertex has the same out-degree 3,
  therefore all of the vertices have the maximum out-degree.
  By \cref{the: largest out-degree is a king in semi-complete digraph},
  all vertices in this graph are kings.

  \begin{corollary}\label{the: existence theorem}
    For any non-empty semi-complete digraph \(G\),
    there exists at least one king.
  \end{corollary}

  \begin{proof}
    This corollary is a result of \cref{the: largest out-degree is a king in semi-complete digraph}.
    because out-degrees are non-negative integers,
    and \(V(G)\) is not empty.
    Thus there exists at least one vertex with maximum out degree.
  \end{proof}

  \begin{theorem}\label{the: beaten by king theorem}
    In a semi-complete digraph \(G\),
    every vertex with a non-empty dominant set
    is beaten by a king.
  \end{theorem}

  \begin{proof}
    Let \(v\) be a vertex in \(G\), such that \(D_v\) is not empty.
    Consider the subgraph induced by the vertices in \(D_v\);
    this subgraph is also a semi-complete digraph.
    By \cref{the: existence theorem},
    there is a king \(k\) in the induced subgraph of \(D_v\).
    We will show that \(k\) is also a king of \(G\).
    \begin{itemize}
      \item
        \(k\) dominates \(D_v\) by 1 or 2 steps,
        because \(k\) is a king in \(D_v\).
      \item
        \(k\) dominates \(v\) by exactly 1 step,
        because \(k\) is in \(D_v\).
      \item
        \(k\) dominates \(DT_v\) and \(S_v\) by 1 or 2 steps,
        because \(k\) dominates \(v\),
        which beats all vertices in \(DT_v\) and \(S_v\).
    \end{itemize}

    Then, by \cref{the: graph partition lemma},
    we know that \(k\) dominates
    every vertex in the graph by 1 or 2 steps.
    Therefore \(k\) is a king in \(G\) and \(k\) beats \(v\).
  \end{proof}

  \begin{corollary}\label{the: if only king then emperor}
    If a semi-complete digraph has only one king,
    then that king is an emperor.
  \end{corollary}

  \begin{proof}
    Suppose \(G\) is a semi-complete digraph
    with only one king \(k\), and \(k\) is not an emperor.

    Then there exists \(v\) in the graph,
    such that \(k\) does not beat \(v\).
    Therefore, \(v \to k\), since there is no tie in the graph.
    Therefore, \(k\) has a non-empty domiant set.
    Then by \cref{the: beaten by king theorem},
    \(k\) is beaten by another king.
    Therefore there exists more than one king. Contradiction.
  \end{proof}

  \begin{figure}
    \centering
    \tikz\graph{
      a <-> b;
    };
    \caption{both vertices \(a\) and \(b\)
       are emperors and kings.}
    \label{fig:more than one emperors}  %chktex 24
  \end{figure}
  Notice, unlike in tournaments,
  the converse of \cref{the: if only king then emperor} is not true.
  In \cref{fig:more than one emperors},
  we show we can have more than one king\
  with some kings being emperors.

  \begin{theorem}\label{the: (n k) digraph exists}
    \((n, k)\) semi-complete digraphs
    exist for all \(n \geq k \geq 1\), where \(n, k\) are integers.
  \end{theorem}

  \begin{proof}
    We prove this theorem by induction.
    Construct an \((n, k)\) semi-complete digraph:
    \begin{itemize}
      \item
        When there is only one vertex in the graph
        then the graph is a \((1, 1)\) semi-complete digraph.

      \item
        See \cref{fig:add a non-king vertex}.
        We can add one vertex that is not a king by
        adding a vertex that is strictly beaten
        by all the vertices in the original graph.
        In other words,
        we can construct
        a \((n + 1, k)\) semi-complete digraph
        from any \((n, k)\) semi-complete digraph.

        \begin{figure}
          \centering
          \tikz\graph[layered layout, grow=right, level sep=1cm] {
            " \((n, k)\) digraph " [draw, circle] -> "\(v\)"; %chktex 18
          };
          \caption{an \((n+1, k)\) semi-complete digraph by
            adding a new vertex \(v\)}
          \label{fig:add a non-king vertex}  %chktex 24
        \end{figure}

      \item
        See \cref{fig:add a king}.
        We can add one king by
        adding one vertex that double ties
        all the vertices in the original graph.
        In other words, we can construct
        an \((n + 1, k + 1)\) semi-complete digraph
        from any \((n, k)\) semi-complete digraph.

        \begin{figure}
          \centering
          \tikz\graph[layered layout, grow=right, level sep=1cm] {
            " \((n, k)\) digraph " [draw, circle] <-> "\(a\)"; %chktex 18
          };
          \caption{an \((n+1, k+1)\) semi-complete digraph by
            adding a new king \(a\).}
          \label{fig:add a king}  %chktex 24
        \end{figure}

    \end{itemize}

    Therefore, we can obtain
    any \((n, k)\) semi-complete digraph by:
    starting with a \((1,1)\) semi-complete digraph,
    first add \(k - 1\) kings to
    get a \((k, k)\) semi-complete digraph;
    then add \(n - k\) non-king vertices
    to get an \((n, k)\) semi-complete digraph.
  \end{proof}

  \begin{figure}
    \centering
    \begin{subfigure}[b]{.45\linewidth}
      \centering
      \tikz\graph[simple necklace layout, math nodes, node sep=1.75cm] {
        a;  %chktex 18
      };
      \caption{start with \((1, 1)\) digraph.}
    \end{subfigure}
    \begin{subfigure}[b]{.45\linewidth}
      \centering
      \tikz\graph[simple necklace layout, math nodes, node sep=1.75cm] {
        a <-> b;
      };
      \caption{add king \(b\).}
    \end{subfigure}
    \begin{subfigure}[b]{.45\linewidth}
      \centering
      \tikz\graph[simple necklace layout, math nodes, node sep=1.75cm] {
        a <-> b;
        {a, b} -> c;
      };
      \caption{add non-king \(c\).}
    \end{subfigure}
    \begin{subfigure}[b]{.45\linewidth}
      \centering
      \tikz\graph[simple necklace layout, math nodes, node sep=1.75cm] {
        a <-> b;
        {a, b} -> c;
        {a, b, c} -> d;
      };
      \caption{add non-king \(d\).}
    \end{subfigure}
    \caption{to construct a \((4, 2)\) semi-complete digraph.}
    \label{fig: (4 2) digraph construction} % chktex 24
  \end{figure}

  In \cref{fig: (4 2) digraph construction},
  we give an example of how to construct a
  \((4,2)\) semi-complete digraph
  using the inductive algorithm we introduced in
  \cref{the: (n k) digraph exists}:
  \begin{enumerate}
    \item
      Start with a single vertex,
      which is a \((1, 1)\) semi-complete digraph.
    \item
      Add a king \(b\), by letting it double tie with \(a\)
      which gives us a \((2, 2)\) semi-complete digraph.
    \item
      Add a non-king \(c\),
      by letting every vertex in the
      \((2,2)\) semi-complete digraph beat \(c\).
      which gives us a \((3, 2)\) semi-complete digraph.
    \item
      Add a non-king \(d\),
      by letting every vertex
      in the \((3, 2)\) semi-complete digraph beat \(d\),
      which gives us a \((4, 2)\) semi-complete digraph.
  \end{enumerate}

  Using the method in \cref{the: (n k) digraph exists},
  when we add the \(i_{th}\) king,
  we are creating \((i - 1)\) ties,
  so adding \(k\) kings would require a total of
  \(1 + 2 + \cdots + (k - 1)\) double ties.
  The summation simplifies to \(\frac{k(k-1)}{2}\).
  Thus, the number of double ties blows up with
  the increase of \(k\) (number of kings).

  Therefore we want to investigate how to use double ties
  more ``efficiently'':
  what is the minimum number of double ties needed to get an
  \((n, k)\) semi-complete digraph.

  \begin{lemma}\label{the: (n 2) digraph with one tie}
    All \((n, 2)\) semi-complete digraphs with \(n \geq 2\)
    can be constructed with only one double tie.
  \end{lemma}

  \begin{proof}
    use the method described in \cref{the: (n k) digraph exists}.

    First add 1 king to a \((1, 1)\) semi-complete digraph
    to get a \((2, 2)\) semi-complete digraph
    (one double tie added).
    Then add \((n - 2)\) non-king vertices to get
    an \((n, 2)\) semi-complete digraph (no double tie added).

    Therefore we add only one double tie to construct
    any \((n, 2)\) semi-complete digraph.
  \end{proof}

  \begin{lemma}\label{the: (4 4) digraph exists with one tie}
    There exists a \((4, 4)\) semi-complete digraph
    with only one double tie.
  \end{lemma}

  \begin{proof}
    \begin{figure}
      \centering
      \tikz\graph[simple necklace layout, math nodes, node sep=1.75cm]{
        a -> {b, d};
        b -> d -> c;
        c -> a;
        c <-> b;
      };
      \caption{a \((4,4)\) semi-complete digraph
        with only one double tie}
      \label{fig: (4 4) digraph with one tie}  %chktex 24
    \end{figure}
    \cref{fig: (4 4) digraph with one tie} gives an example
    of a \((4,4)\) semi-complete digraph with only one double tie.
    The only double tie is between \(b\) and \(c\).
  \end{proof}

  \begin{theorem}
    for all \(n \geq k \geq 1\),
    there exists an \((n, k)\) semi-complete digraph with
    at most one double tie.
  \end{theorem}

  \begin{proof}
    Recall \cref{the: (n k) tournament exists},
    there exists an \((n, k)\) tournament
    with the exception of \((n, 2)\) and \((4, 4)\).
    Because tournaments are also semi-complete digraphs,
    therefore we can use \cref{the: (n k) tournament exists}
    along with \cref{the: (n 2) digraph with one tie}
    and \cref{the: (4 4) digraph exists with one tie}
    to get the result.

    The proofs of \cref{the: (n k) tournament exists},
    \cref{the: (n 2) digraph with one tie},
    and \cref{the: (4 4) digraph exists with one tie}
    all provide constructions for these semi-complete digraphs.
    Therefore we can construct an \((n, k)\)
    semi-complete digraph with at most one double tie,
    for all \(n \geq k \geq 1\).
  \end{proof}



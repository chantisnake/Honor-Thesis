\chapter{Introduction}

\section{Summary of Previous Works}

In~\cite{maurer_king_1980}, the author introduced
the idea of kings (or 2-kings) in tournaments,
and proved that kings in tournaments have many nice properties.
In the end of~\cite{maurer_king_1980},
the author proposed several ways to generalize kings,
one of them is \(s\)-kings where \(s\) is a positive integer.
When we say ``king'' in a graph theory context,
we usually mean 2-king.
In~\cite{reid_every_1982}, the author
further investigated the properties of kings in tournaments,
and proved many interesting theorem about them.

With the more and more understanding for the properties
of kings in tournaments,
people begin to curious about generalized tournaments.

In~\cite{bangjensen_quasitransitive_1995}, the author
proposed a generalization of tournaments called
quasi-transitive digraphs (a super set of
quasi-transitive oriented graphs studied in \cref{chap: quasi-transitive}).
Then the same author investigated the properties
of 3-kings in quasi-transitive digraph
in~\cite{bang-jensen_kings_1998}.
And~\cite{galeana-sanchez_existence_2013}
generalized the results in~\cite{bang-jensen_kings_1998}
to \((k+1)\)-kings and \(k\)-quasi-transitive digraphs.

In~\cite{bang-jensen_generalizations_1998},
the author surveys various types of generalized tournaments,
including semi-complete digraphs, oriented graphs and
quasi-transitive digraphs.
However, this article did not focus on the properties
of kings in these generalizations of tournaments.


\section{Structure of This Thesis}

\begin{figure}
  \centering
  \begin{tikzpicture}

    \begin{scope}[shift={(3cm,-5cm)}, fill opacity=0.5,
      mytext/.style={text opacity=1, font=\large\bfseries}]

      \draw[draw = black,name path=circle 1] (-1.5,0) circle (3);
      \draw[draw = black,name path=circle 2] (1.5,0) circle (3);

      \pgftransparencygroup
      \clip (-1.5,0) circle (3);
      \fill[gray] (1.5,0) circle (3);
      \filldraw[draw, fill=gray, name intersections={of=circle 1 and circle 2}] (intersection-1) .. controls +(-4,1) and +(-4,-1) ..(intersection-2);
      \endpgftransparencygroup

      \node[mytext] at (-3.8,0) (B) {O};
      \node[mytext] at (3.8,0) (C) {S};
      \node[mytext] at (0,0) (D) {T};
      \node[mytext] at (-2,1) (E) {Q};
    \end{scope}
  \end{tikzpicture}
  \caption{the relationship between tournaments,
    semi-complete digraphs, oriented graphs,
    and quasi-transitive oriented graphs.}
  \label{fig: generalized tournaments relationship}  % chktex 24
\end{figure}

The core content of this paper is in
\cref{chap: background}, \cref{chap: semi-complete digraph},
\cref{chap: general oriented graph},
and \cref{chap: quasi-transitive}.
The latter three chapters focused on kings in
three generalization of tournaments:
semi-complete digraph (see \cref{def: semi-compelete digraph}),
oriented graph (see \cref{def:oriented graph}),
and quasi-transitive oriented graph
(see \cref{def: quasi-transitive oriented graph}),
respectively.

It is helpful to know the relation of these
three generalizations of tournaments.
In \cref{fig: generalized tournaments relationship},
we show the relationship between these graphs.
The ``O'' represents the set of all oriented graphs;
the ``Q'' represents the set of all quasi-transitive oriented graphs;
the ``T'' represents the set of all tournaments;
the ``S'' represents the set of all semi-compelete digraphs.
tournaments are a subset of all three generalized tournaments.
Quasi-transitive oriented graphs are a subset of oriented graphs,
and the intersection between oriented graphs and
semi-compelete digraphs is exactly all the tournaments
(hence the name ``semi-compelete oriented graph'',
see \cref{def: tournaments}).


In \cref{chap: background}, we discussed some of the terminology
and previously-proven theorems that are used in other chapters.
\cref{chap: semi-complete digraph}, \cref{chap: general oriented graph},
and \cref{chap: quasi-transitive} are all independent from each other.
We recommend the readers to start with \cref{chap: background},
and then choose the chapter that interest them
the most to continue reading.

\cref{chap: semi-complete digraph} and \cref{chap: general oriented graph}
are two relatively easy chapter, comparing to \cref{chap: quasi-transitive}.
These two chapters mainly focused on constructing
\((n, k)\) semi-complete digraphs and
\((n, k)\) oriented graph (see \cref{def: (n k) graphs}).

\cref{chap: quasi-transitive} is the most interesting chapter.
Unfortunately, it is harder than
\cref{chap: semi-complete digraph} and \cref{chap: general oriented graph}.
This chapter is proof heavy and
moves at a faster pace than the previous two chapters.
In this chapter we proved the four most important results
in this thesis; namely
\cref{the: tie condensation results in tournament},
\cref{the: tie component condensation unique}
(these 2 results can be merged into one),
\cref{the: tie condensation effcient},
and \cref{the: king in quasi-transitive}.

\chapter{Oriented Graph}\label{chap: general oriented graph}

The properties of semi-complete digraphs basically inherit
the properties of tournaments,
and most of the proofs are almost the same.

We now move on to another family of graphs that are less
similar to tournaments: oriented graphs.
We will investigate general properties of oriented graphs
in this chapter.
In later \cref{chap: quasi-transitive}, we will focus on
specific families of oriented graphs.

Recall \cref{def:oriented graph},
a oriented graph is a digraph that doesn't
have self-loops (a vertex beats itself)
and double ties.

\begin{theorem}\label{the: emperor then only king in oriented graph}
  If there is an emperor \(k\) in an oriented graph,
  then \(k\) is the only king in the graph.
\end{theorem}
\begin{proof}
  The emperor \(k\) beats every vertex in one step,
  therefore it is a king.

  Because there is no double ties in oriented graph,
  if for every vertex \(k \to v\),
  then \(v\) does not beat \(k\).
  Therefore no vertex can beat \(k\),
  then no vertex can beat \(k\) in one or two steps.
  Therefore, \(k\) is the only king in the graph.
\end{proof}

\begin{corollary}\label{the: if vertex with out-degree n-1 then only one king}
  In an oriented graph with \(n\) vertices,
  if there exists a vertex with out-degree \(n-1\),
  then there is only one king in the graph.
\end{corollary}

\begin{proof}
  A vertex with out-degree \(n-1\) needs to beat every vertex
  in the graph except itself.
  Therefore That vertex is an emperor,
  and because of \cref{the: emperor then only king in oriented graph},
  there will be only one king in this graph.
\end{proof}

\begin{lemma}\label{the: add edge only add king}
  For a oriented graph \(G\),
  if we add a new edge into \(G\) to get \(G'\),
  then \(G'\) will not have less kings than \(G\).
\end{lemma}

\begin{proof}
  Need to show that for every king \(k\) in \(G\),
  \(k\) is also a king in \(G'\).

  If \(k\) beats a vertex \(v\) by one step in \(G\),
  then \(k\) beats vertex \(v\) by one step in \(G'\),
  because if edge \((k, v)\) in \(G\) then
  edge \((k, v)\) in \(G'\).

  If \(k\) beats a vertex \(v\) by 2 steps in \(G\),
  then exists vertex \(a\) such that
  \(k \to a \to v\) in \(G'\).
  By previous reasoning, \(k \to a \to v\) in \(G\),
  then \(k \to a \to v\) in \(G'\)
  and \(k\) beats \(v\) by 2 steps in \(G'\).

  Therefore, \(k\) is also a king in \(G'\),
  then every king in \(G\) is preserved in \(G'\).
  Therefore, \(G'\) have less king than \(G\).
\end{proof}

\begin{lemma}\label{the: no (1 0) oriented graph}
  There do not exists a \((1, 0)\) oriented graph.
\end{lemma}
\begin{proof}
  If any oriented graph only have 1 vertex,
  then that vertex by definition is a king.
  Therefore, there does not exist a \((1, 0)\) oriented graph.
\end{proof}

\begin{lemma}\label{the: no (2 2) oriented graph}
  There does not exist a \((2, 2)\) oriented graph
\end{lemma}

\begin{proof}
  \begin{figure}
    \centering
    \begin{subfigure}{0.45\linewidth}
      \centering
      \tikz\graph[layered layout, math nodes, grow=right, sibling distance=2cm, level sep=0.75cm] {
      a; b;
      };
      \caption{2 vertices with no edge.}
      \label{fig: all oriented graph with 2 vertices: no edge}  % chktex 24
    \end{subfigure}
    \begin{subfigure}{0.45\linewidth}
      \centering
      \tikz\graph[layered layout, math nodes, grow=right, sibling distance=2cm, level sep=0.75cm] {
      a -> b;
      };
      \caption{2 vertices with 1 edge.}
      \label{fig: all oriented graph with 2 vertices: 1 edge}  % chktex 24
    \end{subfigure}
    \caption{all the oriented graph with 2 vertices.}
    \label{fig: all oriented graph with 2 vertices}  % chktex 24
  \end{figure}
  \cref{fig: all oriented graph with 2 vertices}
  shows all the possible oriented graphs with 2 vertices.
  Notice, the oriented graph in
  \cref{fig: all oriented graph with 2 vertices: no edge}
  has 0 king,
  and the oriented graph in
  \cref{fig: all oriented graph with 2 vertices: 1 edge}
  has only 1 king.
  Therefore, there does not exists a \((2,2)\) oriented graph.
\end{proof}
\begin{lemma}\label{the: no (3 2) oriented graph}
  There does not exist a \((3, 2)\) oriented graph
\end{lemma}

\begin{proof}
  We consider all the oriented graph with 3 vertices.

  First, consider the graph with maximum out-degree 0.
  The graph with maximum out-degree 0 will have no edge,
  therefore there cannot be a king,
  since no vertex can beats other vertices by 2 steps.

  \begin{figure}
    \centering
    \begin{subfigure}{0.3\linewidth}
      \centering
      \tikz\graph[simple necklace layout, math nodes, node sep=1cm] {
      a -> b; c;
      };
      \caption{3 vertices with no edge.}
      \label{fig: 3 oriented graph with max out-degree 1: 1 edge}  % chktex 24
    \end{subfigure}
    \begin{subfigure}{0.3\linewidth}
      \centering
      \tikz\graph[simple necklace layout, math nodes, node sep=1cm] {
      a -> b -> c;
      };
      \caption{3 vertices with 1 edge.}
      \label{fig: 3 oriented graph with max out-degree 1: 2 edge}  % chktex 24
    \end{subfigure}
    \begin{subfigure}{0.3\linewidth}
      \centering
      \tikz\graph[simple necklace layout, math nodes, node sep=1cm] {
      a -> b -> c -> a;
      };
      \caption{3 vertices with 3 edges.}
      \label{fig: 3 oriented graph with max out-degree 1: 3 edge}  % chktex 24
    \end{subfigure}
    \caption{all the oriented graph with 3 vertices and maximum out-degree 1.}
    \label{fig: 3 oriented graph with max out-degree 1}  % chktex 24
  \end{figure}

  Then, consider the graph with maximum out-degree 1.
  In \cref{fig: 3 oriented graph with max out-degree 1},
  We show every oriented graph with 3 vertices and
  maximum out-degree 1.
  In \cref{fig: 3 oriented graph with max out-degree 1: 1 edge},
  there is no king;
  in \cref{fig: 3 oriented graph with max out-degree 1: 2 edge},
  there is 1 king;
  in \cref{fig: 3 oriented graph with max out-degree 1: 3 edge},
  every vertex is a king, therefore it has 3 kings.

  Finally, we consider the oriented graph with maximum out-degree 2.
  By \cref{the: if vertex with out-degree n-1 then only one king},
  because we have 3 vertex in the graph,
  and at least one vertex need to have degree 2,
  there there can only be 1 king.

  Therefore, there is no \((3,2)\) oriented graph.
\end{proof}

\begin{lemma}\label{the: no (4 4) oriented graph}
  There does not exist a \((4,4)\) oriented graph.
\end{lemma}
\begin{proof}
  Assume there exists a \((4, 4)\) oriented graph,
  then according to \cref{the: add edge only add king},
  we can keep adding edge until every edge is adjacent
  and still remain 4 vertices and 4 kings.
  Then there exists a \((4,4)\) tournament.
  By \cref{the: (n k) tournament exists},
  there do not exists \((4,4)\) tournaments.

  Therefore, there cannot exists \((4,4)\) oriented graph.
\end{proof}

After investigating what kinds of oriented graph do not exist,
we then move on to all the possible oriented graph that
we can construct.

\begin{lemma}\label{the: (n 2) oriented graph}
  There exists an \((n, 2)\) oriented graph for \(n \geq 4\).
\end{lemma}

\begin{proof}
  \begin{figure}
    \centering
    \tikz\graph[tree layout, grow=down, math nodes, sibling distance=1cm,level sep=1cm] {
    a -> b -> {c_1, c_2, "\ldots", c_{n-3}, c_{n-2}};  %chktex 18
    c_1 -> a;
    };
    \caption{only \(a\) and \(b\) are kings for \(n \geq 4\).}
    \label{fig: (n 2) oriented graph}  %chktex 24
  \end{figure}
  We can see in \cref{fig: (n 2) oriented graph} that
  \(c_1\) cannot dominate \(c_2\) by 1 or 2 steps,
  and other \(c_i (i \neq 1)\) cannot dominate any other vertex,
  because the out-degree of these vertices are 0.

  \(a\) is a king because \(a \to b\) and \(b\) beats every \(c_i\),
  therefore \(a\) beats \(b\) by one step,
  and \(a\) beats every \(c_i\) by 2 steps.
  \(b\) beats every \(c_i\) by 2 steps,
  and \(b\) beats \(a\) by 2 steps: \(b \to c_1 \to a\).
\end{proof}

\begin{lemma}\label{the: (n 0) oriented graph}
  There exists \((n, 0)\) oriented graph for \(n \geq 0\)
  except \(n = 1\)
\end{lemma}

\begin{proof}
  For a oriented graph with \(n\) vertices with no edge,
  every vertex cannot beat any other vertex,
  therefore, the graph has 0 kings,
  and the graph is a \((n, 0)\) oriented graph.
\end{proof}

\begin{theorem}\label{the: (n k) oriented graph}
  There exists an \((n, k)\) oriented graph for all \(n \geq k \geq 0\),
  with the exception of \((1, 0)\), \((2, 2)\), \((3, 2)\),
  and \((4, 4)\) oriented graph.
\end{theorem}

\begin{proof}
  \cref{the: (n k) tournament exists} shows that
  there exists an \((n, k)\) tournament for all \(n \geq k \geq 1\)
  with the exception of \((n, 2)\), and \((4, 4)\).

  Because tournaments are also oriented graphs and
  by \cref{the: no (2 2) oriented graph},
  \cref{the: no (3 2) oriented graph},
  \cref{the: no (1 0) oriented graph},
  \cref{the: no (4 4) oriented graph},
  \cref{the: (n 2) oriented graph},
  and \cref{the: (n 0) oriented graph},
  we can conclude that the theorem is correct.
\end{proof}

We generalize the result from~\cite{maurer_king_1980}
on tournaments to oriented graphs
and show that there are only 4
\((n, k)\) oriented graphs that do not exists.

Following the idea from \cref{chap: semi-complete digraph},
one of the questions to ask is how can we use
ties more ``efficiently''.
The construction method in the proof of
\cref{the: (n 2) oriented graph} is very inefficient.

Here we present a better way to construct these oriented graphs.

\begin{lemma}\label{the: (n 2) with one tie}
  There exists an \((n, 2)\) oriented graph
  with only one tie, for \(n \geq 4\).
\end{lemma}

\begin{proof}
  \begin{figure}
    \centering
    \tikz\graph[tree layout, math nodes, grow=down, sibling distance=2cm,level sep=0.75cm] {
    a -> b -> {c, T_{n-3} [draw, circle]};
    T_{n-3} <- c;
    T_{n-3} -> [bend right] a;
    };
    \caption{the constructive proof for
    \cref{the: (n 2) with one tie}}
    \label{fig: (n 2) with one tie} %chktex 24
  \end{figure}

  See \cref{fig: (n 2) with one tie},
  \(T_{n - 3}\) is a tournament of \(n - 3\) vertices.
  In this graph, the only tie is between \(a\) and \(c\)
  and the only kings are \(a\) and \(b\).

  \(a\) is a king because, \(a \to b \to c\)
  and \(a \to b \to T_{n - 3}\),
  therefore \(a\) beats \(b\) by 1 step
  and \(a\) beats \(c\) and \(T_{n-3}\) by 2 steps.
  \(b\) is a king because, \(b \to c\)
  and \(b \to T_{n - 3} \to a\)
  (because \(T_{n-3}\) is not empty),
  therefore \(b\) beats \(c\) and \(T_{n - 3}\),
  and \(b\) beats \(a\) by 2 steps.

  \(c\) is not a king, there is no path from \(c\) to \(b\).
  Any vertex \(v\) in \(T_{n-3}\) cannot be a king,
  because the closest path between \(v\) and \(c\)
  is \(v \to a \to b \to c\) which has length 3.
\end{proof}

\begin{lemma}\label{the: (n 0) oriented graph with 1 tie}
  There exists \((n, 0)\) oriented graph where \(n \neq 1\)
  with at most 1 tie.
\end{lemma}
\begin{proof}
  First, we can see that \((0, 0)\) oriented graph exists,
  it is just an empty graph with no vertex and edge.

  \begin{figure}
  \centering
    \tikz\graph[layered layout, math nodes, grow=right, sibling distance=2cm, level sep=0.75cm] {
      a; b;
    };
    \caption{\((2, 0)\) oriented graph.}
    \label{fig: (2 0) oriented graph with one tie}  % chktex 24
  \end{figure}
  Then, in \cref{fig: (2 0) oriented graph with one tie}
  we show that \((2, 0)\) oriented graph with one tie exists.

  \begin{figure}
  \centering
    \tikz\graph[layered layout, grow=right, sibling distance=2cm, level sep=2cm] {
      "\((n, 0)\) oriented graph" [draw, circle]  %chktex 18
      -> "\(s\)";  %chktex 18
    };
    \caption{construct \((n+1, 0)\) oriented graph from \((n, 0)\) oriented graph.}
    \label{fig: (n+1 0) oriented graph with one tie}  % chktex 24
  \end{figure}
  In \cref{fig: (n+1 0) oriented graph with one tie},
  we give a way to construction of an \((n+1, 0)\) oriented graph
  given an \((n, 0)\) oriented graph.
  We denote the \((n, 0)\) oriented graph as \(G\).

  We need to show the resulting graph
  in \cref{fig: (n+1 0) oriented graph with one tie}
  has no king:
  first, \(s\) cannot be a king,
  because it does not beat any vertex;
  any vertex \(v \in V(G)\) cannot be a king too.
  Because \(v\) is not a king in \(G\),
  there exists a vertex \(v' \in V(G)\)
  such that \(v\) cannot beat it in 1 or 2 steps in \(G\).
  Then \(v\) still cannot beats \(v'\) in this graph,
  because \(s\) is the only added vertex,
  and \(v \to s \to v'\) do not exists.
\end{proof}

With \cref{the: (n 2) with one tie} and
\cref{the: (n 0) oriented graph with 1 tie},
we can prove the following theorem.

\begin{theorem}\label{the: (n k) oriented graph with one tie}
  There exists an \((n, k)\) oriented graph
  with at most one tie for all \(n \geq k \geq 0\),
  with the exception of \((1, 0)\), \((2, 2)\),
  \((3, 2)\), and \((4, 4)\) oriented graphs.
\end{theorem}

\begin{proof}
  Almost the same proof as \cref{the: (n k) oriented graph},
  just substitute \cref{the: (n 2) oriented graph},
  \cref{the: (n 0) oriented graph} with
  \cref{the: (n 2) with one tie}
  and \cref{the: (n 0) oriented graph with 1 tie},
  respectively.
\end{proof}

in other words,
\cref{the: (n k) oriented graph with one tie},
states that for all \((n, k)\) oriented graphs
that exists, they can be constructed with only one tie.

\section{Definitions}

As mentioned in \cref{chap: general oriented graph}
In an oriented graph, for any pair of vertices \(a, b\),
either \(a\) beats \(b\), \(b\) beats \(a\)
or \(a\) ties \(b\).
So, there are \(3^\binom{n}{2}\) different oriented graphs
on n vertices.
with no other restrictions, it is difficult to
produce theorems about them.
Therefore we decided to focus on another special case of oriented graph,
which is called ``quasi-transitive oriented graph'':

\begin{definition}\label{def: quasi-transitive oriented graph}
  a \keyword{quasi-transitive oriented graph} \(G\) is
  an oriented graph such that,
  for all vertices \(a, b, c \in V(G)\), if \(a \to b \to c\),
  then \(a\) is adjacent to \(c\).
  See \cref{fig: quasi-transitive oriented graph}
\end{definition}

\begin{figure}
  \centering
  \begin{subfigure}[b]{0.45\linewidth}
    \centering
    \tikz\graph[simple necklace layout, math nodes, node sep=2cm] {
      b <- a;
      b -> c;
      a -> c;
    };
  \end{subfigure}
  \begin{subfigure}[b]{0.45\linewidth}
    \centering
    \tikz\graph[simple necklace layout, math nodes, node sep=2cm] {
      b <- a;
      b -> c;
      a <- c;
    };
  \end{subfigure}
  \caption{if \(a \to b \to c\) then \(a\) is adjacent to \(c\).}
  \label{fig: quasi-transitive oriented graph}  %chktex 24
\end{figure}

Because every vertex is adjacent to every other vertex
 in a tournament,
tournaments is a special case of quasi-transitive oriented graph.

Although tournaments are just a small subset
of quasi-transitive oriented graphs,
and many quasi-transitive oriented graph are very different
from tournaments
(for example, every graph with no edge is
 a quasi-transitive oriented graph),
quasi-transitive oriented graphs inherits many properties of
tournaments~\cite{bangjensen_quasitransitive_1995}.

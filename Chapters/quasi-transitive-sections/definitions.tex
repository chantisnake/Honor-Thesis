\section{Definitions}

\cref{chap: general oriented graph}
is a rather short chapter,
because the behavior of oriented graph is
so unpredictable that it is hard to produce theorems about them.
Therefore we decided to focus on a special case of oriented graph,
which is called ``quasi-transitive oriented graph'':

\begin{definition}
  a \keyword{quasi-transitive oriented graph} \(G\) is
  an oriented graph such that,
  for all vertices \(a, b, c \in V(G)\), if \(a \to b \to c\),
  then \(a\) is adjacent to \(c\).
  See \cref{fig: quasi-transitive oriented graph}
\end{definition}

\begin{figure}
  \centering
  \begin{subfigure}[b]{0.45\linewidth}
    \centering
    \tikz\graph[simple necklace layout, math nodes, node sep=2cm] {
      b <- a;
      b -> c;
      a -> c;
    };
  \end{subfigure}
  \begin{subfigure}[b]{0.45\linewidth}
    \centering
    \tikz\graph[simple necklace layout, math nodes, node sep=2cm] {
      b <- a;
      b -> c;
      a <- c;
    };
  \end{subfigure}
  \caption{if \(a \to b \to c\) then \(a\) is adjacent to \(c\)}
  \label{fig: quasi-transitive oriented graph}  %chktex 24
\end{figure}

Because every vertex is adjacent to each other in tournaments,
therefore tournaments is a special case of
quasi-transitive oriented graph.

Although quasi-transitive oriented graph can be quite strange
(for example, graphs with no edges are quasi-transitive oriented graphs),
there are many connections between quasi-transitive oriented graphs
and tournaments~\cite{bangjensen_quasitransitive_1995}.

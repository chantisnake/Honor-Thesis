\section{Tie Components}

We keep exploring the possibility of tie paths and the idea of
 ``arrow direction transmitted by ties''.
We will be able to define the following structure with
the intuition of ``path'' and ``connected component''
in undirected graph.

\begin{definition}
  In a digraph \(G\),
  a \keyword{tie component} of vertex \(a\): \(C(a)\)
  contains the vertex \(a\) and all vertices \(v\)
  such that there exists a tie path between \(a\) and \(v\).
  \(a\) is called the \keyword{representation} of \(C(a)\)
\end{definition}

\begin{figure}
  \centering
  \begin{subfigure}[b]{0.45\linewidth}
    \centering
    \tikz\graph[simple necklace layout, math nodes, node sep=1cm] {
      a -> {c, d, e};
      b -> {c, d, e};
      {c, d, e} -> f;
      {a, b} -> f;
      c -> d --[dashed] e --[dashed] c;
      a --[dashed] b;
    };
    \caption{distinct tie components are
    \(\set{a, b}, \set{c, d, e}, \set{f}\)}
    \label{fig: tie component example} %chktex 24
  \end{subfigure}
  \begin{subfigure}[b]{0.45\linewidth}
    \centering
    \tikz\graph[simple necklace layout, math nodes, node sep=1cm] {
      a;
      c;
      d --e --c;
      a --b;
      f;
    };
    \caption{connected components are
    \(\set{a, b}, \set{c, d, e}, \set{f}\)}
    \label{fig: connected component example} %chktex 24
  \end{subfigure}
  \caption{the tie component in (a) is the connected component in (b)}
  \label{fig: tie components and connected components}  % chktex 24
\end{figure}

In \cref{fig: tie components and connected components},
we first change every tie in
\cref{fig: tie component example} to an undirected edge
and then remove all the directed edges
to obtain \cref{fig: connected component example},
and tie component in \cref{fig: tie component example}
exactly corresponded to connected components in
\cref{fig: connected component example}.
Therefore, just like tie path is similar to path
in undirected graph,
tie component is very similar to connected component
in undirected graphs,
and we can bring all the properties and intuitions
of connected component to tie component.

We then prove some of these useful properties
of tie components in digraphs.
Notice, theses theorems work in all digraphs,
not just in quasi-transitive oriented graphs.

\begin{lemma}\label{the: component equal if tie path}
  For any two vertices \(a\) and \(b\) in a digraph \(G\)
  if there exists a tie path between \(a\) and \(b\),
  then \(C(a) = C(b)\).
\end{lemma}

\begin{proof}
  Take any vertex \(v\) in \(C(a)\),
  then there exists a tie path between \(a\) and \(v\).
  Because there is a tie path between \(a\) and \(b\),
  thus there exists a tie path between \(b\) and \(v\)
  by \cref{the: tie path connection lemma}.
  Then \(v \in C(b)\). Therefore \(C(a) \subseteq C(b)\)

  By symmetry, \(C(b) \subseteq C(a)\), then \(C(a) = C(b)\)
\end{proof}

\begin{lemma}\label{the: component disjoint if no tie path}
  In a digraph, there does not exists a tie path between
  two vertex \(a, b\), then \(C(a)\) and \(C(b)\) disjoint.
\end{lemma}
\begin{proof}
  Assume \(C(a)\) and \(C(b)\) are not disjoint.
  Then there exists vertex \(v \in C(a)\) and \(v \in C(b)\).
  Therefore, there exists a tie path between \(v\) and \(a\)
  and there exists a tie path between \(v\) and \(b\).
  Then there exists a tie path between \(a\) and \(b\)
  by \cref{the: tie path connection lemma}.
  Contradiction.
\end{proof}

\begin{lemma}\label{the: adjacent if not in component}
  In a digraph, if vertex \(v\) not in tie component \(C(a)\),
  then \(v\) is adjacent to \(C(a)\).
\end{lemma}
\begin{proof}
  recall \(v\) is adjacent to \(C(a)\)
  means \(v\) is adjacent to every vertex in \(C(a)\).

  Assume there exists a vertex \(b \in C(a)\)
  that is not adjacent to \(v\),
  that is, \(v\) ties \(b\).
  Then there exists a tie path \([v, b]\) between \(v\) and \(b\).
  Because \(b \in C(a)\),
  then there exists a tie path between \(a\) and \(b\).
  Therefore, by \cref{the: tie path connection lemma},
  there exists a tie path between \(v\) and \(a\),
  and \(v \in C(a)\).
  Contradiction.
\end{proof}


\begin{theorem}\label{the: tie component partition}
  Distinct tie components of a digraph \(G\) forms a partition
  of the \(V(G)\).
\end{theorem}

\begin{proof}
  Prove that distinct tie components are disjoint:
  by \cref{the: component equal if tie path} and
  \cref{the: component disjoint if no tie path},
  any two tie component \(C(a), C(b)\) either equal or disjoint.
  Because distinct tie components cannot equal each other,
  therefore all distinct tie components are disjoint.

  Prove that every vertex \(v\) in a tie component:
  \(v \in C(v)\) by definition of tie component.
\end{proof}

\begin{theorem}\label{the: tie components partition unique}
  Given a quasi-transitive oriented graph \(Q\),
  the partition formed by distinct tie components is unique.
\end{theorem}

\begin{proof}
  Suppose there exists two different partition \(P\) and \(P'\),
  then there exists a tie component \(C(a) \in P\) that is
  not in \(P'\).

  take a because \(P'\) partitions the graph,
  then there exists \(C(a')\) such that \(a \in C(a')\).
  Therefore, by \cref{the: component equal if tie path},
  and \cref{the: component disjoint if no tie path},
  then \(C(a)\) and \(C(a')\) either disjoint or equal.
  Because \(a\) in both \(C(a)\) and \(C(a')\),
  therefore \(C(a) = C(a')\).
  Because the tie component \(C(a) \in P\) that is
  not in \(P'\), contradiction.
\end{proof}

Follow our intuition that tie ``transmits'' arrow directions
in quasi-transitive oriented graph,
we can see that for a tie component \(C\)
and a vertex \(v \notin C\), \(C \to v\) or \(v \to C\).
See \cref{fig: transmitting arrow direction in tie component},
for a visualization.

\begin{figure}
  \centering
  \begin{subfigure}[b]{0.45\linewidth}
    \centering
    \tikz\graph[simple necklace layout, math nodes, node sep=1cm] {
      b <- a;
      / [label = left:\(C\), draw] // {a};
    };
    \caption{\(a\) beats \(b\)}
  \end{subfigure}
  \begin{subfigure}[b]{0.45\linewidth}
    \centering
    \tikz\graph[simple necklace layout, math nodes, node sep=1cm] {
      a --[dashed] c;
      {a, c} -> b;
      / [label = left:\(C\), draw] // {a, c};
    };
    \caption{direction transmitted from \(a\) to \(c\)}
  \end{subfigure}
  \begin{subfigure}[b]{0.45\linewidth}
    \centering
    \tikz\graph[simple necklace layout, math nodes, node sep=1cm] {
      a --[dashed] c --[dashed] d;
      ""; "";  % chktex 18
      a -> b;
      c -> b;
      d -> b;
      / [label = left:\(C\), draw] // {a, c, d};
    };
    \caption{transmitted from \(a\) to \(c\) to \(d\)}
  \end{subfigure}
  \begin{subfigure}[b]{0.45\linewidth}
    \centering
    \tikz\graph[simple necklace layout, math nodes, node sep=1cm] {
      a --[dashed] c --[dashed] d;
      c --[dashed] e;
      "";  % chktex 18
      a -> b;
      c -> b;
      d -> b;
      e -> b;
      / [label = left:\(C\), draw] // {a, c, d, e};
    };
    \caption{transmitted from \(a\) to \(c\) to \(e\)}
  \end{subfigure}
  \caption{the arrow direction to \(b\) was transmitted in tie component \(C\)}
  \label{fig: transmitting arrow direction in tie component}  % chktex 24
\end{figure}

We then formalize and generalize this idea into two theorems:

\begin{lemma}\label{the: component and a single point}
  In a quasi-transitive oriented graph,
  for any tie component \(C(a)\) and
  any vertex \(v\) such that \(v \notin C(a)\),
  then either \(C(a) \to v\) or \(v \to C(a)\).
\end{lemma}

\begin{proof}
  Because \(v\) is not in \(C(a)\),
  then \(v\) is adjacent to every vertex in \(C(a)\)
  by \cref{the: adjacent if not in component}.
  Take any vertex \(b \in C\),
  there exists a tie path between \(a\) and \(b\),
  and by \cref{the: tie path division lemma},
  for every vertex \(c\) on this tie path,
  there exists a tie path between \(c\) and \(a\),
  therefore \(c \in C(a)\),
  thus \(v\) is adjacent to every vertex \(c\) on the tie path.
  Therefore \(v \to \set{a, b}\) or \(\set{a, b} \to v\),
  by \cref{the: tie transimission}.

  Case 1, \(a \to v\): for any vertex \(b \in C(a)\),
  \(b \to v\), by above argument.
  Therefore \(C(a) \to v\).

  Case 2, \(v \to a\): then for any vertex \(b \in C(a)\),
  \(v \to b\), by above argument.
  Therefore \(v \to C(a)\).

  Case 3, \(v\) ties \(a\):
  \(v\) cannot tie \(a\) because \(v \notin C(a)\).
\end{proof}

\begin{lemma}\label{the: vertex force component beating}
  In a quasi-transitive oriented graph,
  for any tie component \(C(a)\),
  If there exists one vertex \(v \in C(a)\)
  that beats (be beaten) a vertex \(v' \notin C(a)\).
  Then \(C(a) \to v'\) (\(v' \to C(a)\)).
  See \cref{fig: vertex force component beating}
\end{lemma}

\begin{figure}
  \centering
  \begin{subfigure}[b]{0.45\linewidth}
    \centering
    \tikz\graph[simple necklace layout, math nodes, node sep=1cm] {
      a --[dashed] b --[dashed] "\ldots" --[dashed] c --[dashed] d;  % chktex 18
      c --[dashed] e;
      ""; % chktex 18
      b -> f;
      ""; ""; % chktex 18
      / [label = 45:\(C(a)\), draw] // {a, b, "\ldots", c, d, e};  % chktex 18
    };
    \caption{a single vertex in \(C(a)\) beats \(f\)}
  \end{subfigure}
  \begin{subfigure}[b]{0.45\linewidth}
    \centering
    \tikz\graph[simple necklace layout, math nodes, node sep=1cm] {
      a --[dashed] b --[dashed] "\ldots" --[dashed] c --[dashed] d;  % chktex 18
      c --[dashed] e;
      ""; % chktex 18
      {a, b, "\ldots", c, d, e} -> f;  % chktex 18
      ""; ""; % chktex 18
      / [label = 45:\(C(a)\), draw] // {a, b, "\ldots", c, d, e};  % chktex 18
    };
    \caption{component \(C(a)\) has to beat \(f\)}
  \end{subfigure}
  \caption{a vertex \(b\) beats \(f\)
    will force the component \(C(a)\) to beats \(f\)}
  \label{fig: vertex force component beating}  % chktex 24
\end{figure}

\begin{proof}
  Because \(v' \notin C(a)\),
  then either \(v' \to C(a)\) or \(C(a) \to v'\)
  by \cref{the: component and a single point}.

  Case 1, \(v \to v'\): because \(v \in C(a)\),
  then \(v'\) cannot beat all of \(C(a)\).
  Therefore \(C(a) \to v'\).

  Case 2, \(v' \to v\): because \(v \in C(a)\),
  then \(v'\) cannot be beaten by all of \(C(a)\).
  Therefore \(v' \to C(a)\).
\end{proof}

To further generalize \cref{the: component and a single point},
we prove the following theorem:

\begin{theorem}\label{the: tie component beats tie component}
  For any two distinct tie components \(C(a)\) and \(C(b)\)
  in a quasi-transitive oriented graph,
  \(C(a) \to C(b)\) or \(C(b) \to C(a)\).
\end{theorem}

\begin{proof}
  Take any vertex \(a'\) in \(C(a)\).
  By \cref{the: tie component partition}, \(a'\) is not in \(C(b)\).
  Then because of \cref{the: component and a single point},
  \(a' \to C(b)\) or \(C(b) \to a'\).

  Case 1, \(a' \to C(b)\): take any element \(b' \in C(b)\).
  By \cref{the: vertex force component beating}, \(C(a) \to b'\).
  Therefore \(C(a)\) beats every vertex in \(C(b)\),
  then \(C(a) \to C(b)\).

  Case 2, \(C(b) \to a'\):
  by almost the same reasoning, \(C(b) \to C(a)\).
\end{proof}

After couple pages of theorems,
we can finally take a pause
and understand what we are saying here.
We combine \cref{the: tie component partition} and
\cref{the: tie component beats tie component},
then try to understand it visually.

\begin{figure}
  \centering
  \begin{subfigure}[b]{0.45\linewidth}
    \centering
    \tikz\graph[simple necklace layout, math nodes, node sep=1cm] {
      a; b;
      a -> {c, d, e};
      b -> {c, d, e};
      {c, d, e} -> f;
      {a, b} -> f;
      j -> {c, d, e};
      {a, b} -> j;
      f -> j;
      c -> d --[dashed] e --[dashed] c;
      a --[dashed] b;
    };
    \caption{a quasi-transitive oriented graph}
  \end{subfigure}
  \begin{subfigure}[b]{0.45\linewidth}
    \centering
    \tikz\graph[simple necklace layout, math nodes, node sep=1cm] {
      a; b; c; d; e; f; j;
      d --[dashed] e --[dashed] c;
      a --[dashed] b;
      / [label = left:\(A\), draw, circle] // {a, b};
      / [label = left:\(B\), draw] // {d, e, c};
      / [label = right:\(C\), draw, circle] // {f};
      / [label = right:\(D\), draw, circle] // {j};
    };
    \caption{find its tie components}
  \end{subfigure}
  \begin{subfigure}[b]{0.45\linewidth}
    \centering
    \tikz\graph[simple necklace layout, math nodes, node sep=1cm] {
      A [draw, circle, minimum size=1.5cm] -> B [draw, circle, minimum size=1.75cm]
      -> C [draw, circle, minimum size=1cm] -> D [draw, circle, minimum size=1cm];
      A -> D -> B
    };
    \caption{components beats each other}
  \end{subfigure}
  \begin{subfigure}[b]{0.45\linewidth}
    \centering
    \tikz\graph[simple necklace layout, math nodes, node sep=1cm] {
      A' -> B' -> C' -> D';
      A' -> D' -> B';
    };
    \caption{components are just like vertex}
  \end{subfigure}
  \caption{condense the tie components and get a tournament}
  \label{fig: tie components condensation}  % chktex 24
\end{figure}

See \cref{fig: tie components condensation}.
For every quasi-transitive digraph,
we can split it into tie components,
and every tie component either beats another tie component
or be beaten by another tie component.
Then, tie components are kind of like vertices.
What even more exciting is that
there cannot be ties between tie components.
The graph formed by tie components is a tournament.

\section{Kings}\label{sec: quasi-transitive king}

The definition of king states that it can beat every vertex
by one or two steps,
and the definition of quasi-transitive oriented graph states that
if there exists a path of length two (implies beats by 1 or 2 steps)
from one vertex to another vertex, then they are adjacent.

Both definitions are related to ``beats by 1 or 2 steps'',
therefore kings in quasi-transitive oriented graphs
have nice properties.

\begin{lemma}\label{the: king adjacent}
  In a quasi-transitive oriented graph,
  a king is adjacent to every vertex.
\end{lemma}

\begin{proof}
  Given a king \(k\), and another vertex \(v\),
  then there are two possibilities:
  \begin{itemize}
    \item
      Case 1, \(k\) beats \(v\) by one step:
      therefore \(k \to v\), \(k\) and \(v\) are adjacent.
    \item
      Case 2, \(k\) beats \(v\) by two steps:
      there exists vertex \(a\), such that \(k \to a \to v\).
      Then by definition of quasi-transitive oriented graph,
      \(k\) needs to be adjacent to \(v\)
  \end{itemize}
\end{proof}

\begin{lemma}\label{the: king partitions in quasi-transitive}
  In a quasi-transitive oriented graph,
  for any king \(k\), \(\set{D_k, S_k, \set{k}}\)
  partitions the graph.
\end{lemma}

\begin{proof}
  Proof disjoint: pretty obvious.
  \(D_k, S_k\) disjoint because we are working in an oriented graph.
  \(D_k, S_k\) disjoint with \(\set{k}\)
  because a vertex cannot beat itself.

  Proof the union is the whole vertex set:
  by \cref{the: king adjacent}
\end{proof}

\begin{lemma}\label{the: D S of king adjacent in quasi-transitive}
  In a quasi-transitive oriented graph,
  for any king \(k\), \(D_k, S_k\) are adjacent.
\end{lemma}

\begin{proof}
  By definition of quasi-transitive oriented graph,
  because every vertex in \(D_k\) beats \(k\)
  and then beats every vertex in \(S_k\),
  every vertex in \(D_k\) is adjacent to every vertex in \(S_k\).
\end{proof}

\begin{theorem}
  In a quasi-transitive oriented graph,
  if we have a king \(k\), then
  \begin{itemize}
    \item \({D_k, S_k, \set{k}}\) partitions the vertex set.
    \item \(D_k\) and \(S_k\) is adjacent.
  \end{itemize}
\end{theorem}

\begin{figure}
  \centering
  \tikz\graph[simple necklace layout, math nodes, node sep=1.75cm] {
      D_k [draw, circle, minimum size=2cm] ->
      k ->
      S_k [draw, circle, minimum size=2cm];
      ""; % chktex 18
      D_k --[bend left] S_k;
  };
  \caption{the rich structure of a king in a quasi-transitive oriented graph.}
  \label{fig: king in quasi-transitive}  % chktex 24
\end{figure}

See \cref{fig: king in quasi-transitive},
this figure shows the rich structure of a king
in quasi-transitive oriented graph.
We were able to see a similar partition structure mentioned in
\cref{chap: semi-complete digraph},
and \cref{the: graph partition lemma}
(the partition structure in \cref{fig: king in quasi-transitive}
is the same as \(\set{\set{v}, D_v, S_v}\)
for any vertex \(v\) in a tournament~\cite{maurer_king_1980}).

\(D_k\) cannot tie with anything outside of \(D_k\),
and similarly for \(S_k\).
If there is a king in a quasi-transitive oriented graph,
then this graph becomes ``very connected'',
the only places ties can appear are inside the
induced subgraphs of \(D_k\) and \(S_k\).

Since we have deduced nice structures about ties
in quasi-transitive oriented graphs,
it is only logical to combine the property of king
with the property of tie (tie component).

\begin{theorem}\label{the: king in quasi-transitive}
  A vertex \(k\) is a king in a quasi-transitive oriented
  graph if and only if
  \begin{itemize}
    \item \(k\) is in a trivial tie component.
    \item the result of \(k\) after tie component condensation
    is a king in the underlying tournament.
  \end{itemize}
\end{theorem}

\begin{proof}
  By \cref{the: condensation preserves king},
  \(k\) is a king if and only if
  \begin{itemize}
    \item \(k\) is a king in the induced subgraph of its tie component.
    \item the result of \(k\) after tie component condensation
    is a king in the underlying tournament.
  \end{itemize}

  So we need to show \(k\) is a king in
  induced subgraph of its tie component
  if and only if \(k\) is in a trivial tie component.

  Suppose \(k\) is a king in the induced graph of
  its tie component,
  and the fact that the induced subgraph is a quasi-transitive
  oriented graph.
  Because of \cref{the: king adjacent},
  \(k\) is a king in its tie component implies
  \(k\) is adjacent to all vertices in the tie component.
  There can be no tie path in this tie component.
  Therefore \(k\) has to be in a trivial tie component.

  Suppose \(k\) is is in a trivial tie component,
  then \(k\) is a king in the induced graph of
  its tie component.
 Recall that the definition of king says that
  a digraph with one vertex,
  that vertex is a king in the digraph.
\end{proof}

\begin{figure}
\centering
  \begin{subfigure}[b]{0.45\linewidth}
  \centering
    \tikz\graph[simple necklace layout, math nodes, node sep=1cm] {
      a_1 -> {b_1, b_2};
      {b_1, b_2} -> c_1;
      {b_1, b_2} -> c_2;
      {c_1, c_2} -> a_1;
      {b_1, b_2} -> d_1;
      {c_1, c_2} -> d_1;
      a_1 -> d_1;
      / [label=\(a\), draw] // {a_1};
      / [label=left:\(b\), draw] // {b_1, b_2};
      / [label=right:\(c\), draw] // {c_1, c_2};
      / [label=right:\(d\), draw] // {d_1};
    };
    \caption{quasi-transitive oriented graph \(G\).}
  \end{subfigure}
  \begin{subfigure}[b]{0.45\linewidth}
  \centering
    \tikz\graph[simple necklace layout, math nodes, node sep=1cm] {
      a -> b -> c -> a;
      {a, b, c} -> d;
    };
    \caption{the underlying tournament of \(G\).}
  \end{subfigure}
  \caption{a quasi-transitive digraph and its underlying tournament.}
  \label{fig: quasi-transitive king example}  % chktex 24
\end{figure}

In \cref{fig: quasi-transitive king example},
we show a quasi-transitive oriented graph \(G\)
and its underlying tournament.
In the underlying tournament the kings are
\(a, b, c\), but only \(a_1\) can be a king of \(G\),
since vertices \(b\) and \(c\) do not correspond to
trivial tie components.
(tie component \(\set{b_1, b_2}\) is condensed into \(b\),
and tie component \(\set{c_1, c_2}\) is condensed into \(c\)).
Notice, although vertex \(d_1\) is in a trivial tie component,
\(d_1\) is not a king in \(G\),
because in the underlying tournament, \(d\)
(the image of \(d_1\) under tie condensation) is not a king.

This theorem not only gives us a way to identify kings
in a quasi-transitive oriented graph,
but it helps us to construct a quasi-transitive oriented graph
with a certain number of kings.

\begin{figure}
\centering
  \begin{subfigure}[b]{0.45\linewidth}
  \centering
    \tikz\graph[simple necklace layout, math nodes, node sep=1cm] {
      a -> b -> c -> a;
      {a, b, c} -> d
    };
    \caption{start with a tournament with 3 kings.}
  \end{subfigure}
  \begin{subfigure}[b]{0.45\linewidth}
  \centering
    \tikz\graph[simple necklace layout, math nodes, node sep=1cm] {
      a -> {b_1, b_2} -> c -> a;
      b_1 --[dashed] b_2;
      {a, b_1, b_2, c} -> d;
      / [label=left:\(b\), draw] // {b_1, b_2}
    };
    \caption{then change king \(b\) to \(\set{b_1, b_2}\).}  %chktex 44
  \end{subfigure}
  \caption{construct a quasi-transitive oriented graph with 2 kings.}
  \label{fig: quasi-transitive 2 kings}  % chktex 24
\end{figure}

For example, if we want to construct a quasi-transitive
oriented graph with \(k\) kings,
we first start with a quasi-transitive graph with \(k'\) kings,
where \(k' > k\),
and then we change \(k' - k\) kings into
``non-trivial tie components''.
This method enables us to construct
a quasi-transitive oriented graphs
with arbitrary number of kings.
See \cref{fig: quasi-transitive 2 kings},
we start with a tournament with 3 kings \(\set{a, b, c}\),
then we change the king \(b\) into a non-trivial tie component
\(\set{b_1, b_2}\).
Then the only kings left are \(a\) and \(c\),
and we get a quasi-transitive oriented graph with 2 kings.

Another significant usage of \cref{the: king in quasi-transitive}
is to use the properties of kings in tournaments to prove
properties of kings in quasi-transitive oriented graphs.

\begin{definition}
  \(k\) is a \keyword{great king} in a digraph \(G\),
  if and only if \(k\) is the only king in digraph \(G\),
  and \(k\) is not an emperor.
\end{definition}

By \cref{the: one king iff emperor},
\cref{the: if only king then emperor},
a great king does not exists in semi-complete digraphs and
tournaments.

\begin{corollary}
  If there exists a great king in a
  quasi-transitive oriented graph \(G\),
  then there exists at least two ties in \(G\).
\end{corollary}

\begin{proof}
  Assume \(k\) is the great king in \(G\).
  Then there exists a vertex  \(v\) such that \(v \to k\).
  Denote the underlying tournament of \(G\) as \(T\),
  and the tie component condensation of \(G\) as \(f\).

  By \cref{the: vertex force image beating}, \(f(v) \to f(k)\).
  Therefore, by \cref{the: king in quasi-transitive}
  \(f(k)\) is a king, but not an emperor in \(T\).
  Because a tournament never has exactly 2 kings
  and exactly 1 king must be an emperor,
  there exists at least 3 kings in \(T\).
  Because \(k\) is the only king in \(G\),
  there has to exist at least 2 non-trivial tie components.
  Therefore, \(G\) have at least 2 ties.
\end{proof}

\cref{the: king in quasi-transitive}
provides us a way to detect, construct, and understand kings
in quasi-transitive oriented graph using
the properties and constructions of kings in tournaments.

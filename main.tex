%% Please compile this file using LuaLatex
\documentclass[12pt, oneside]{book}


%% Language and font encodings
\usepackage[english]{babel}
\usepackage{csquotes}
\usepackage[T1]{fontenc}

%% Sets page size, margins and spacing
\usepackage[top=3cm,bottom=2cm,left=3cm,right=3cm,marginparwidth=1.75cm]{geometry}
\usepackage{setspace,caption}
% line spacing
\linespread{1.5}
\selectfont
% reduce item separation
\usepackage{enumitem}
\setlist[1]{itemsep=-5px}
% set the bibliography
\usepackage[style=numeric-comp]{biblatex}
\addbibresource{bib_file/bib_file.bib}

%% Useful packages
\usepackage{xparse} % define commands
\usepackage{amsmath}
\usepackage{amsthm}
\usepackage{mathtools}
\usepackage{amssymb}
\usepackage{graphicx}
\usepackage[colorinlistoftodos]{todonotes}
\usepackage[colorlinks=true, allcolors=black]{hyperref}


%% Additional Packages
% for graphics
\usepackage{tikz}
\usetikzlibrary[graphs, graphdrawing, graphs.standard]
\usegdlibrary{circular, layered, trees}
% Use index
\usepackage{makeidx}
\makeindex
% Use sub figure
\usepackage{caption}
\usepackage[justification=centering, position=bottom]{subcaption}
% better reference
\usepackage[capitalise]{cleveref}
% Add bibliography to table of contents
\usepackage[nottoc,numbib]{tocbibind}
% to reference title
\usepackage{titling}
% date time
\usepackage{datetime}
\newdateformat{monthyeardate}{\monthname[\THEMONTH], \THEYEAR}

%% Define theorem environments
\newtheorem{definition}{Definition}[chapter]
\newtheorem{theorem}{Theorem}[chapter]
\newtheorem{lemma}[theorem]{Lemma}
\newtheorem{corollary}[theorem]{Corollary}
\newtheorem{example}{Example}[chapter]

%% Define useful functions
% \keyword function is used for crucial words,
% you can both highlight and index that word.
% Typically used in definitions.
\NewDocumentCommand\keyword{mo}{ %'mo' means #1 is 'm'andatory and #2 is 'o'ptional
  \IfNoValueTF{#2}
   {\textbf{\boldmath{#1}}\index{\MakeLowercase{#1}}}
   {\textbf{\boldmath{#1}}\index{#2}}
}

%% Define the set notation
\DeclarePairedDelimiter\set\{\}  %chktex 21

\title{Kings in Generalized Tournaments}
\author{Cheng Zhang}

\begin{document}

  %%% Preface
  % The title page
  \begin{titlepage}
    \thispagestyle{empty}
    \doublespacing\centering

    \vfill
    \LARGE \textbf{\thetitle{}} \\
    \large BY \\
    \textbf{\theauthor{}} \\

    \vfill
    A Study \\
    Presented to the Faculty \\
    of \\
    Wheaton College \\
    in Partial Fulfillment of the Requirements \\
    for \\
    Graduation with Departmental Honors \\
    in Mathematics \\
    Norton, Massachusetts \\
    \monthyeardate\today

    \vfill
  \end{titlepage}

  % The menu
  \tableofcontents

  %%% The chapters
  \chapter{Abstract}

This thesis explores how to find and construct kings
in three generalizations of tournament:
semi-compelete digraphs, oriented graphs and
quasi-transitive oriented graphs.

In \cref{chap: semi-complete digraph} and \cref{chap: general oriented graph},
We present a way to interpret semi-compelete digraphs
and oriented graphs as tournaments with ``ties''
(we call the ``ties'' in semi-compelete digraphs ``double ties'',
and the ``ties'' in oriented graphs ``ties'').
In \cref{chap: semi-complete digraph},
we prove there exists an \((n, k)\) semi-complete digraphs
if and only if \(n \geq k \geq 1\),
and all the \((n, k)\) semi-compelete digraphs that exists
can be constructed with at most 1 double tie.
In \cref{chap: general oriented graph},
we prove there exists an \((n, k)\) oriented
for all \(n \geq k \geq 0\) except
\((1,0)\), \((2,2)\), \((3,2)\), and \((4,4)\) oriented graphs
and all the \((n, k)\) oriented graphs that exists
can be constructed with at most 1 tie.

The main focus of this thesis is quasi-transitive oriented graph,
which is discussed in \cref{chap: quasi-transitive}.
We showed a interesting fact that
all the quasi-transitive oriented graphs
can be condensed into tournaments by
``tie component condensations''.
Then, we showed that the tie component condensation
on a quasi-transitive oriented graph
is a most efficient condensation to tournament
in all the condensations to tournaments
defined on all the oriented graph with the same tie structure.
Finally we prove that the kings in
quasi-transitive oriented graph \(Q\) is related to
the kings in the ``underlying tournament of \(Q\)''
(result of \(Q\) after tie component condensation).
This result gives us a way to understand the
properties of kings in quasi-transitive oriented graphs
using the properties of king in tournaments.

  \chapter{Acknowledgement}

  \chapter{Introduction}

\section{Summary of Previous Works}

In~\cite{maurer_king_1980}, the author introduced
the idea of kings (or 2-kings) in tournaments,
and proved that kings in tournaments have many nice properties.
In the end of~\cite{maurer_king_1980},
the author proposed several ways to generalize kings,
one of them is \(s\)-kings where \(s\) is a positive integer.
When we say ``king'' in a graph theory context,
we usually mean 2-king.
In~\cite{reid_every_1982}, the author
further investigated the properties of kings in tournaments,
and proved many interesting theorem about them.

With the more and more understanding for the properties
of kings in tournaments,
people begin to curious about generalized tournaments.

In~\cite{bangjensen_quasitransitive_1995} the author
proposed a generalization of tournaments called
quasi-transitive digraph (a generalization of
quasi-transitive oriented graphs studied in \cref{chap: quasi-transitive}).
Then in~\cite{bang-jensen_kings_1998}
the same author investigated the properties
of 3-kings in quasi-transitive digraph.
And~\cite{galeana-sanchez_existence_2013}
generalized the result in~\cite{bang-jensen_kings_1998}
to \((k+1)\)-kings and \(k\)-quasi-transitive digraphs.

In~\cite{bang-jensen_generalizations_1998},
the author surveys various types of generalized tournaments,
including semi-complete digraphs, oriented graphs and
quasi-transitive digraphs.
However, this article did not focus on the properties
of kings in these generalizations of tournaments.


\section{Structure of This Thesis}

\begin{figure}
  \centering
  \begin{tikzpicture}

    \begin{scope}[shift={(3cm,-5cm)}, fill opacity=0.5,
      mytext/.style={text opacity=1, font=\large\bfseries}]

      \draw[draw = black,name path=circle 1] (-1.5,0) circle (3);
      \draw[draw = black,name path=circle 2] (1.5,0) circle (3);

      \pgftransparencygroup
      \clip (-1.5,0) circle (3);
      \fill[gray] (1.5,0) circle (3);
      \filldraw[draw, fill=gray, name intersections={of=circle 1 and circle 2}] (intersection-1) .. controls +(-4,1) and +(-4,-1) ..(intersection-2);
      \endpgftransparencygroup

      \node[mytext] at (-3.8,0) (B) {O};
      \node[mytext] at (3.8,0) (C) {S};
      \node[mytext] at (0,0) (D) {T};
      \node[mytext] at (-2,1) (E) {Q};
    \end{scope}
  \end{tikzpicture}
  \caption{the relationship between tournaments,
    semi-complete digraphs, oriented graphs,
    and quasi-transitive oriented graphs.}
  \label{fig: generalized tournaments relationship}  % chktex 24
\end{figure}

The core content of this paper is in
\cref{chap: background}, \cref{chap: semi-complete digraph},
\cref{chap: general oriented graph},
and \cref{chap: quasi-transitive}.
The latter three chapters focused on kings in
three generalization of tournaments:
semi-complete digraph (see \cref{def: semi-compelete digraph}),
oriented graph (see \cref{def:oriented graph}),
and quasi-transitive oriented graph
(see \cref{def: quasi-transitive oriented graph}),
respectively.

It is helpful to know the relation of these
three generalizations of tournaments.
In \cref{fig: generalized tournaments relationship},
we show the relationship between these graphs.
The ``O'' represents the set of all oriented graphs;
the ``Q'' represents the set of all quasi-transitive oriented graphs;
the ``T'' represents the set of all tournaments;
the ``S'' represents the set of all semi-compelete digraphs.
tournaments are a subset of all 3 generalized tournaments.
Quasi-transitive oriented graphs are a subset of oriented graphs,
and the intersection between oriented graph and
semi-compelete digraph is exactly all the tournaments
(hence the name ``semi-compelete oriented graph'',
see \cref{def: tournaments}).


In \cref{chap: background}, we discussed some of the terminology
and previously-proven theorems that are used in other chapters.
\cref{chap: semi-complete digraph}, \cref{chap: general oriented graph},
and \cref{chap: quasi-transitive} are all independent from each other.
We recommend the reader starts with the chapter that interests
them the most.

\cref{chap: semi-complete digraph} and \cref{chap: general oriented graph}
are two relatively easy chapter, comparing to \cref{chap: quasi-transitive}.
These two chapters mainly focused on constructing
\((n, k)\) semi-complete digraphs and
\((n, k)\) oriented graph (see \cref{def: (n k) graphs}).

\cref{chap: quasi-transitive} is the most interesting chapter.
Unfortunately, it is harder to read than
\cref{chap: semi-complete digraph} and \cref{chap: general oriented graph}.
This chapter is proof heavy and
moves at a faster pace than the previous two chapters.
In this chapter we proved the four most important result
in this thesis; namely
\cref{the: tie condensation results in tournament},
\cref{the: tie components partition unique},
\cref{the: tie condensation effcient},
and \cref{the: king in quasi-transitive}.

  \chapter{Background}\label{chap: background}

\section{Directed Graph}

  \begin{definition}
    a \keyword{directed graph} or \keyword{digraph}
    consists of a vertex set \(V(G)\),
    and an edge set (ordered pair of vertices) \(E(G)\).
  \end{definition}

  \begin{figure}
    \centering
    \tikz\graph[simple necklace layout, math nodes, node sep=1.75cm] {
        e -> a;
        b -> c;
        a -> b;
        e <-> [bend left] c;
        d;
        a -> [loop above] a;
    };
    \caption{example of a directed graph}
    \label{fig:digraph example} %chktex 24
  \end{figure}

  For example, in \cref{fig:digraph example},
  if we call this graph \(G\), then
  the vertex set or \(V(G)\) is \(\set{a, b, c, d, e}\),
  and the edge set or \(E(G)\) is
  \(\set{(a, a), (a, b), (b, c), (c, e), (e, a), (e,c)}\).
  We sometimes will refers to ``edges'' as ``arrows''.

\section{Beating Relations}

  To simplify the notation, we think of an edge as a
  \emph{beating relation}:

  \begin{definition}
    In a directed graph \(G\), if \((a, b) \in E(G)\)
    then we say \(a\) \keyword{beats} (\keyword{dominates})
    \(b\) or \(a \to b\).
  \end{definition}

  \begin{definition}
    A \keyword{path} or a \keyword{walk}
    from vertex \(a_0\) to vertex \(a_n\)
    is a sequence of vertices \([a_0, a_1, \ldots, a_{n-1}, a_{n}]\)
    such that \(\forall 0 \leq k < n, ~ a_k \to a_{k+1}\).
    We will sometimes write this path as
    \(a_0 \to a_1 \to \cdots \to a_{n - 1} \to a_n\).
  \end{definition}

  \begin{definition}
    For a given path \(P\), if the sequence has \(n + 1\) vertices,
    then we say the \keyword{length of path} \(P\) is \(n\).
  \end{definition}

  The length of a path measures the number of edges on this path.
  For example, in \cref{fig:digraph example},
  there is a path \(P = e \to a \to b \to c\).
  And there are edges \((e, a), (a, b), (b, c)\) on this path,
  therefore the length of this path is 3.
  There is another path \(P' = e \to c\) goes from \(e\) to \(c\).
  This path \(P'\) only have length \(1\).
  Therefore, path \(P'\) is shorter than path \(P\),
  and if we go through every possible path from \(e\) to \(c\),
  we can find out that \(P'\) is the \emph{shortest path}
  from \(e\) to \(c\).

  \begin{definition}
    In a directed graph \(G\), \(a\) beats (dominates) \(b\) by \(n\) steps,
    if the shortest path from \(a\) to \(b\) has length \(n\).
  \end{definition}

  In \cref{fig:digraph example},
  \(e \to a \to b\),
  and \(e\) does not beat \(b\) by 1 step,
  therefore \(e\) beats \(b\) by 2 steps.

  However, although \(e \to a \to b \to c\),
  \(e\) \emph{does not} beat \(c\) by 3 steps,
  because the shortest path from \(e\) to \(c\) is \(e \to c\).
  Therefore, \(e\) beats \(c\) by 1 step.

  \begin{definition}
    In a directed graph, vertex \(a\) is \keyword{adjacent} to vertex \(b\) if
    \(a \to b\) or \(b \to a\) or both.
  \end{definition}

  \begin{definition}
    In a directed graph, for two distinct vertices \(a, b\),
    if \(a\) is not adjacent to \(b\),
     then \(a\) \keyword{tie}s \(b\).
  \end{definition}

  In \cref{fig:digraph example}, \(c\) is adjacent to \(e\);
  \(a\) is adjacent to \(b\); \(b\) is adjacent to \(c\).
  Whereas, \(c\) ties \(a\),
  because there is no edge \((c, a)\) and no edge \((a, c)\)
  in this digraph.

  Since we don't always know the exact structure of the graph,
  it is sometimes useful to look at
  the beating relationships between sets of vertices.
  We then define beating, adjacency, and tie between vertex sets.

  \begin{definition}
    In digraph \(G\), we write \(A \to B\)
    where \(A\) and \(B\) are disjoint subsets of \(V(G)\),
    when every vertex in \(A\) beats every vertex in \(B\).
    We also write \(A \to b\) as a shorthand for \(A \to \set{b}\),
    and \(a \to B\) as a shorthand for \(\set{a} \to B\).
  \end{definition}

  \begin{definition}
    In digraph \(G\), \(A\) is adjacent to \(B\)
    where \(A\) and \(B\) are disjoint subsets of \(V(G)\),
    when every vertex in \(A\) is adjacent to every vertex in \(B\).
    We also write ``\(A\) is adjacent to \(b\)''
    as a shorthand for ``\(A\) is adjacent to \(\set{b}\)'',
    and ``\(a\) is adjacent to \(B\)''
    as a shorthand for ``\(\set{a}\) is adjacent to \(B\)''.
  \end{definition}

  \begin{definition}
    In digraph \(G\), \(A\) ties \(B\)
    where \(A\) and \(B\) are disjoint subsets of \(V(G)\),
    when every vertex in \(A\) ties every vertex in \(B\).
    We also write ``\(A\) ties \(b\)''
    as a shorthand for ``\(A\) ties \(\set{b}\)''
    and ``\(a\) ties \(B\)''
    as a shorthand for ``\(\set{a}\) ties \(B\)''
  \end{definition}

  \begin{figure}
    \centering
    \begin{subfigure}[b]{.45\linewidth}
      \centering
      \tikz\graph[simple necklace layout, math nodes, node sep=1.75cm] {
        {a_1, a_2} -> b;
        b -> {c_1, c_2};
        a_1 -> {a_2, c_2};
        c_1 -> c_2;
      };
      \subcaption{}
    \end{subfigure}
    \begin{subfigure}[b]{.45\linewidth}
      \centering
      \tikz\graph[simple necklace layout, math nodes, node sep=1.75cm] {
        A [draw, circle, minimum size=2cm] -> b -> C [draw, circle, minimum size=2cm];
      };
      \subcaption{}
      \label{fig: condenced beating: condenced subfig} % chktex 24
    \end{subfigure}
    \caption{we can draw graph (a) as graph (b)}
    \label{fig:condenced beating} % chktex 24
  \end{figure}

  In \cref{fig:condenced beating},
  we show how we draw the beating relationship with
  subsets \(A = \set{a_1, a_2}\) and \(C = \set{c_1, c_2}\).
  Notice, in the \cref{fig: condenced beating: condenced subfig},
  set \(A\) and set \(C\) are not adjacent.
  This \emph{does not} means that
  set \(A\) ties set \(C\);
  it only means that we have not drawn out
  the relationship between these 2 sets in
  \cref{fig: condenced beating: condenced subfig}.

  \begin{figure}
    \centering
    \begin{subfigure}[b]{.45\linewidth}
      \centering
      \tikz\graph[simple necklace layout, math nodes, node sep=1.75cm] {
        a_1 -> b;
        b -> {c_1, c_2, a_2};
        a_1 -> a_2;
        c_1 -> c_2;
      };
      \caption{}
    \end{subfigure}
    \begin{subfigure}[b]{.45\linewidth}
      \centering
      \tikz\graph[simple necklace layout, math nodes, node sep=1.75cm] {
        A [draw, circle, minimum size=2cm] --  %chktex 8
        b ->
        C [draw, circle, minimum size=2cm] -- [dashed] %chktex 8
        A;  % chktex 13
      };
      \caption{}
    \end{subfigure}
    \caption{we can draw graph (a) as graph (b)}
    \label{fig:condenced tie and adjacency} % chktex 24
  \end{figure}

  We draw adjacency between sets
  (for example, set \(A\) is adjacent to set \(\set{b}\))
  using a solid edge without arrow.
  We draw tie between sets (for example, set \(A\) ties set \(C\))
  using a dashed edge without arrow.
  (see \cref{fig:condenced tie and adjacency})


  \begin{definition}
    The \keyword{submissive set} (\keyword{dominant set})
    of a vertices \(v\) in graph \(G\) is the set of vertices in \(G\)
    that are beaten by \(v\) (beat \(v\)),
    formally it is \(\set{e \in V(G) \mid v \to e}\)
    (or \(\set{e \in V(G) \mid e \to v}|\)).
    We will denote the submissive set of \(v\) as \(S_v\),
    and the dominant set of \(v\) as \(D_v\).
    (See \cref{fig: dominate and submissive set example})
  \end{definition}

  \begin{figure}
    \centering
    \tikz\graph[layered layout, math nodes, grow=right, sibling distance=2cm,level sep=0.75cm] {
        D_v [draw, circle, minimum size=2cm] ->
        v ->
        S_v [draw, circle, minimum size=2cm]
    };
    \caption{\(D_v\) is the dominant set of \(v\);
       \(S_v\) is the submissive set of \(v\)}
    \label{fig: dominate and submissive set example}  % chktex 24
  \end{figure}

  \begin{definition}
    \keyword{Out-degree} (\keyword{in-degree}) of a vertex \(v\) in graph \(G\) is
    the size of the submissive set (dominant set) of \(v\).
  \end{definition}

  For example, in \cref{fig:digraph example},
  \(S_a = \set{a, b}\), therefore vertex \(a\) has out-degree 2;
  \(D_a = \set{e, a}\), therefore vertex \(a\) has in-degree 2.
  Vertex \(d\) has in-degree 0, and out-degree 0
  because \(d\) is not adjacent to any other vertex in the graph.
  Therefore, \(D_d\) and \(S_d\) are both empty sets.

  Notice it is very natural to relate
  out-degree and in-degree with the ``power'' or ``strength''
   of a vertex.
  However, we will discuss in later sections and chapters
  that having larger out-degree does not guarantee
  a vertex will be a ``king'' (defined in \cref{sec:kings}),
  However, this property is true in some nice families of graphs
  (see \cref{sec:kings} and
  \cref{sec: semi-complete properties}).


  \begin{definition}
    A \keyword{induced subgraph} or \keyword{vertex-induced subgraph}
    \(H\) of digraph \(G\) is a subgraph of \(G\) such that:
    for all pairs of vertices \(a, b \in V(H)\),
    if \((a, b) \in E(G)\), then \((a, b) \in E(H)\).
  \end{definition}

  \begin{figure}
    \centering
    \begin{subfigure}[b]{.3\linewidth}
      \centering
      \tikz\graph[simple necklace layout, math nodes, node sep=1.75cm] {
        a -> b;
        b -> {c, d, e};
        a -> c;
        d -> e;
      };
      \caption{start with graph \(G\)}
    \end{subfigure}
    \begin{subfigure}[b]{.3\linewidth}
      \centering
      \tikz\graph[simple necklace layout, math nodes, node sep=1.75cm] {
        a; b; c; ""; "";  %chktex 18
      };
      \caption{let \(V(H) = \set{a, b, c}\)}
    \end{subfigure}
    \begin{subfigure}[b]{.3\linewidth}
      \centering
      \tikz\graph[simple necklace layout, math nodes, node sep=1.75cm] {
        a -> {b, c};
        b -> c;
        ""; "";  %chktex 18
      };
      \caption{induced subgraph \(H\)}
    \end{subfigure}
    \caption{the process to create a induced subgraph.}
    \label{fig:induced subgraph example} % chktex 24
  \end{figure}

  In \cref{fig:induced subgraph example},
  we show the process of creating a induced subgraph.
  We redraw the vertices in \(V(H)\), that is \(a, b, c\),
  and then simply copy the edges between those vertices,
  that is, \((a, b), (b, c), (a, c)\) from \(G\) to get \(H\).

  This definition can be viewed in 2 ways:
  \begin{itemize}
    \item
      Take the vertices \(a, b, c\)
      together with the edges between them
      to form subgraph \(H\).
    \item
      Take away vertices \(e, d\)
      together with all the edges
      having \(e\) or \(d\) as an end point,
      the rest of the graph is the subgraph \(H\).
  \end{itemize}

  For all the families of graphs discussed in this paper
  (oriented graph, semi-complete digraph,
  tournament,quasi-transitive oriented graph),
  induced subgraphs preserve the property of the original graphs.

\section{Oriented Graph and Tournament}

  \begin{definition}\label{def:oriented graph}
    An \keyword{oriented graph} is a digraph, such that:
    \begin{enumerate}
      \item for all vertices \(a\), \(a\) does not beat itself.
      \item for all pairs of adjacent vertices \(a, b\),
        if \(a\) beats \(b\), then \(b\) does not beats \(a\).
    \end{enumerate}
  \end{definition}

  \begin{figure}
    \centering
    \tikz\graph[layered layout, math nodes, grow=right, sibling distance=2cm,level sep=0.75cm] {
      {a, c}
      -> [bend right] d
      -> b
      -> [bend right] c;
      a -> [bend left] b;
    };
    \caption{example of an oriented graph}
    \label{fig:oriented graph example} %chktex 24
  \end{figure}

  We show an oriented graph in \cref{fig:oriented graph example}.
  This definition of oriented graph can be viewed as
  a mathematical model of a round-robin tournament with ties allowed.
  A team cannot compete with itself, therefore property 1 holds;
  each pair of teams \(a\) and \(b\) competes exactly once,
  either \(a\) beats \(b\), \(b\) beats \(a\),
  or there is a tie between \(a\) and \(b\).
  therefore property 2 holds.

  Therefore, it is very natural to view an edge
  going from \(a\) to \(b\) as \(a\) ``beats'' \(b\),
  and two non-adjacent vertices \(a\), \(b\) as
  a tie between \(a\) and \(b\).

  \begin{definition}
    A \keyword{tournament}
    (sometimes called \keyword{semi-complete oriented graph})
    is an oriented graph without ties.
  \end{definition}

  \begin{figure}
    \centering
    \tikz\graph[simple necklace layout, math nodes, node sep=1.75cm] {
        e -> {a, b};
        b -> c -> d -> e;
        a -> d -> b;
        c -> a -> b;
        e -> c
    };
    \caption{example of a tournament}
    \label{fig:tournament example} % chktex 24
  \end{figure}

  We give a simple example of a tournament
  in \cref{fig:tournament example}.
  A tournament can be interpreted as,
  what we call in the real world, a round-robin tournament,
  where every pair of vertices compete
  and the competition results in exactly one winner and one loser.

  In~\cite{maurer_king_1980},
  the author uses tournament to model a flock of chicken,
  between every two chickens there is a ``pecking relation''
  (we call it the ``beating relation'').
  Given two chickens,
  one chicken has to peck the other chicken,
  or be pecked by the other chicken.


\section{Kings}\label{sec:kings}

  \begin{figure}
    \centering
    \tikz\graph[simple necklace layout, math nodes, node sep=1.75cm] {
        e -> {a, b, c};
        b -> c -> d -> e;
        a -> d;
        b -> a;
    };
    \caption{a more complicated example of kings}
    \label{fig:complicated king example} % chktex 24
  \end{figure}

  \begin{definition}
    a \keyword{king} in a digraph is a vertex that
    beats every other vertex by 1 or 2 steps.
  \end{definition}

  In \cref{fig:complicated king example},
  the kings in this graph are vertices \(e\) and \(d\).
  Vertex \(e\) beats \(a, b, c\) by one step,
  and beats \(d\) by 2 steps, so \(e\) is a king.
  Vertex \(d\) beats \(e\) by one step,
  and \(e\) beats \(a, b, c\),
  so \(d\) beats \(a, b, c\) by 2 steps,
  which makes \(d\) a king.

  The other vertices (\(a, b, c\)) are not kings,
  because \(a\) cannot beat \(c\) by one or two steps
  (\(a \to d \to e \to c\) is the shortest path);
  \(b\) cannot beat \(e\) by one or two steps
  (\(b \to c \to d \to e\) is the shortest path);
  \(c\) cannot beat \(b\) by one or two steps
  (\(c \to d \to e \to b\) is the shortest path).

  There are two points to note in this example:
  \begin{itemize}
    \item
      \cref{fig:complicated king example} shows that
      larger out-degree does not associate with
      more ``power'' or ``strength''.
      Vertex \(d\) only has out-degree 1, and \(d\) is a king.
      Whereas vertex \(b\) has out-degree 2,
      but \(b\) is not a king.
      Vertex \(d\) is more ``powerful'' than \(b\),
      even though \(d\) has smaller out-degree.
      (trying to understand this phenomenon will be a interesting practice)

    \begin{itemize}
      \item
        A vertex with the \emph{highest} out-degree
        in a digraph may not be a king.
        In \cref{fig: largest out-degree is not king},
        vertex \(a\) has out-degree 3,
        which is the highest out-degree in this graph.
        However, it is not a king,
        since it cannot beats \(k\) by 1 or 2 steps

      \item
        Also, a vertex with \emph{lowest} out-degree
        in a digraph may be a king.
        In \cref{fig: largest out-degree is not king},
        the vertex \(k\) has the smallest out-degree in the graph,
        however it is the only king in this graph.
    \end{itemize}

    \item
      \cref{fig:complicated king example} shows that
      a digraph may have more than one king.
      In fact,~\cite{maurer_king_1980} proved the probability
      that every vertex in a random tournament is a king approaches 1
      as the number of vertices in the graph approaches \(\infty \).
  \end{itemize}

  \begin{figure}
    \centering
    \tikz\graph[tree layout, grow=down, math nodes, sibling distance=2cm,level sep=0.75cm] {
        k -> a;
        a -> {b, c, d};
        c <-> b;
        c <-> d;
        d <-> [bend left] b;
    };
    \caption{the vertex with smallest out-degree is the only king}
    \label{fig: largest out-degree is not king}  % chktex 24
  \end{figure}

  \begin{definition}
    An \keyword{emperor} in a digraph is a vertex that beats every other vertex.
    (see \cref{fig: emperor example})
  \end{definition}

  \begin{figure}
    \centering
    \tikz\graph[tree layout, math nodes, grow=down, sibling distance=2cm,level sep=0.75cm] {
        e -> {s_1, s_2, s_3, "\ldots", s_n};  %chktex 18
    };
    \caption{the vertex \(e\) is the emperor of this graph}
    \label{fig: emperor example}  % chktex 24
  \end{figure}

  \begin{definition}
    an \keyword{\((n, k)\) digraph}
    (\keyword{\((n, k)\) oriented graph},
    \keyword{\((n, k)\) tournament}, etc.)
    is a digraph (oriented graph, tournament, etc.)
    with \(n\) vertices and \(k\) kings.
  \end{definition}

  The following work has been done on kings in tournaments.
  These theorems will be useful when we investigate
  properties of kings in generalized tournaments in later chapters.

  \begin{theorem}\label{the: one king iff emperor}
    A tournament has only one king
    if and only if that king is an emperor.~\cite{maurer_king_1980}
  \end{theorem}

  \begin{theorem}\label{the: no 2 kings}
    A tournament cannot have exactly 2 kings.~\cite{maurer_king_1980}
  \end{theorem}

  \begin{theorem}\label{the: largest out-degree is king}
    In a tournament, a vertex with largest out-degree is a king.~\cite{maurer_king_1980}
  \end{theorem}

  \begin{corollary}\label{the: king exists}
    In a tournament, there always exists at least 1 king.~\cite{maurer_king_1980}
  \end{corollary}

  \begin{theorem}\label{the: (n k) tournament exists}
    for all integers \(n \geq k \geq 1\),
    \((n, k)\)-tournaments exist with following exception:
    \((n, 2)\) with any \(n \geq 2\) and  \((4, 4)\).~\cite{maurer_king_1980}
  \end{theorem}







  \chapter{Semi-complete Digraph}\label{chap: semi-complete digraph}

\section{Definitions}

  In \cref{chap: background},
  we introduced one way to model ``tie'' as
  the non-adjacency between 2 vertices.
  However, if we think of ``beating'' as a weak order
  (like ``subset'' relation),
  then we can define ``double tie'' to capture this idea:

  \begin{definition}
    There exists a \keyword{double tie} between
    vertices \(a\) and \(b\) if \(a \to b\) and \(b \to a\).
  \end{definition}

  With the definition of double tie,
  we can then formalize another model of
  round-robin tournament with double ties.
  This kind of graph is called ``semi-complete digraph''.

  \begin{definition}\label{def: semi-compelete digraph}
    A \keyword{semi-complete digraph} is a digraph
    where every vertex is adjacent to every other vertex,
    but does not beat itself.
    see \cref{fig:semi-complete digraph example}
  \end{definition}

  This definition says, that a semi-complete digraph
  is a digraph such that between each pair of vertices,
  there exists at least one edge between them.

  \begin{figure}
    \centering
    \tikz\graph[simple necklace layout, math nodes, node sep=1.75cm] {
      a -> b;
      a <-> c;
      b -> {c, d};
      c -> e;
      c <-> d;
      d -> {a, e};
      e -> {a, b};
    };
    \caption{an example of semi-complete digraph.}
    \label{fig:semi-complete digraph example} % chktex 24
  \end{figure}

  \begin{definition}
    In a semi-complete digraph,
    if \(a \to b\) and \(b\) does not beat \(a\),
    then vertex \(a\) \keyword{strictly beats}[strictly beat]
    vertex \(b\).
  \end{definition}

  Tournaments are a special case of semi-complete digraphs.
  In tournaments, for each pair of vertices \(a, b\)
  \(a\) always strictly beats \(b\)
  or \(b\) strictly beaten by \(b\).
  We can view a tournament as a semi-complete digraph
  that only has strict beatings; no double tie allowed.
  In other words, tournaments is an oriented graph
  that is also a semi-complete digraph,
  hence the name ``semi-complete oriented graph''
  (see \cref{def: tournaments})

  \begin{definition}
    In a semi-complete digraph,
    the \keyword{double tie set of vertex \(v\)}[double tie set of vertex]
    is the set of vertices that double ties \(v\).
    We denote the double tie set of \(v\) as \(DT_v\).
  \end{definition}

  For example, in \cref{fig:semi-complete digraph example},
  there is a double tie between \(a\) and \(c\),
  and another double tie between \(c\) and \(d\).
  Every other beating relation between
  any other pair of vertices are ``strict beating'' relations.
  For example, \(a\) strictly beats \(b\),
  \(d\) strictly beats \(e\), and \(b\) strictly beats \(d\).

  For vertex \(c\),
  \(DT_c\) (double tie set of \(c\)) is \(\set{a, d}\).
  The set of vertices that strictly beat \(c\),
  which can be expressed as \(D_c - DT_c\), is \(\set{b}\).
  The set of vertices that are strictly beaten by \(c\),
  which can be expressed as \(S_c - DT_c\), is \(\set{e}\).



\section{Properties}\label{sec: semi-complete properties}

  Although tournaments are a special case of semi-complete digraphs,
  many useful properties of tournaments
  are also true for semi-complete digraphs.


  \begin{lemma}\label{the: graph partition lemma}
    for every vertex \(v\) in any semi-complete digraph \(G\),
    \(\set{S_v - DT_v, DT_v, D_v - DT_v, \set{v}}\)
    forms a partition of \(G\).
  \end{lemma}

  \begin{figure}
    \centering
    \tikz\graph[simple necklace layout, math nodes, node sep=2cm] {
      "D_v - DT_v" [draw, circle, minimum size = 2cm] ->  %chktex 18 %chktex 8
      v ->
      "S_v - DT_v"[draw, circle, minimum size = 2cm];  %chktex 18 %chktex 8
      v <-> DT_v [draw, circle, minimum size = 2cm];
    };
    \caption{illustration of \cref{the: graph partition lemma}.}
    \label{fig: graph partition lemma} % chktex 24
  \end{figure}

  \begin{proof}
    \(D_v - DT_v\) is the set of vertices that strictly beat \(v\),
    \(S_v - DT_v\) is the set of vertices that are strictly beaten by \(v\),
    \(DT_v\) is all the vertices that double tie with \(v\).
    They are clearly disjoint by definition,
    and all of them are disjoint to \(\set{v}\) by definition.

    Every vertex in \(V(G)\) needs to be in one of
    the sets \(S_v - DT_v, DT_v, D_v - DT_v, \set{v}\),
    because every other vertex needs to be adjacent to \(v\).
  \end{proof}

  See \cref{fig: graph partition lemma} for a visualization
  for the proof of \cref{the: graph partition lemma}.

  \cref{the: graph partition lemma} and
  \cref{fig: graph partition lemma} are useful
  in proofs of many properties of semi-complete digraphs.

  When we move on to general oriented graphs,
  the lack of this property in oriented graphs will make the
  structure of oriented graphs harder to work with.

  \begin{theorem}\label{the: largest out-degree is a king in semi-compelete digraph}
    All vertices with the maximum out-degree
    in a semi-complete digraph are kings.
  \end{theorem}

  \begin{proof}
    Let \(v\) be a vertex with the maximum out-degree
    in a semi-complete digraph \(G\).

    Suppose \(v\) is not a king in \(G\).
    Since \(v\) is not a king and
    \(v\) beats every vertex in \(S_v\) by exactly 1 step,
    then by \cref{the: graph partition lemma},
    there exists a vertex
    \(d \in D_v - DT_v = V(G) - S_v - \set{v}\)
    that is not beaten by \(v\) by 1 or 2 steps.
    Therefore there cannot be any vertex in \(S_v\) that beats \(d\).
    Because \(d\) is adjacent to every vertex in \(S_v\)
    \(d\) strictly dominates \(v\) and \(S_v\).
    Hence \(|S_d| \geq 1 + |S_v|\).
    Therefore \(|S_d| > |S_v|\),
    that is, \(d\) has larger out-degree than \(v\).
    Contradiction.
  \end{proof}

  \begin{figure}
    \centering
    \tikz\graph[simple necklace layout, math nodes, node sep=2cm] {
      a -> {b, c, d};
      b -> {c, e, a};
      c -> {e, d, b};
      d -> {b, a, c};
      e -> {a, d, c};
    };
    \caption{every vertex has the same out-degree.}
    \label{fig: multiple max out-degree} % chktex 24
  \end{figure}

  \cref{fig: multiple max out-degree} shows that
  there can be more than one vertex with maximum out-degree.
  In fact, in this graph,
  every vertex has the same out-degree 3,
  therefore all of the vertices have the maximum out-degree.
  By \cref{the: largest out-degree is a king in semi-compelete digraph},
  all vertices in this graph are kings.

  \begin{corollary}\label{the: existence theorem}
    For any non-empty semi-complete digraph \(G\),
    there exists at least one king.
  \end{corollary}

  \begin{proof}
    This corollary is a result of \cref{the: largest out-degree is a king in semi-compelete digraph}.
    because out-degrees are non-negative integers,
    and \(V(G)\) is not empty.
    Thus there exists at least one vertex with maximum out degree.
  \end{proof}

  \begin{theorem}\label{the: beaten by king theorem}
    In a semi-complete digraph \(G\),
    every vertex with a non-empty dominant set
    is beaten by a king.
  \end{theorem}

  \begin{proof}
    Let \(v\) be a vertex in \(G\), such that \(D_v\) is not empty.
    Consider the subgraph induced by the vertices in \(D_v\);
    this subgraph is also a semi-complete digraph.
    By \cref{the: existence theorem},
    there is a king \(k\) in the induced subgraph of \(D_v\).
    We will show that \(k\) is also a king of \(G\).
    \begin{itemize}
      \item
        \(k\) dominates \(D_v\) by 1 or 2 steps,
        because \(k\) is a king in \(D_v\).
      \item
        \(k\) dominates \(v\) by exactly 1 step,
        because \(k\) is in \(D_v\).
      \item
        \(k\) dominates \(DT_v\) and \(S_v\) by 1 or 2 steps,
        because \(k\) dominates \(v\),
        which beats all vertices in \(DT_v\) and \(S_v\).
    \end{itemize}

    Then, by \cref{the: graph partition lemma},
    we know that \(k\) dominates
    every vertex in the graph by 1 or 2 steps.
    Therefore \(k\) is a king in \(G\) and \(k\) beats \(v\).
  \end{proof}

  \begin{corollary}\label{the: if only king then emperor}
    If a semi-complete digraph has only one king,
    then that king is an emperor.
  \end{corollary}

  \begin{proof}
    Suppose \(G\) is a semi-complete digraph
    with only one king \(k\), and \(k\) is not an emperor.

    Then there exists \(v\) in the graph,
    such that \(k\) does not beats \(v\).
    Therefore, \(v \to k\), since there is no tie in the graph.
    Therefore, \(k\) has an non-empty dominate set.
    Then by \cref{the: beaten by king theorem},
    \(k\) is beaten by another king.
    Therefore there exists more than one king. Contradiction.
  \end{proof}

  \begin{figure}
    \centering
    \tikz\graph{
      a <-> b;
    };
    \caption{both vertices \(a\) and \(b\)
       are emperors and kings.}
    \label{fig:more than one emperors}  %chktex 24
  \end{figure}
  Notice, unlike in tournaments,
  the converse of \cref{the: if only king then emperor} is not true
  In \cref{fig:more than one emperors},
  we show we can have more than one kings that are emperors.

  \begin{theorem}\label{the: (n k) digraph exists}
    \((n, k)\) semi-complete digraphs
    exist for all \(n \geq k \geq 1\), where \(n, k\) are integers.
  \end{theorem}

  \begin{proof}
    We prove this theorem by induction.
    Construct an \((n, k)\) semi-complete digraph:
    \begin{itemize}
      \item
        When there is only one vertex in the graph
        then the graph is a \((1, 1)\) semi-complete digraph.

      \item
        See \cref{fig:add a non-king vertex}.
        We can add one vertex that is not a king by
        adding a vertex that is strictly beaten
        by all the vertices in the original graph.
        In other words,
        we can construct
        a \((n + 1, k)\) semi-complete digraph
        from any \((n, k)\) semi-complete digraph.

        \begin{figure}
          \centering
          \tikz\graph[layered layout, grow=right, level sep=1cm] {
            " \((n, k)\) digraph " [draw, circle] -> "\(v\)"; %chktex 18
          };
          \caption{an \((n+1, k)\) semi-complete digraph by
            adding a new vertex \(v\)}
          \label{fig:add a non-king vertex}  %chktex 24
        \end{figure}

      \item
        See \cref{fig:add a king}.
        We can add one king by
        adding one vertex that double ties
        all the vertices in the original graph.
        In other words, we can construct
        an \((n + 1, k + 1)\) semi-complete digraph
        from any \((n, k)\) semi-complete digraph.

        \begin{figure}
          \centering
          \tikz\graph[layered layout, grow=right, level sep=1cm] {
            " \((n, k)\) digraph " [draw, circle] <-> "\(a\)"; %chktex 18
          };
          \caption{an \((n+1, k+1)\) semi-complete digraph by
            adding a new king \(a\).}
          \label{fig:add a king}  %chktex 24
        \end{figure}

    \end{itemize}

    Therefore, we can obtain any \((n, k)\) flock by:
    start with a \((1,1)\) semi-complete digraph
    first add \(k - 1\) kings to
    get a \((k, k)\) semi-complete digraph;
    then add \(n - k\) non-king vertices
    to get an \((n, k)\) semi-complete digraph.
  \end{proof}

  \begin{figure}
    \centering
    \begin{subfigure}[b]{.45\linewidth}
      \centering
      \tikz\graph[simple necklace layout, math nodes, node sep=1.75cm] {
        a;  %chktex 18
      };
      \caption{start with \((1, 1)\) digraph.}
    \end{subfigure}
    \begin{subfigure}[b]{.45\linewidth}
      \centering
      \tikz\graph[simple necklace layout, math nodes, node sep=1.75cm] {
        a <-> b;
      };
      \caption{add king \(b\).}
    \end{subfigure}
    \begin{subfigure}[b]{.45\linewidth}
      \centering
      \tikz\graph[simple necklace layout, math nodes, node sep=1.75cm] {
        a <-> b;
        {a, b} -> c;
      };
      \caption{add non-king \(c\).}
    \end{subfigure}
    \begin{subfigure}[b]{.45\linewidth}
      \centering
      \tikz\graph[simple necklace layout, math nodes, node sep=1.75cm] {
        a <-> b;
        {a, b} -> c;
        {a, b, c} -> d;
      };
      \caption{add non-king \(d\).}
    \end{subfigure}
    \caption{to construct a \((4, 2)\) semi-complete digraph.}
    \label{fig: (4 2) digraph construction} % chktex 24
  \end{figure}

  In \cref{fig: (4 2) digraph construction},
  we give an example of how to construct a
  \((4,2)\) semi-complete digraph
  using the inductive algorithm we introduced in
  \cref{the: (n k) digraph exists}:
  \begin{enumerate}
    \item
      Start with a single vertex,
      which is a \((1, 1)\) semi-complete digraph.
    \item
      Add a king \(b\), by letting it double tie with \(a\)
      which gives us a \((2, 2)\) semi-complete digraph.
    \item
      Add a non-king \(c\),
      by letting every vertex in the
      \((2,2)\) semi-complete digraph beat \(c\).
      which gives us a \((3, 2)\) semi-complete digraph.
    \item
      Add a non-king \(d\),
      by letting every vertex
      in the \((3, 2)\) semi-complete digraph beat \(d\),
      which gives us a \((4, 2)\) semi-complete digraph.
  \end{enumerate}

  Using the method in \cref{the: (n k) digraph exists},
  when we add the \(i_{th}\) king,
  we are creating \((i - 1)\) ties,
  so adding \(k\) kings would require a total of
  \(1 + 2 + \cdots + (k - 1)\) double ties.
  The summation can be simplified to \(\frac{k(k-1)}{2}\).
  Thus, the number of double ties blows up with
  the increase of \(k\) (number of kings).

  Therefore we want to investigate how to use double ties
  more ``efficiently'':
  what is the minimum number of double ties needed to get an
  \((n, k)\) semi-complete digraph.

  \begin{lemma}\label{the: (n 2) digraph with one tie}
    All \((n, 2)\) semi-complete digraphs with \(n \geq 2\)
    can be constructed with only one double tie.
  \end{lemma}

  \begin{proof}
    use the method described in \cref{the: (n k) digraph exists}.

    First add 1 king to a \((1, 1)\) semi-complete digraph
    to get a \((2, 2)\) semi-complete digraph
    (one double tie added).
    Then add \((n - 2)\) non-king vertices to get
    an \((n, 2)\) semi-complete digraph (no double tie added).

    Therefore we add only one double tie to construct
    any \((n, 2)\) semi-complete digraph.
  \end{proof}

  \begin{lemma}\label{the: (4 4) digraph exists with one tie}
    there exists a \((4, 4)\) semi-complete digraph
    with only one double tie.
  \end{lemma}

  \begin{proof}
    \begin{figure}
      \centering
      \tikz\graph[simple necklace layout, math nodes, node sep=1.75cm]{
        a -> {b, d};
        b -> d -> c;
        c -> a;
        c <-> b;
      };
      \caption{a \((4,4)\) semi-complete digraph
        with only one double tie}
      \label{fig: (4 4) digraph with one tie}  %chktex 24
    \end{figure}
    \cref{fig: (4 4) digraph with one tie} gives an example
    of a \((4,4)\) semi-complete digraph with only one double tie.
    The only double tie is between \(b\) and \(c\).
  \end{proof}

  \begin{theorem}
    for all \(n \geq k \geq 1\),
    there exists an \((n, k)\) semi-complete digraph with
    at most one double tie.
  \end{theorem}

  \begin{proof}
    Recall \cref{the: (n k) tournament exists},
    there exists an \((n, k)\) tournament
    with the exception of \((n, 2)\) and \((4, 4)\)
    Because tournaments are also semi-complete digraphs,
    therefore, we can use \cref{the: (n k) tournament exists}
    along with \cref{the: (n 2) digraph with one tie}
    and \cref{the: (4 4) digraph exists with one tie},
    to get the result.

    The proofs of \cref{the: (n k) tournament exists},
    \cref{the: (n 2) digraph with one tie},
    and \cref{the: (4 4) digraph exists with one tie}
    all provide constructions for these semi-complete digraphs.
    Therefore we can construct an \((n, k)\)
    semi-complete digraph with at most one double tie,
    for all \(n \geq k \geq 1\).
  \end{proof}



  \chapter{Oriented Graph}\label{chap: general oriented graph}

The properties of semi-complete digraphs basically inherit
the properties of tournaments,
and most of the proofs are almost the same.

We now move on to another family of graphs that are less
similar to tournaments: oriented graphs.
We will investigate general properties of oriented graphs
in this chapter.
In later \cref{chap: quasi-transitive}, we will focus on
specific families of oriented graphs.

Recall \cref{def:oriented graph},
a oriented graph is a digraph that doesn't
have self-loops (a vertex beats itself)
and double ties.

\begin{theorem}\label{the: emperor then only king in oriented graph}
  If there is an emperor \(k\) in an oriented graph \(G\),
  then \(k\) is the only king in the graph.
\end{theorem}
\begin{proof}
  The emperor \(k\) beats every vertex in one step,
  therefore it is a king.

  Because there is no double ties in oriented graph,
  if \(k\) is an emperor,
  then no vertex in \(G\) can beats \(k\),
  hence no vertex can beat \(k\) in one or two steps.
  Therefore, \(k\) is the only king in the graph.
\end{proof}

\begin{corollary}\label{the: if vertex with out-degree n-1 then only one king}
  In an oriented graph with \(n\) vertices,
  if there exists a vertex with out-degree \(n-1\),
  then there is only one king in the graph.
\end{corollary}

\begin{proof}
  A vertex with out-degree \(n-1\) needs to beat every vertex
  in the graph except itself.
  Therefore that vertex is an emperor,
  and because of \cref{the: emperor then only king in oriented graph},
  there will be only one king in this graph.
\end{proof}

\begin{lemma}\label{the: add edge only add king}
  For a oriented graph \(G\),
  if we add a new edge into \(G\) to get \(G'\),
  then \(G'\) will not have less kings than \(G\).
\end{lemma}

\begin{proof}
  Need to show that for every king \(k\) in \(G\),
  \(k\) is also a king in \(G'\).
  We can see that for every pair of vertices \(a, b\)
  if \(a \to b\) in \(G\), then \(a \to b\) in \(G'\),
  since we are not removing any edges.

  \begin{itemize}
    \item
      If \(k\) beats a vertex \(v\) by one step in \(G\),
      then \(k \to v\) in \(G\),
      therefore \(k \to v\) in \(G'\)
    \item
      If \(k\) beats a vertex \(v\) by 2 steps in \(G\),
      then exists vertex \(a\) such that
      \(k \to a \to v\) in \(G\), then \(k \to a \to v\) in \(G'\)
      and \(k\) beats \(v\) by 2 steps in \(G'\).
  \end{itemize}

  Therefore, \(k\) is also a king in \(G'\),
  then every king in \(G\) is preserved in \(G'\).
  Therefore, \(G'\) have less king than \(G\).
\end{proof}

\begin{lemma}\label{the: no (1 0) oriented graph}
  There do not exists a \((1, 0)\) oriented graph.
\end{lemma}
\begin{proof}
  If any oriented graph only have 1 vertex,
  then that vertex by definition is a king.
  Therefore, there does not exist a \((1, 0)\) oriented graph.
\end{proof}

\begin{lemma}\label{the: no (2 2) oriented graph}
  There does not exist a \((2, 2)\) oriented graph
\end{lemma}

\begin{proof}
  \begin{figure}
    \centering
    \begin{subfigure}{0.45\linewidth}
      \centering
      \tikz\graph[layered layout, math nodes, grow=right, sibling distance=2cm, level sep=0.75cm] {
      a; b;
      };
      \caption{2 vertices with no edge.}
      \label{fig: all oriented graph with 2 vertices: no edge}  % chktex 24
    \end{subfigure}
    \begin{subfigure}{0.45\linewidth}
      \centering
      \tikz\graph[layered layout, math nodes, grow=right, sibling distance=2cm, level sep=0.75cm] {
      a -> b;
      };
      \caption{2 vertices with 1 edge.}
      \label{fig: all oriented graph with 2 vertices: 1 edge}  % chktex 24
    \end{subfigure}
    \caption{all the oriented graph with 2 vertices.}
    \label{fig: all oriented graph with 2 vertices}  % chktex 24
  \end{figure}
  \cref{fig: all oriented graph with 2 vertices}
  shows all the possible oriented graphs with 2 vertices.
  Notice, the oriented graph in
  \cref{fig: all oriented graph with 2 vertices: no edge}
  has 0 king,
  and the oriented graph in
  \cref{fig: all oriented graph with 2 vertices: 1 edge}
  has only 1 king.
  Therefore, there does not exists a \((2,2)\) oriented graph.
\end{proof}
\begin{lemma}\label{the: no (3 2) oriented graph}
  There does not exist a \((3, 2)\) oriented graph
\end{lemma}

\begin{proof}
  We consider all the oriented graph with 3 vertices.

  First, consider the graph with maximum out-degree 0.
  The graph with maximum out-degree 0 will have no edge,
  therefore there cannot be a king,
  since no vertex can beats other vertices by 2 steps.

  \begin{figure}
    \centering
    \begin{subfigure}{0.3\linewidth}
      \centering
      \tikz\graph[simple necklace layout, math nodes, node sep=1cm] {
      a -> b; c;
      };
      \caption{3 vertices with no edge.}
      \label{fig: 3 oriented graph with max out-degree 1: 1 edge}  % chktex 24
    \end{subfigure}
    \begin{subfigure}{0.3\linewidth}
      \centering
      \tikz\graph[simple necklace layout, math nodes, node sep=1cm] {
      a -> b -> c;
      };
      \caption{3 vertices with 1 edge.}
      \label{fig: 3 oriented graph with max out-degree 1: 2 edge}  % chktex 24
    \end{subfigure}
    \begin{subfigure}{0.3\linewidth}
      \centering
      \tikz\graph[simple necklace layout, math nodes, node sep=1cm] {
      a -> b -> c -> a;
      };
      \caption{3 vertices with 3 edges.}
      \label{fig: 3 oriented graph with max out-degree 1: 3 edge}  % chktex 24
    \end{subfigure}
    \caption{all the oriented graph with 3 vertices and maximum out-degree 1.}
    \label{fig: 3 oriented graph with max out-degree 1}  % chktex 24
  \end{figure}

  Then, consider the graph with maximum out-degree 1.
  In \cref{fig: 3 oriented graph with max out-degree 1},
  We show every oriented graph with 3 vertices and
  maximum out-degree 1.
  In \cref{fig: 3 oriented graph with max out-degree 1: 1 edge},
  there is no king;
  in \cref{fig: 3 oriented graph with max out-degree 1: 2 edge},
  there is 1 king;
  in \cref{fig: 3 oriented graph with max out-degree 1: 3 edge},
  every vertex is a king, therefore it has 3 kings.

  Finally, we consider the oriented graph with maximum out-degree 2.
  By \cref{the: if vertex with out-degree n-1 then only one king},
  because we have 3 vertex in the graph,
  and at least one vertex need to have degree 2,
  there there can only be 1 king.

  Therefore, there is no \((3,2)\) oriented graph.
\end{proof}

\begin{lemma}\label{the: no (4 4) oriented graph}
  There does not exist a \((4,4)\) oriented graph.
\end{lemma}
\begin{proof}
  Assume there exists a \((4, 4)\) oriented graph.
  When we add edges to this \((4, 4)\) oriented graph:
  \begin{itemize}
    \item
      the number of kings cannot increase,
      since a graph cannot have more kings than vertices.
    \item
      the number of kings cannot decrease,
      because of \cref{the: add edge only add king}
  \end{itemize}

  Then we can keep adding edges until
  every pair of vertices are adjacent
  and get a \((4,4)\) tournament.
  By \cref{the: (n k) tournament exists},
  there do not exists \((4,4)\) tournaments.

  Therefore, there cannot exists \((4,4)\) oriented graph.
\end{proof}

After investigating what kinds of oriented graphs do not exist,
we then move on to all the possible oriented graphs that
we can construct.

\begin{lemma}\label{the: (n 2) oriented graph}
  There exists an \((n, 2)\) oriented graph for \(n \geq 4\).
\end{lemma}

\begin{proof}
  \begin{figure}
    \centering
    \tikz\graph[tree layout, grow=down, math nodes, sibling distance=1cm,level sep=1cm] {
    a -> b -> {c_1, c_2, "\ldots", c_{n-3}, c_{n-2}};  %chktex 18
    c_1 -> a;
    };
    \caption{only \(a\) and \(b\) are kings for \(n \geq 4\).}
    \label{fig: (n 2) oriented graph}  %chktex 24
  \end{figure}
  We can see in \cref{fig: (n 2) oriented graph} that
  \(c_1\) cannot dominate \(c_2\) by 1 or 2 steps,
  and other \(c_i (i \neq 1)\) cannot dominate any other vertex,
  because the out-degree of these vertices are 0.

  \(a\) is a king because \(a \to b\) and \(b\) beats every \(c_i\),
  therefore \(a\) beats \(b\) by one step,
  and \(a\) beats every \(c_i\) by 2 steps.
  \(b\) beats every \(c_i\) by 2 steps,
  and \(b\) beats \(a\) by 2 steps: \(b \to c_1 \to a\).
\end{proof}

\begin{lemma}\label{the: (n 0) oriented graph}
  There exists \((n, 0)\) oriented graph for \(n \geq 0\)
  except \(n = 1\)
\end{lemma}

\begin{proof}
  For a oriented graph with \(n\) vertices with no edge,
  every vertex cannot beat any other vertex.
  Therefore, the graph has 0 kings
  and the graph is a \((n, 0)\) oriented graph.
\end{proof}

\begin{theorem}\label{the: (n k) oriented graph}
  There exists an \((n, k)\) oriented graph for all \(n \geq k \geq 0\),
  with the exception of \((1, 0)\), \((2, 2)\), \((3, 2)\),
  and \((4, 4)\) oriented graph.
\end{theorem}

\begin{proof}
  \cref{the: (n k) tournament exists} shows that
  there exists an \((n, k)\) tournament for all \(n \geq k \geq 1\)
  with the exception of \((n, 2)\), and \((4, 4)\).

  Because tournaments are also oriented graphs and
  by \cref{the: no (2 2) oriented graph},
  \cref{the: no (3 2) oriented graph},
  \cref{the: no (1 0) oriented graph},
  \cref{the: no (4 4) oriented graph},
  \cref{the: (n 2) oriented graph},
  and \cref{the: (n 0) oriented graph},
  we can conclude that the theorem is correct.
\end{proof}

We generalize the result from~\cite{maurer_king_1980}
on tournaments to oriented graphs
and show that there are only 4
\((n, k)\) oriented graphs that do not exists.

Following the idea from \cref{chap: semi-complete digraph},
one of the questions to ask is how can we use
ties more ``efficiently''.
The construction method in the proof of
\cref{the: (n 2) oriented graph} is very inefficient.

Here we present a ``better'' way to construct
these oriented graphs that only uses one tie.

\begin{lemma}\label{the: (n 2) with one tie}
  There exists an \((n, 2)\) oriented graph
  with only one tie, for \(n \geq 4\).
\end{lemma}

\begin{proof}
  \begin{figure}
    \centering
    \tikz\graph[tree layout, math nodes, grow=down, sibling distance=2cm,level sep=0.75cm] {
    a -> b -> {c, T_{n-3} [draw, circle]};
    T_{n-3} <- c;
    T_{n-3} -> [bend right] a;
    };
    \caption{the constructive proof for
    \cref{the: (n 2) with one tie}}
    \label{fig: (n 2) with one tie} %chktex 24
  \end{figure}

  See \cref{fig: (n 2) with one tie},
  \(T_{n - 3}\) is a tournament of \(n - 3\) vertices.
  In this graph, the only tie is between \(a\) and \(c\)
  and the only kings are \(a\) and \(b\).

  \(a\) is a king because, \(a \to b \to c\)
  and \(a \to b \to T_{n - 3}\),
  therefore \(a\) beats \(b\) by 1 step
  and \(a\) beats \(c\) and \(T_{n-3}\) by 2 steps.
  \(b\) is a king because, \(b \to c\)
  and \(b \to T_{n - 3} \to a\)
  (because \(T_{n-3}\) is not empty),
  therefore \(b\) beats \(c\) and \(T_{n - 3}\),
  and \(b\) beats \(a\) by 2 steps.

  \(c\) is not a king, there is no path from \(c\) to \(b\).
  Any vertex \(v\) in \(T_{n-3}\) cannot be a king,
  because the closest path between \(v\) and \(c\)
  is \(v \to a \to b \to c\) which has length 3.
\end{proof}

\begin{lemma}\label{the: (n 0) oriented graph with 1 tie}
  There exists \((n, 0)\) oriented graph where \(n \neq 1\)
  with at most 1 tie.
\end{lemma}
\begin{proof}
  First, we can see that \((0, 0)\) oriented graph exists,
  it is just an empty graph with no vertex and edge.

  \begin{figure}
  \centering
    \tikz\graph[layered layout, math nodes, grow=right, sibling distance=2cm, level sep=0.75cm] {
      a; b;
    };
    \caption{\((2, 0)\) oriented graph.}
    \label{fig: (2 0) oriented graph with one tie}  % chktex 24
  \end{figure}
  Then, in \cref{fig: (2 0) oriented graph with one tie}
  we show that \((2, 0)\) oriented graph with one tie exists:
  it is just 2 vertices and no edge between them.

  \begin{figure}
  \centering
    \tikz\graph[layered layout, grow=right, sibling distance=2cm, level sep=2cm] {
      "\((n, 0)\) oriented graph" [draw, circle]  %chktex 18
      -> "\(s\)";  %chktex 18
    };
    \caption{construct \((n+1, 0)\) oriented graph from \((n, 0)\) oriented graph.}
    \label{fig: (n+1 0) oriented graph with one tie}  % chktex 24
  \end{figure}
  In \cref{fig: (n+1 0) oriented graph with one tie},
  we give a way to construct an \((n+1, 0)\) oriented graph
  from an \((n, 0)\) oriented graph.
  We denote the \((n, 0)\) oriented graph as \(G\).

  We need to show the resulting graph
  in \cref{fig: (n+1 0) oriented graph with one tie}
  has no king:
  first, \(s\) cannot be a king,
  because it does not beat any vertex.
  We then show any vertex \(v \in V(G)\) is not a king.
  Because \(v\) is not a king in \(G\),
  there exists a vertex \(v' \in V(G)\)
  such that \(v\) cannot beat it in 1 or 2 steps in \(G\).
  Then \(v\) still cannot beats \(v'\) in this graph,
  because \(s\) is the only added vertex,
  and the path \(v \to s \to v'\) do not exists.
\end{proof}

With \cref{the: (n 2) with one tie} and
\cref{the: (n 0) oriented graph with 1 tie},
we can prove the following theorem.

\begin{theorem}\label{the: (n k) oriented graph with one tie}
  There exists an \((n, k)\) oriented graph
  with at most one tie for all \(n \geq k \geq 0\),
  with the exception of \((1, 0)\), \((2, 2)\),
  \((3, 2)\), and \((4, 4)\) oriented graphs.
\end{theorem}

\begin{proof}
  Almost the same proof as \cref{the: (n k) oriented graph},
  just substitute \cref{the: (n 2) oriented graph},
  \cref{the: (n 0) oriented graph} with
  \cref{the: (n 2) with one tie}
  and \cref{the: (n 0) oriented graph with 1 tie},
  respectively.
\end{proof}

in other words,
\cref{the: (n k) oriented graph with one tie}
states that for all \((n, k)\) oriented graphs
that exists, they can be constructed with only one tie.

  \chapter{Quasi-transitive Oriented Graph}\label{chap: quasi-transitive}

\section{Definitions}

As mentioned in \cref{chap: general oriented graph}
In an oriented graph, for any pair of vertices \(a, b\),
either \(a\) beats \(b\), \(b\) beats \(a\)
or \(a\) ties \(b\).
So, there are \(3^\binom{n}{2}\) different oriented graphs
on n vertices.
with no other restrictions, it is difficult to
produce theorems about them.
Therefore we decided to focus on another special case of oriented graph,
which is called ``quasi-transitive oriented graph'':

\begin{definition}\label{def: quasi-transitive oriented graph}
  a \keyword{quasi-transitive oriented graph} \(G\) is
  an oriented graph such that,
  for all vertices \(a, b, c \in V(G)\), if \(a \to b \to c\),
  then \(a\) is adjacent to \(c\).
  See \cref{fig: quasi-transitive oriented graph}
\end{definition}

\begin{figure}
  \centering
  \begin{subfigure}[b]{0.45\linewidth}
    \centering
    \tikz\graph[simple necklace layout, math nodes, node sep=2cm] {
      b <- a;
      b -> c;
      a -> c;
    };
  \end{subfigure}
  \begin{subfigure}[b]{0.45\linewidth}
    \centering
    \tikz\graph[simple necklace layout, math nodes, node sep=2cm] {
      b <- a;
      b -> c;
      a <- c;
    };
  \end{subfigure}
  \caption{if \(a \to b \to c\) then \(a\) is adjacent to \(c\).}
  \label{fig: quasi-transitive oriented graph}  %chktex 24
\end{figure}

Because every vertex is adjacent to every other vertex
 in a tournament,
tournaments is a special case of quasi-transitive oriented graph.

Although tournaments are just a small subset
of quasi-transitive oriented graphs,
and many quasi-transitive oriented graph are very different
from tournaments
(for example, every graph with no edge is
 a quasi-transitive oriented graph),
quasi-transitive oriented graphs inherits many properties of
tournaments~\cite{bangjensen_quasitransitive_1995}.

\section{Ties and Tie Paths}

Properties of ties in quasi-transitive digraphs are
vastly important in this chapter.

\begin{lemma}\label{the: arrow direction lemma}
  In a quasi-transitive oriented graph \(G\),
  if vertex \(a\) ties vertex \(b\),
  then for each vertex \(v\) that is adjacent to both \(a\) and \(b\),
  either \(v\) beats both \(a\) and \(b\),
  or \(v\) is beaten by both \(a\) and \(b\)
  (\(\set{a, b} \to v\) or \(v \to \set{a, b}\)).
\end{lemma}

\begin{proof}
  Because \(v\) is adjacent to both \(a\) and \(b\),
  we show all the possible beating relationships
  between \(v\) and \(a, b\) in
  \cref{fig: possible beating between v and a b}.
  Because \(G\) is a quasi-transitive oriented graph,
  if \(a \to v \to b\) or \(b \to v \to a\)
  then \(a, b\) have to be adjacent.
  Therefore, only the possible beating relationships are
  \(\set{a, b} \to v\) or \(v \to \set{a, b}\).
\end{proof}

\begin{figure}
  \centering
  \begin{subfigure}[b]{0.2\linewidth}
    \centering
    \tikz\graph[simple necklace layout, math nodes, node sep=1.75cm] {
      v -> {a, b};
      a --[dashed] b;
    };
    \caption{\(v \to \set{a, b}\).}  % chktex 44
  \end{subfigure}
  \begin{subfigure}[b]{0.2\linewidth}
    \centering
    \tikz\graph[simple necklace layout, math nodes, node sep=1.75cm] {
      v <- {a, b};
      a --[dashed] b
    };
    \caption{\(\set{a, b} \to v\).} % chktex 44
  \end{subfigure}
  \begin{subfigure}[b]{0.2\linewidth}
    \centering
    \tikz\graph[simple necklace layout, math nodes, node sep=1.75cm] {
      v <- a;
      v -> b;
      a --[dashed] b
    };
    \caption{\(a \to v \to b\).}
  \end{subfigure}
  \begin{subfigure}[b]{0.2\linewidth}
    \centering
    \tikz\graph[simple necklace layout, math nodes, node sep=1.75cm] {
      v -> a;
      v <- b;
      a --[dashed] b
    };
    \caption{\(b \to v \to a\).}
  \end{subfigure}
  \caption{all the possible beating relations between
    vertex \(v\) and vertices \(a, b\)}
  \label{fig: possible beating between v and a b}  % chktex 24
\end{figure}

\cref{fig: arrow direction lemma: uncondensed} shows a nice
visualization of \cref{the: arrow direction lemma}.
Consider the set of vertices that are adjacent to both
\(a\) and \(b\).
In this figure the set is
\( \set{d_1, d_2, d_3, s_1, s_2, s_3, s_4} \).
This set is partitioned into 2 parts:
those vertices that dominates both \(a\) and \(b\)
\(\set{d_1, d_2, d_3}\),
and those vertices that are beaten by both \(a\) and \(b\)
\(\set{s_1, s_2, s_3, s_4}\)
The behavior of \(\set{a, b}\) is somewhat like a vertex,
see \cref{fig: arrow direction lemma: condensed}.

\begin{figure}
  \centering
  \begin{subfigure}[b]{0.45\linewidth}
    \centering
    \tikz\graph[layered layout, grow=right, math nodes, level sep=2cm] {
      d_1 -> {a, b} -> s_1;
      d_2 -> {a, b} -> s_2;
      d_3 -> {a, b} -> s_3;
      {a, b} -> s_4;
      / [draw] // {a, b};
    };
    \caption{all the vertices are adjacent to \(a, b\).}
    \label{fig: arrow direction lemma: uncondensed}  % chktex 24
  \end{subfigure}
  \begin{subfigure}[b]{0.45\linewidth}
    \centering
    \tikz\graph[layered layout, grow=right, math nodes, level sep=2cm] {
      d_1 -> v -> s_1;
      d_2 -> v -> s_2;
      d_3 -> v -> s_3;
      v -> s_4;
    };
    \caption{the set \(\set{a, b}\) behaves like \(v\).}  %chktex 44
    \label{fig: arrow direction lemma: condensed}  % chktex 24
  \end{subfigure}
  \caption{only look at all the vertices adjacent
  to both \(a\) and \(b\),
  then \(\set{a, b}\) behaves like a vertex.}
  \label{fig: arrow direction lemma}  % chktex 24
\end{figure}

However, this nice visualization in
\cref{fig: arrow direction lemma} has a very strong prerequisite,
that is, all the vertices need to be adjacent to
both \(a\) and \(b\).
What will happen when we add another vertex that
ties either \(a\) or \(b\) and is adjacent to \(v\)?

\begin{figure}
  \centering
  \begin{subfigure}[b]{0.45\linewidth}
    \centering
    \tikz\graph[simple necklace layout, math nodes, node sep=2cm] {
      v <- {a, b};
      a --[dashed] b --[dashed] c;
      v --c;
    };
    \caption{add a vertex that \(c\) ties \(b\), but not \(v\).}
    \label{fig: add tie to arrow direction lemma: start} % chktex 24
  \end{subfigure}
  \begin{subfigure}[b]{0.45\linewidth}
    \centering
    \tikz\graph[simple necklace layout, math nodes, node sep=2cm] {
      v <- {a, b};
      a --[dashed] b --[dashed] c;
      v <- c;
    };
    \caption{by \cref{the: arrow direction lemma}, \(c \to v\).}
    \label{fig: add tie to arrow direction lemma: finish} % chktex 24
  \end{subfigure}
  \caption{ties are ``transmitting'' arrow directions.}
  \label{fig: add tie to arrow direction lemma}  % chktex 24
\end{figure}

In \cref{fig: add tie to arrow direction lemma: start}
we add a vertex \(c\) that ties \(b\),
but needs to be adjacent to \(v\)
(we use an edge without arrow to denote adjacency).
Because \(b\), \(c\) are both adjacent to \(v\),
and \(c\) ties \(b\),
therefore either \(\set{b, c} \to v\) or \(v \to \set{b, c}\).
However, in the graph, \(b \to v\), therefore \(c \to v\).
Thus, we get \cref{fig: add tie to arrow direction lemma: finish}.

We find out that because of the ties between \(a, b, c\),
the arrow direction between \(v, a\) and \(v, b\),
got ``transmitted'' to \(v, c\)
via the \cref{the: arrow direction lemma}.
We formalize this ``arrow transmission'' idea:

\begin{definition}
  In a digraph, a \keyword{tie path} from
  vertex \(a_0\) to vertex \(a_n\),
  or a tie path between vertices \(a_0\) and \(a_n\),
  is a sequence of vertices
  \([a_0, a_1, a_2, \ldots, a_{n-1}, a_n]\),
  such that for all \(0 \leq k < n\), \(a_k\) ties \(a_{k + 1}\).
  If there is \(n + 1\) vertices in that sequence,
  we will say \keyword{the length of the tie path} is \(n\).
  Note that all the \(a_i\)'s do not have to be distinct.
\end{definition}

A tie path is similar to a path
(called a ``walk'' in~\cite{west_introduction_2001})
in an undirected graph;
just substitute all the edges in a path with ties.
See \cref{fig: tie path and path}.


\begin{figure}
  \centering
  \begin{subfigure}[b]{0.45\linewidth}
    \centering
    \tikz\graph[layered layout, math nodes, level sep=1cm] {
      a --b --c --d
    };
    \caption{a path from \(a\) to \(d\).}
  \end{subfigure}
  \begin{subfigure}[b]{0.45\linewidth}
    \centering
    \tikz\graph[layered layout, math nodes, level sep=1cm] {
      a --[dashed] b --[dashed] c --[dashed] d
    };
    \caption{a tie path from \(a\) to \(d\).}
  \end{subfigure}
  \caption{tie paths are just like paths.}
  \label{fig: tie path and path}  % chktex 24
\end{figure}

Tie path are useful in analyzing quasi-transitive
oriented graphs:

\begin{lemma}\label{the: tie path connection lemma}
  In a digraph,
  if there exists a tie path from vertex \(a\) to vertex \(b\)
  and there exists a tie path from vertex \(b\) to vertex \(c\),
  then there exists a tie path from \(a\) to \(c\).
\end{lemma}
\begin{proof}
  Write the tie path from \(a\) to \(b\) as:
  \([a, a_0, a_1, \ldots, a_n, b]\),
  and the tie path from \(b\) to \(c\) as:
  \([b, b_0, b_1, \ldots, b_m, c]\).
  Then there exists a tie path
  \([a, a_0, a_1, \ldots, a_n, b, b_0, b_1, \ldots, b_m, c]\)
  from \(a\) to \(c\).
  See \cref{fig: tie path connection}.
  \begin{figure}
    \centering
    \tikz\graph[simple necklace layout, math nodes, node sep=0.5cm] {
      a --[dashed] a_0 --[dashed] a_1 --[dashed] dots_a [as=\ldots]
      --[dashed] a_n --[dashed] b --[dashed] b_0 --[dashed]
      b_1 --[dashed] dots_b [as=\ldots] --[dashed] b_m  --[dashed] c;
    };
    \caption{tie path from \(a\) to \(b\), and from \(b\) to \(c\).}
    \label{fig: tie path connection}  % chktex 24
  \end{figure}
\end{proof}

\begin{lemma}\label{the: tie path division lemma}
  In a digraph,
  if there exists a tie path \([a_0, a_1, \ldots, a_{n-1}, a_n]\),
  then for all \(p \neq q\) \(0 \leq p \leq n\) and \(0 \leq q \leq n\),
  there exists a tie path between \(p\) and \(q\).
\end{lemma}
\begin{proof}
  Without loss of generality, assume \(p < q\).
  Then we can find a tie path
  \([a_p, a_{p+1}, \ldots, a_{q-1}, a_q]\),
  see \cref{fig: tie path division}.
  \begin{figure}
    \centering
    \tikz\graph[simple necklace layout, math nodes, node sep=0.5cm] {
      a_0 --[dashed] a_1 --[dashed] dots_a[as=\ldots]
      --[dashed] a_{p-1} --[dashed] a_p --[dashed]
      a_{p+1} --[dashed] dots_b[as=\ldots] --[dashed] a_{q-1}
      --[dashed] a_q --[dashed] a_{q+1} --[dashed] dots_c[as=\ldots]
      --[dashed] a_{n-1} --[dashed] a_{n};
      / [label = left:tie path from \(a_p\) to \(a_q\), draw] //
        {a_p, a_{p+1}, dots_b, a_{q-1}, a_q};
    };
    \caption{the tie path from \(a_0\) to \(a_n\).}
    \label{fig: tie path division}  % chktex 24
  \end{figure}
\end{proof}

After looking at
\cref{fig: add tie to arrow direction lemma},
we can hypothesize that tie paths can transmit
arrow directions across the whole path
in quasi-transitive oriented graph.
See \cref{fig: tie path transmit arrow}.

\begin{figure}
  \centering
  \begin{subfigure}[b]{0.45\linewidth}
    \centering
    \tikz\graph[simple necklace layout, math nodes, node sep=1cm] {
      a --[dashed] b --[dashed] "\ldots" --[dashed] c --[dashed] d;  % chktex 18
      ""; % chktex 18
      a <- f;
      {b, "\ldots", c, d} --f;  % chktex 18
      ""; ""; % chktex 18
    };
    \caption{\(a\) is beaten by \(f\),
      and other vertices adjacent to \(f\)}
  \end{subfigure}
  \begin{subfigure}[b]{0.45\linewidth}
    \centering
    \tikz\graph[simple necklace layout, math nodes, node sep=1cm] {
      a --[dashed] b --[dashed] "\ldots" --[dashed] c --[dashed] d;  % chktex 18
      ""; % chktex 18
      {a, b, "\ldots", c, d} <- f;  % chktex 18
      ""; ""; % chktex 18
    };
    \caption{the arrow direction is
      ``transmitted'' through the tie path}
  \end{subfigure}
  \caption{tie path transmits the arrow direction.}
  \label{fig: tie path transmit arrow}  % chktex 24
\end{figure}

\begin{lemma}\label{the: tie transimission}
  For every tie path in a quasi-transitive oriented graph,
  if vertex \(v\) is adjacent to all the vertices
  on the tie path,
  then \(v\) beats every vertex on the tie path,
  or \(v\) is beaten by every vertex on the tie path.
\end{lemma}

\begin{figure}
  \centering
  \begin{subfigure}[b]{0.45\linewidth}
    \centering
    \tikz\graph[simple necklace layout, math nodes, node sep=1cm] {
      a_1 --[dashed] a_2 --[dashed] "\ldots" % chktex 18
       --[dashed] a_n --[dashed] a_{n+1};
      ""; % chktex 18
      {a_1, a_2, "\ldots", a_n} <- v;  % chktex 18
      a_{n+1} --v;
      ""; ""; % chktex 18
    };
    \caption{\(a_{n+1}\) is adjacent to \(v\).}  % chktex 44
  \end{subfigure}
  \begin{subfigure}[b]{0.45\linewidth}
    \centering
    \tikz\graph[simple necklace layout, math nodes, node sep=1cm] {
      a_1 --[dashed] a_2 --[dashed] "\ldots" % chktex 18
       --[dashed] a_n --[dashed] a_{n+1};
      ""; % chktex 18
      {a_1, a_2, "\ldots", a_n, a_{n+1}} <- v;  % chktex 18
      ""; ""; % chktex 18
    };
    \caption{then \(a_{n+1}\) has to be beaten by \(v\).}  % chktex 44
  \end{subfigure}
  \caption{the induction step of the proof of
    \cref{the: tie transimission}}
  \label{fig: tie transimission proof}  % chktex 24
\end{figure}

\begin{proof}
  We can prove this lemma via induction,
  follow the intuition in
  \cref{fig: add tie to arrow direction lemma}.

  Start with a tie path of length 1,
  then we have \(a_0\) ties \(a_1\) in this tie path,
  and \(v\) is adjacent to both of them.
  \cref{the: arrow direction lemma} proves the result
  for path of length 1.

  Assume the property holds for
  any tie path of length \(n\) and vertex \(v\),
  then we need to prove this property holds for
  any tie path of length \(n + 1\).
  See \cref{fig: tie transimission proof},
  take a tie path \([a_0, a_1, \ldots, a_n, a_{n+1}]\).
  we can apply the induction hypothesis to the
  tie path \([a_0, a_1, \ldots, a_n]\).
  Because \(a_n\) ties \(a_{n+1}\), and
  \(v\) is adjacent to both of them.
  By \cref{the: arrow direction lemma}
  \begin{itemize}
    \item
      Case 1, \(v \to \set{a_0, \ldots, a_n}\):
      then \(v \to a_{n+1}\),
      then \(v\) beats the whole tie path;
    \item
      Case 2, \(\set{a_0, \ldots, a_n} \to v\),
      then \(a_{n+1} \to v\),
      then \(v\) is beaten by the whole tie path.
  \end{itemize}
  Therefore this lemma still holds for tie path of length \(n + 1\).
\end{proof}

\cref{the: tie transimission} generalizes
\cref{the: arrow direction lemma},
however, it is still far from elegant,
because we still requires the vertex \(v\)
to be adjacent to the whole tie path.

\section{Tie Components}

We continue to explore the relationship between tie paths and
arrow direction transmitted by ties.
We to define the following structure with
the intuition of ``path'' and ``connected component''
in undirected graph\cite{west_introduction_2001}.

\begin{definition}
  In a digraph \(G\), a
  \keyword{tie component of vertex \(a\)}[tie component of a vertex]:
  \(C(a)\), all vertices \(v\)
  such that there exists a tie path between \(a\) and \(v\).
  \(a\) is called a \keyword{representitive} of \(C(a)\)
\end{definition}

\begin{figure}
  \centering
  \begin{subfigure}[b]{0.45\linewidth}
    \centering
    \tikz\graph[simple necklace layout, math nodes, node sep=1cm] {
      a -> {c, d, e};
      b -> {c, d, e};
      {c, d, e} -> f;
      {a, b} -> f;
      c -> d --[dashed] e --[dashed] c;
      a --[dashed] b;
    };
    \caption{distinct tie components are
    \(\set{a, b}, \set{c, d, e}, \set{f}\)}
    \label{fig: tie component example} %chktex 24
  \end{subfigure}
  \begin{subfigure}[b]{0.45\linewidth}
    \centering
    \tikz\graph[simple necklace layout, math nodes, node sep=1cm] {
      a;
      c;
      d --e --c;
      a --b;
      f;
    };
    \caption{connected components are
    \(\set{a, b}, \set{c, d, e}, \set{f}\)}
    \label{fig: connected component example} %chktex 24
  \end{subfigure}
  \caption{the tie components in (a) are the connected components in (b).}
  \label{fig: tie components and connected components}  % chktex 24
\end{figure}

In \cref{fig: tie components and connected components},
we first change every tie in
\cref{fig: tie component example} to an undirected edge
and then remove all the directed edges
to obtain \cref{fig: connected component example},
and tie components in \cref{fig: tie component example}
exactly correspond to connected components in
\cref{fig: connected component example}.
Therefore, just like a tie path is similar to a path
in an undirected graph,
a tie component is very similar to a connected component
(called a ``component'' in~\cite{west_introduction_2001})
in an undirected graph,
and we can bring all the properties and intuitions
of connected components to tie components.

We prove results about tie components in digraphs.
Notice that theses results work in all digraphs,
not just in quasi-transitive oriented graphs.

\begin{theorem}\label{the: tie path equivalence relation}
  For a digraph \(G\),
  we define a relation \(\mathrel{R}\) on \(V(G)\) by
  for vertices \(a, b \in V(G)\),
  \(a\mathrel{R}b\) if there exists a tie path between \(a, b\).
  \(\mathrel{R}\) is an equivalence relation.
\end{theorem}

\begin{proof}
  Reflexive: given a vertex \(a\),
  \([a]\) is a tie path of length 0.
  therefore there exists a tie path from \(a\) to itself.

  Symmetry: if there exists a tie path from \(a_0\) to \(a_n\):
  \([a_0, a_1, a_2, \ldots, a_n]\),
  then because tie do not have directions,
  \([a_n, a_{n-1}, \ldots, a_1, a_0]\) is a tie path from
  \(a_n\) to \(a_0\).

  Transitive: by \cref{the: tie path connection lemma}.
\end{proof}

\begin{corollary}\label{the: tie component equivalence class}
  For a digraph \(G\), tie component is an equivalence class
  on \(V(G)\).
\end{corollary}
\begin{proof}
  By \cref{the: tie path equivalence relation},
  and the definition of tie component.
\end{proof}

\begin{corollary}\label{the: tie components partition unique}
  For digraph \(G\), tie components form a unique
  partition of \(V(G)\).
\end{corollary}

\begin{proof}
  By \cref{the: tie component equivalence class},
  tie components form an equivalence class on \(V(G)\),
  and equivalence classes form a unique partition
  of the set \(V(G)\)~\cite{epp_discrete_2011}.
\end{proof}

\begin{corollary}\label{the: adjacent if not in component}
  In a digraph, if vertex \(v\) is not in tie component \(C(a)\),
  then \(v\) is adjacent to \(C(a)\).
\end{corollary}

\begin{proof}
  Because \(C(a)\) is an equivalence class on the relation
  \(\mathrel{R}\) such that
  \(a\mathrel{R}b\) if there exists a tie path between \(a\) and \(b\).
  Because \(v\) not in the equivalence class \(C(a)\),
  for every vertex \(a' \in C(a)\),
  \(v\mathrel{R}a'\) is false~\cite{epp_discrete_2011}.
  Therefore there is no tie path between \(v\) and \(a\),
  thus \(v\) does not tie \(a'\).

  Therefore, \(v\) is adjacent to every vertex in \(C(a)\).
\end{proof}

Our previous results that ties ``transmit'' arrow directions
in quasi-transitive oriented graph,
had us see that for a tie component \(C\)
and a vertex \(v \notin C\), \(C \to v\) or \(v \to C\).
See \cref{fig: transmitting arrow direction in tie component},
for a visualization.

\begin{figure}
  \centering
  \begin{subfigure}[b]{0.45\linewidth}
    \centering
    \tikz\graph[simple necklace layout, math nodes, node sep=1cm] {
      b <- a;
      / [label = left:\(C\), draw] // {a};
    };
    \caption{\(a\) beats \(b\).}
  \end{subfigure}
  \begin{subfigure}[b]{0.45\linewidth}
    \centering
    \tikz\graph[simple necklace layout, math nodes, node sep=1cm] {
      a --[dashed] c;
      {a, c} -> b;
      / [label = left:\(C\), draw] // {a, c};
    };
    \caption{direction transmitted from \(a\) to \(c\).}
  \end{subfigure}
  \begin{subfigure}[b]{0.45\linewidth}
    \centering
    \tikz\graph[simple necklace layout, math nodes, node sep=1cm] {
      a --[dashed] c --[dashed] d;
      ""; "";  % chktex 18
      a -> b;
      c -> b;
      d -> b;
      / [label = left:\(C\), draw] // {a, c, d};
    };
    \caption{transmitted from \(a\) to \(c\) to \(d\).}
  \end{subfigure}
  \begin{subfigure}[b]{0.45\linewidth}
    \centering
    \tikz\graph[simple necklace layout, math nodes, node sep=1cm] {
      a --[dashed] c --[dashed] d;
      c --[dashed] e;
      "";  % chktex 18
      a -> b;
      c -> b;
      d -> b;
      e -> b;
      / [label = left:\(C\), draw] // {a, c, d, e};
    };
    \caption{transmitted from \(a\) to \(c\) to \(e\).}
  \end{subfigure}
  \caption{the arrow direction to \(b\) was transmitted in tie component \(C\).}
  \label{fig: transmitting arrow direction in tie component}  % chktex 24
\end{figure}

We then formalize and generalize this idea into two theorems:

\begin{lemma}\label{the: component and a single point}
  In a quasi-transitive oriented graph,
  for any tie component \(C(a)\) and
  any vertex \(v\) such that \(v \notin C(a)\),
  then either \(C(a) \to v\) or \(v \to C(a)\).
\end{lemma}

\begin{proof}
  By \cref{the: adjacent if not in component},
  we know that \(v\) is adjacent to \(C(a)\).
  Let \(x, y\) be two vertices in \(C(a)\).
  There exists a tie path between them,
  so both \(x\) and \(y\) beats \(v\)
  or \(v\) beats both \(x, y\).
  Since this is true for all pairs of vertices on \(C(a)\),
  we get \(v \to C(a)\) or \(C(a) \to v\).
\end{proof}

\begin{corollary}\label{the: vertex force component beating}
  In a quasi-transitive oriented graph,
  for any tie component \(C(a)\),
  If there exists one vertex \(v \in C(a)\)
  that beats (be beaten) a vertex \(v' \notin C(a)\).
  Then \(C(a) \to v'\) (\(v' \to C(a)\)).
  See \cref{fig: vertex force component beating}
\end{corollary}

\begin{figure}
  \centering
  \begin{subfigure}[b]{0.45\linewidth}
    \centering
    \tikz\graph[simple necklace layout, math nodes, node sep=1cm] {
      a --[dashed] b --[dashed] "\ldots" --[dashed] c --[dashed] d;  % chktex 18
      c --[dashed] e;
      ""; % chktex 18
      b -> f;
      ""; ""; % chktex 18
      / [label = 45:\(C(a)\), draw] // {a, b, "\ldots", c, d, e};  % chktex 18
    };
    \caption{a single vertex in \(C(a)\) beats \(f\).}
  \end{subfigure}
  \begin{subfigure}[b]{0.45\linewidth}
    \centering
    \tikz\graph[simple necklace layout, math nodes, node sep=1cm] {
      a --[dashed] b --[dashed] "\ldots" --[dashed] c --[dashed] d;  % chktex 18
      c --[dashed] e;
      ""; % chktex 18
      {a, b, "\ldots", c, d, e} -> f;  % chktex 18
      ""; ""; % chktex 18
      / [label = 45:\(C(a)\), draw] // {a, b, "\ldots", c, d, e};  % chktex 18
    };
    \caption{component \(C(a)\) has to beat \(f\).}
  \end{subfigure}
  \caption{a vertex \(b\) beats \(f\)
    will force the component \(C(a)\) to beats \(f\).}
  \label{fig: vertex force component beating}  % chktex 24
\end{figure}

\begin{proof}
  Because \(v' \notin C(a)\),
  then either \(v' \to C(a)\) or \(C(a) \to v'\)
  by \cref{the: component and a single point}.

  Case 1, \(v \to v'\): because \(v \in C(a)\),
  then \(v'\) cannot beat all of \(C(a)\).
  Therefore \(C(a) \to v'\).

  Case 2, \(v' \to v\): because \(v \in C(a)\),
  then \(v'\) cannot be beaten by all of \(C(a)\).
  Therefore \(v' \to C(a)\).
\end{proof}

To further generalize \cref{the: component and a single point},
we prove the following theorem:

\begin{theorem}\label{the: tie component beats tie component}
  For any two distinct tie components \(C(a)\) and \(C(b)\)
  in a quasi-transitive oriented graph,
  \(C(a) \to C(b)\) or \(C(b) \to C(a)\).
\end{theorem}

\begin{proof}
  Take any vertex \(a'\) in \(C(a)\).
  By \cref{the: tie components partition unique}, \(a'\) is not in \(C(b)\).
  Then because of \cref{the: component and a single point},
  \(a' \to C(b)\) or \(C(b) \to a'\).

  \begin{itemize}
    \item
      Case 1, \(a' \to C(b)\): take any element \(b' \in C(b)\).
      By \cref{the: vertex force component beating}, \(C(a) \to b'\).
      Therefore \(C(a)\) beats every vertex in \(C(b)\),
      then \(C(a) \to C(b)\).
    \item
      Case 2, \(C(b) \to a'\):
      by the same reasoning, \(C(b) \to C(a)\).
  \end{itemize}
\end{proof}

After several pages of theorems,
we can finally take a pause
and understand what we are saying here.
We combine \cref{the: tie components partition unique} and
\cref{the: tie component beats tie component},
and try to understand it visually.

\begin{figure}
  \centering
  \begin{subfigure}[b]{0.45\linewidth}
    \centering
    \tikz\graph[simple necklace layout, math nodes, node sep=1cm] {
      a; b;
      a -> {c, d, e};
      b -> {c, d, e};
      {c, d, e} -> f;
      {a, b} -> f;
      j -> {c, d, e};
      {a, b} -> j;
      f -> j;
      c -> d --[dashed] e --[dashed] c;
      a --[dashed] b;
    };
    \caption{a quasi-transitive oriented graph.}
  \end{subfigure}
  \begin{subfigure}[b]{0.45\linewidth}
    \centering
    \tikz\graph[simple necklace layout, math nodes, node sep=1cm] {
      a; b; c; d; e; f; j;
      d --[dashed] e --[dashed] c;
      a --[dashed] b;
      / [label = left:\(A\), draw, circle] // {a, b};
      / [label = left:\(B\), draw] // {d, e, c};
      / [label = right:\(C\), draw, circle] // {f};
      / [label = right:\(D\), draw, circle] // {j};
    };
    \caption{find its tie components.}
  \end{subfigure}
  \begin{subfigure}[b]{0.45\linewidth}
    \centering
    \tikz\graph[simple necklace layout, math nodes, node sep=1cm] {
      A [draw, circle, minimum size=1.5cm] -> B [draw, circle, minimum size=1.75cm]
      -> C [draw, circle, minimum size=1cm] -> D [draw, circle, minimum size=1cm];
      A -> D -> B;  % chktex 13
      A -> C;  % chktex 13
    };
    \caption{components beat each other.}
  \end{subfigure}
  \begin{subfigure}[b]{0.45\linewidth}
    \centering
    \tikz\graph[simple necklace layout, math nodes, node sep=1cm] {
      A' -> B' -> C' -> D';
      A' -> D' -> B';
      A' -> C';
    };
    \caption{components are just like vertices.}
  \end{subfigure}
  \caption{condense the tie components and get a tournament.}
  \label{fig: tie components condensation}  % chktex 24
\end{figure}

See \cref{fig: tie components condensation}.
For every quasi-transitive digraph,
we can split it into tie components,
and every tie component either beats another tie component
or is beaten by another tie component.
Then, tie components behave like vertices.
Because there are no ties between tie components,
The graph formed by tie components is a tournament,
which is the ``most well understood class of
directed graphs''~\cite{bang-jensen_generalizations_1998}.

\section{Graph Condensations}

\begin{definition}
  A \keyword{condensation} is a function \(f: G \to H\),
  where \(G\) and \(H\) are oriented graphs,
  and \(f\) maps \(V(G)\) to \(V(H)\) surjectively,
  such that
  \begin{itemize}
    \item for any two vertices \(a, b \in V(G)\),
      if \(f(a) \to f(b)\) in \(H\)
      then \(a \to b\) in \(G\).
    \item for any two vertices \(a, b \in V(G)\),
      if \(f(a)\) ties \(f(b)\) in \(H\)
      then \(a\) ties \(b\) in \(G\).
  \end{itemize}
  We call \(G\) the \keyword{uncondensed graph},
  and \(H\) the \keyword{condensed graph}.
\end{definition}

Two examples of condensations are identity condensation
and trivial condensation.

Identity condensation \(i\) maps an oriented graph \(G\) to itself,
such that \(i\) maps every vertex to itself,
and \(f(a) \to f(b)\) if and only if \(a \to b\).
The identity condensation preserves all the edges
and ties, and do not change the graph at all.
We can see that this map is a condensation by definition.

Another example is trivial condensation \(t\),
which maps any oriented graph \(G\) into a single point.
Because for any two vertices \(a, b \in G\),
\(t(a) = t(b)\), therefore by definition,
\(t(a)\) does not tie or beat \(t(b)\).
Therefore, this mapping also satisfies
the definition of condensation.

\begin{definition}
  Given oriented graph \(G\) as the uncondensed graph of
  condensation \(f\),
  A \keyword{component} of vertex \(a \in V(G)\)
  is the pre-image of \(f(a)\),
  in other words, the component of \(a\) is
  \(\set{x \in V(G) \mid f(x) = f(a)}\).
\end{definition}

A component of vertex \(a\) is a set of vertices
that are condensed into the same vertex as \(a\).
For example, the component of vertex \(a\)
in oriented graph \(G\) with identity condensation
is a set that only contains \(a\) itself,
and the component of a vertex \(a\) in oriented graph \(G\)
with the trivial condensation is set of all the vertices,
because all vertices in \(G\) are mapped onto a
single vertex in \(H\),
therefore all the vertices in \(G\) are
condensed into the same vertex as \(a\).

\begin{figure}
  \centering
  \begin{subfigure}[b]{0.45\linewidth}
    \centering
    \tikz\graph[simple necklace layout, math nodes, node sep=1cm] {
      a; b;
      a --[dashed] {c, d, e};
      b --[dashed] {c, d, e};
      {c, d, e} -> f;
      {a, b} -> f;
      j -> {c, d, e};
      {a, b} -> j;
      f -> j;
      c -> d -> e --[dashed] c;
      a -> b;
    };
    \caption{an oriented graph \(G\).}
  \end{subfigure}
  \begin{subfigure}[b]{0.45\linewidth}
    \centering
    \tikz\graph[simple necklace layout, math nodes, node sep=1cm] {
      a; b; c; d; e; f; j;
      d -> e --[dashed] c;
      a -> b;
      / [label = left:\(A\), draw, circle] // {a, b};
      / [label = left:\(B\), draw] // {d, e, c};
      / [label = right:\(C\), draw, circle] // {f};
      / [label = right:\(D\), draw, circle] // {j};
    };
    \caption{components of \(G\) in this condensation.}
  \end{subfigure}
  \begin{subfigure}[b]{0.45\linewidth}
    \centering
    \tikz\graph[simple necklace layout, math nodes, node sep=1cm] {
      A [draw, circle, minimum size=1.5cm] --[dashed]
      B [draw, circle, minimum size=1.75cm] ->
      C [draw, circle, minimum size=1cm] ->
      D [draw, circle, minimum size=1cm];
      A -> D -> B
    };
    \caption{find the relationship between components.}
  \end{subfigure}
  \begin{subfigure}[b]{0.45\linewidth}
    \centering
    \tikz\graph[simple necklace layout, math nodes, node sep=1cm] {
      A' --[dashed] B' -> C' -> D';
      A' -> D' -> B';
    };
    \caption{condensed graph \(H\).}
  \end{subfigure}
  \caption{an example of graph condensation.}
  \label{fig: condensation example}  % chktex 24
\end{figure}

In \cref{fig: condensation example},
we show a general example of condensation.
In this example,
the distinct components in \(G\) are \(A\), \(B\), \(C\) and \(D\),
and they are condensed into vertices
\(A'\), \(B'\), \(C'\) and \(D'\) respectively.
The component of \(a\) is \(A\),
the component of \(d\) is \(B\),
and the component of \(c\) is also \(B\),
because both \(c\) and \(d\) are mapped into vertex
\(B'\) in oriented graph \(H\).

We proof several theorems involving condensation and components
to help us with our goal of identifying kings in
quasi-transitive oriented graphs.

\begin{theorem}\label{the: condensation component partition}
  Given any oriented graph \(G\) and condensation \(f\),
  distinct components in \(G\) partition \(V(G)\).
\end{theorem}

\begin{proof}
  Prove distinct component disjoint:
  Assume there exists two vertex \(a\) and \(b\),
  such that the component of \(a\) is \(A\);
  component of \(b\) is \(B\),
  and \(A \neq B\); \(A \cap B \neq \emptyset \).
  Therefore, there exists \(v \in A \cap B\).
  By definition of components,
  \(f(v) = f(a)\), and \(f(v) = f(b)\),
  therefore \(f(b) = f(a)\),
  and \(B
  = \set{x \in V(G) \mid f(x) = f(b)}
  = \set{x \in V(G) \mid f(x) = f(a)}
  = A\).
   \(A = B\), contradiction.

  Prove every vertex is in a component:
  for every vertex \(v \in V(G)\),
  \(v\) is in the component of \(v\) because \(f(v) = f(v)\).
\end{proof}

\begin{lemma}\label{the: vertex force image beating}
  Given any oriented graph \(G\) and condensation \(f\),
  for any two distinct components \(A\) and \(B\),
  \begin{itemize}
    \item
      for all \(a \in A, b \in B\), if \(a\) ties \(b\),
      then \(f(a)\) ties \(f(b)\).
    \item
      for all \(a \in A, b \in B\), if \(a \to b\),
      then \(f(a) \to f(b)\)
  \end{itemize}
\end{lemma}

\begin{proof}
  Because \(A\) and \(B\) are distinct components,
  then by \cref{the: condensation component partition},
  \(A\) and \(B\) are disjoint,
  therefore for any vertex \(a \in A\),
  and any vertex \(b \in B\), \(f(a) \neq f(b)\).

  Given vertex \(a \in A\) ties vertex \(b \in B\).
  Then \(f(a)\) cannot beat \(f(b)\),
  otherwise \(a\) should beat \(b\), by definition of condensation.
  Also, \(f(b)\) cannot beat \(f(a)\),
  otherwise \(b\) should beat \(a\),
  therefore \(f(a)\) ties \(f(b)\).

  Given vertex \(a \in A\) beats vertex \(b \in B\).
  Then \(f(a)\) cannot tie \(f(b)\),
  otherwise \(a\) should tie \(b\)
  Also \(f(b)\) cannot beat \(f(a)\),
  otherwise \(b\) should beat \(a\).
  Therefore \(f(a) \to f(b)\).
\end{proof}

\begin{theorem}\label{the: vertex force beating relation in condensation}
  Given any oriented graph \(G\) and condensation \(f\),
  for any two distinct components \(A\) and \(B\),
  \begin{itemize}
    \item
      if any vertex \(a \in A\) ties any vertex \(b \in B\),
      then \(A\) ties \(B\).
    \item
      if any vertex \(a \in A\) beats any vertex \(b \in B\),
      then \(A \to B\).
  \end{itemize}
\end{theorem}

\begin{proof}
  Because \(A\) and \(B\) are distinct components,
  then by \cref{the: condensation component partition},
  \(A\) and \(B\) are disjoint,
  therefore for any vertex \(a \in A\),
  and any vertex \(b \in B\), \(f(a) \neq f(b)\).

  Given vertex \(a \in A\) ties vertex \(b \in B\),
  by \cref{the: vertex force image beating},
  \(f(a)\) ties \(f(b)\).
  Thus, for every vertex \(a' \in A\),
  and every vertex \(b' \in B\),
  because \(f(a') = f(a)\) and \(f(b') = f(b)\),
  and \(f(a)\) ties \(f(b)\), therefore \(f(a')\) ties \(f(b')\).
  By the definition of condensation, \(a'\) ties \(b'\).
  Therefore, \(A\) ties \(B\).

  Given vertex \(a \in A\) beats vertex \(b \in B\).
  by \cref{the: vertex force image beating},
  \(f(a) \to f(b)\).
  Thus for every vertex \(a' \in A\),
  and every vertex \(b' \in B\),
  because \(f(a') = f(a)\) and \(f(b') = f(b)\),
  and \(f(a) \to f(b)\), therefore \(f(a') \to f(b')\).
  By the definition of condensation, \(a' \to b'\).
  Therefore, \(A\) ties \(B\).
\end{proof}

\begin{corollary}\label{the: components are vertex in condensation}
  Given any oriented graph \(G\) and condensation \(f\),
  for any two distinct components \(A\) and \(B\),
  either \(A \to B\), or \(B \to A\), or \(A\) ties \(B\).
\end{corollary}

\begin{proof}
  Because components are not empty
  (component of \(v\) always contains \(v\) itself),
  we can take \(a \in A\), and \(b \in B\).
  Because \(G\) is an oriented graph,
  then either \(a \to b\), \(b \to a\), or \(a\) ties \(b\).

  According to \cref{the: vertex force beating relation in condensation},
  if \(a \to b\), then \(A \to B\);
  if \(b \to a\), then \(B \to A\);
  if \(a\) ties \(b\), then \(A\) ties \(B\).
\end{proof}

In \cref{fig: behaves like a vertex example},
we show an example of a set of vertices \(\set{b, c, d}\)
that behaves like the vertex \(v\).
We then try to visualize \cref{the: components are vertex in condensation},
with the help of \cref{fig: condensation example},
we find out that graph condensation just takes the idea
in \cref{fig: behaves like a vertex example}
to the next level: all the components behave like vertices.
A condensation looks for those sets of vertices
that behave like a single vertex (components),
and then condenses them into a single vertex.

\begin{figure}
  \centering
  \begin{subfigure}[b]{0.4\linewidth}
    \centering
    \tikz\graph[layered layout, math nodes, level sep=1cm] {
      a -> {b, c, d};
      {b, c, d} -> f;
      {b, c, d} -> e;
      / [draw] // {b, c, d};
    };
    \caption{every vertex either beats \(\set{b, c, d}\)  %chktex 44
      or is beaten by them.}
    \label{fig: behaves like a vertex example: uncondensed} % chktex 24
  \end{subfigure}
  \begin{subfigure}[b]{0.4\linewidth}
    \centering
    \tikz\graph[layered layout, math nodes, level sep=1cm] {
      a -> v;
      v -> f;
      v -> e;
    };
    \caption{vertices \(\set{b, c, d}\) in %chktex 44
      \cref{fig: behaves like a vertex example: uncondensed}
     behaves exactly like \(v\) in this graph.}
  \end{subfigure}
  \caption{\(\set{b, c, d}\) behaves like a vertex.}  %chktex 44
  \label{fig: behaves like a vertex example}  % chktex 24
\end{figure}

Notice that all oriented graph have the
identity condensation and trivial condensation,
but some oriented graphs may not have other condensations
defined on them.

\begin{figure}
  \centering
  \tikz\graph[simple necklace layout, math nodes, node sep=1.75cm] {
      a -> b -> c -> a;
  };
  \caption{an oriented graph with only
    identity condensation and trivial condensation.}
  \label{fig: no condensation example} % chktex 24
\end{figure}

For example, the graph in \cref{fig: no condensation example}
cannot have any condensation defined on it
except for identity condensation and trivial condensation,
because we cannot find any component that is not the whole
graph or a single vertex.
For example, let's look at set \(\set{a, c}\).
Vertex \(b\) beats a vertex in the set (\(b\) beats \(a\)),
and is beaten by a vertex in the set (\(c\) is beaten by \(b\)).
Therefore the set \(\set{a, c}\) is not a component.

Condensation is a relatively strong transformation,
because it preserves many properties of the graph.
We will prove some theorems that will be useful
in later sections, when looking for kings

\begin{lemma}\label{the: shortest path different components lemma}
  Let \(f: G \to H\) be a condensation of oriented graph \(G\)
  and let \(a_0, a_n\) be two vertices in \(G\) from different
  components. If \(P: a_0 \to a_1 \to \cdots \to a_n\) is a shortest
  path from \(a_0\) to \(a_n\), then vertices
  \(a_0, a_1, \ldots, a_n\) are all in different components.
\end{lemma}

\begin{figure}
\centering
  \begin{subfigure}[b]{0.45\linewidth}
  \centering
    \tikz\graph[simple necklace layout, math nodes, node sep=1cm] {
      dots1[as=\ldots] -> a_{p-1} -> a_p
      -> dots2[as=\ldots] -> a_q -> dots3[as=\ldots];
      / [label=left:\(C\), draw] // {a_p, a_q};
    };
    \caption{\(a_p\) and \(a_q\) are in the same component.}
  \end{subfigure}
  \begin{subfigure}[b]{0.45\linewidth}
    \centering
      \tikz\graph[simple necklace layout, math nodes, node sep=1cm] {
        dots1[as=\ldots] -> a_{p-1} -> a_p
        -> dots2[as=\ldots] -> a_q -> dots3[as=\ldots];
        a_{p-1} -> a_q;
        / [label=left:\(C\), draw] // {a_p, a_q};
      };
      \caption{by \cref{the: vertex force beating relation in condensation},
        \(a_{p-1} \to a_q\).}
    \end{subfigure}

  \caption{the path took a ``detour'' at \(a_{p-1}\).}  %chktex 44
  \label{fig: same component the not the shortest path}  % chktex 24
\end{figure}

\begin{proof}
  Suppose not. Take the the smallest \(p\) such that
  there exists \(a_p\) and \(a_q (p < q \leq n)\)
  are in the same component \(C\).

  Since \(a_p\) is the first vertex in the \(P\)
  that belongs to the same component as another vertex in \(P\).
  \(a_{p-1}\) cannot be in the same component as \(a_p\),
  then \(a_{p-1} \notin C\).
  Because \(a_{p-1} \to a_p\)
  and \cref{the: vertex force beating relation in condensation},
  \(a_{p-1} \to C\).
  Thus \(a_{p-1} \to a_q\), and path
  \(a_0 \to a_1 \to \cdots
  \to a_{p-2} \to a_{p-1} \to a_q \to a_{q+1} \to \cdots
  \to a_{n-1} \to a_n\) exists and shorter than \(P\).
  See \cref{fig: same component the not the shortest path}.
  Contradiction.
\end{proof}

\begin{theorem}\label{the: condensation preserves shortest path}
  Given a condensation \(f: G \to H\),
  for any \(a_0, a_n \in V(G)\),
  such that \(a_0\) and \(a_n\) are in two distinct components.
  The shortest path from \(a_0\) to \(a_n\) is
  \(a_0 \to a_1 \to \cdots \to a_{n-1} \to a_n\) in \(G\)
  if and only if the shortest path from \(f(a_0)\) to \(f(a_n)\) is
  \(f(a_0) \to f(a_1) \to \cdots \to f(a_{n-1}) \to f(a_n)\)
  in \(H\).
\end{theorem}

\begin{proof}
  Prove \(\Rightarrow \):
  assume \(P: f(a_0) \to f(a_1) \to \cdots \to f(a_{n-1}) \to f(a_n)\)
  is not the shortest path,
  then there exists another path \(P'\)
  from \(f(a_0)\) to \(f(a_n)\): \(f(a_0)
    \to f(b_1) \to f(b_2) \to \cdots \to f(b_{m-1}) \to f(b_m)
    \to f(a_n)\).
  Because \(a_0\) and \(a_n\) are in distinct components,
  then the path \(P'\) does not have length 0.
  If \(P'\) is shorter than \(P\), the path
  \(a_0 \to b_1 \to b_2 \to \cdots \to b_{m-1} \to b_m \to a_n\)
  exists (by definition of condensation) and will be shorter than
  \(a_0 \to a_1 \to \cdots \to a_{n-1} \to a_n\).
  Since \(a_0 \to a_1 \to \cdots \to a_{n-1} \to a_n\)
  is the shortest path from \(a_0\) to \(a_n\),
  contradiction.

  Prove \(\Leftarrow \):
  assume the shortest path is
  \(a_0 \to b_1 \to b_2 \to \cdots \to b_m \to a_n\).
  Because of \cref{the: shortest path different components lemma},
  \(\set{a_0, b_1, b_2, \ldots, b_m, a_n}\)
  are all in different components.
  Then by \cref{the: vertex force image beating},
  \(f(a_0) \to f(b_1) \to f(b_2) \to \cdots \to f(b_m) \to f(a_n)\)
  exists and will be shorter than
  \(f(a_0) \to f(a_1) \to \cdots \to f(a_{n-1}) \to f(a_n)\).
  Because \(f(a_0) \to f(a_1) \to \cdots \to f(a_{n-1}) \to f(a_n)\)
  is the shortest path, contradiction.
\end{proof}

\begin{corollary}\label{the: condensation preserves beating in distinct components}
  Given a condensation \(f: G \to H\),
  for all \(a_0, a_n \in G\),
  such that \(a_0\) and \(a_n\) are in two distinct components,
  \(a_0\) beats \(a_n\) by \(n\) steps in \(G\) if and only if
  \(f(a_0)\) beats \(f(a_n)\) by \(n\) steps in \(H\)
\end{corollary}

\begin{proof}
  Result of \cref{the: condensation preserves shortest path}.
\end{proof}

\begin{corollary}\label{the: condensation preserves king}
  Given a condensation \(f: G \to H\),
  vertex \(k\) is a king in \(G\) if and only if
  \begin{itemize}
    \item \(k\) is a king in the
      induced subgraph of the component of \(k\).
    \item \(f(k)\) is a king in \(H\).
  \end{itemize}
\end{corollary}

\begin{proof}
  Denote component of \(k\) as \(C\).
  Denote the induced subgraph of \(C\) as \(G_c\).

  Prove \(\Rightarrow \):
  If \(k\) is a king, then it beats every vertex in \(V(G)\)
  by 1 or 2 steps.
  First, if \(k\) beats every vertex outside of \(C\) by
  1 or 2 steps,
  because of \cref{the: condensation preserves beating in distinct components},
  then \(f(k)\) beats every vertex in \(H\) in 1 or 2 steps,
  therefore \(f(k)\) is a king in \(H\).
  Then if \(k\) beats every vertex \(v \in C\)
  in 1 or 2 steps in \(G\), we split into 2 cases:
  \begin{itemize}
    \item if \(k\) beats \(v\) by 1 step, then
      \(k\) beats \(v\) by 1 step in \(G_c\).
    \item if \(k\) beats \(v\) by 2 steps, then
      there exists \(a\) such that \(k \to a \to v\).
      \(a \in C\), because of
      \cref{the: vertex force beating relation in condensation}.
      Therefore, \(k\) beats \(v\) by 2 steps in \(G_c\).
  \end{itemize}
  Then \(k\) is a king in \(G_c\),
  and \(f(k)\) is a king in \(H\).

  Prove \(\Leftarrow \):
  If \(f(k)\) is a king in \(H\),
  then \(f(k)\) beats every other vertex \(f(v)\) in \(H\),
  then \(k\) beats every other vertex that is not in \(C\)
  by 1 or 2 steps, by \cref{the: condensation preserves beating in distinct components}.
  If \(k\) is a king in \(G_c\),
  then \(k\) beats every vertex in \(C\) by 1 or 2 steps.
  Therefore, \(k\) beats every vertex in \(V(G)\) by 1 or 2 steps,
  then \(k\) is a king in \(G\).
\end{proof}

Although condensation is a very strong transformation,
some information in the graph does get lost via this transformation.
For example, the beating relationships between all the vertices
in the same components are lost.


\begin{figure}
\centering
  \begin{subfigure}[b]{0.45\linewidth}
  \centering
    \tikz\graph[simple necklace layout, math nodes, node sep=1cm] {
      a_1; a_2;
      a_1 -> {b_1, b_2};
      a_2 -> {b_1, b_2};
      {b_1, b_2} -> c_1;
      {b_1, b_2} -> d_1;
      {a_1, a_2} -> c_1;
      c_1 -> d_1;
      b_1 -> b_2;
      / [label=left:\(a\), draw] // {a_1, a_2};
      / [label=left:\(b\), draw] // {b_1, b_2};
      / [label=right:\(c\), draw] // {c_1};
      / [label=right:\(d\), draw] // {d_1};
    };
    \label{fig: condensation lost information: uncondensed}  % chktex 24
    \caption{the uncondensed graph.}
  \end{subfigure}
  \begin{subfigure}[b]{0.45\linewidth}
  \centering
    \tikz\graph[simple necklace layout, math nodes, node sep=1cm] {
      a -> b -> {c, d};
      a -> c -> d;
    };
    \caption{the condensed graph.}
    \label{fig: condensation lost information: condensed}  % chktex 24
  \end{subfigure}
  \caption{information lost during condensation.}
  \label{fig: condensation lost information}  % chktex 24
\end{figure}

In \cref{fig: condensation lost information},
we show an example of condensation,
where the vertices \(a_1, a_2\) are condensed into \(a\),
\(b_1, b_2\) are condensed into \(b\),
\(c_1\) is condensed into \(c\),
and \(d_1\) is condensed into \(d\).
If we just look at the condensed graph,
we noticed that we cannot recreate the beating relationship
of \(a_1\) and \(a_2\),
therefore the information about the beating relationship
is lost.
This fact is true for all the vertices in the same component.
Another example in this graph is that
we cannot know the beating relationship between \(b_1\) and \(b_2\)
just by the uncondensed graph.

Notice the beating relationship between components
are not lost.
For example, if we want to know the beating relationship
between \(b_1\) and \(a_2\),
we first observed that \(a \to b\) in the condensed graph,
then by definition of a condensation,
\(a_2\) have to beat \(b_1\) in the uncondensed graph.
Another example is that we can know that \(a_1\) ties \(d_1\)
because \(a\) ties \(d\) in the condensed graph.

Therefore, a ``efficient condensation'' should keep as many
vertices as possible.
One of the most ``efficient'' condensation is the
identity condensation, because it does not lose any information
about this graph.
However, identity condensation also do not mutate the graph
at all, therefore it is not very practical.

\begin{definition}
  Given a set of condensation \(F = \set{f_0, f_1, \ldots, f_n}\)
  where \(f_k: G_k \to H_k\)
  and all the \(V(G_k)\) are of the same size for \(0 \leq k \leq n\),
  an \keyword{efficient condensation} \(f: G \to H\) in \(F\)
  is a condensation such that
  \(H\) has largest vertex set in
  \(\set{H_0, H_1, H_2, \ldots H_n}\)
\end{definition}

Notice in this definition,
all the \(G_k\)'s where \(0 \leq k \leq n\) are not
necessarily distinct.
One example of efficient condensation is
given a graph \(G\) and all the condensation defined on \(G\),
that is  \(\set{f_0, f_1, \ldots, f_n}\) where \(f_k: G \to H_k\).
Then the identity condensation,
is an efficient condensation in this set,
since it keeps all the vertices.

\section{Tie Component Condensations}

In the previous section we talked about the general property
of graph condensations.
In this section, we will focus on a specific type of condensation
mentioned in previous sections and
\cref{fig: tie components condensation}.

\begin{definition}
  A \keyword{tie component condensation} is a
  family of function \(f: Q \to T\),
  where \(Q\) is a quasi-transitive oriented graph,
  \(T\) is a directed graph.
  And \(f\) maps \(V(Q)\) to \(V(T)\) surjectively,
  maps \(E(Q)\) to \(E(T)\) surjectively,
  such that:
  \begin{itemize}
    \item if \(a, b \in V(Q)\) in the same tie component,
      then \(f(a) = f(b)\).
    \item if \(a, b \in V(Q)\) in 2 distinct tie component
      \(A, B\) respectively,
      then
      \begin{itemize}
        \item \(f(a) \to f(b)\) if \(A \to B\).
        \item \(f(b) \to f(a)\) if \(B \to A\).
      \end{itemize}
  \end{itemize}
\end{definition}

\begin{corollary}\label{the: tie condensation are condensation}
  All tie component condensations are graph condensation,
  where the components of the uncondensed graphs
  are the tie component of the uncondensed graphs.
\end{corollary}

\begin{corollary}\label{the: tie condensation results in tournament}
  for any tie component condensation \(f: Q \to T\),
  \(T\) will always be a tournament.
\end{corollary}
\begin{proof}
  Because of \cref{the: tie component beats tie component},
  for any two distinct vertices \(f(a), f(b) \in T\),
  either \(f(a) \to f(b)\) or \(f(b) \to f(a)\).
  Therefore, \(T\) is a tournament.
\end{proof}

\cref{fig: tie components condensation} is a great example of
tie component condensation.
In this condensation, the function \(g\) maps \(a, b\) to \(A'\),
maps \(c, d, e\) to \(B'\), and maps \(f\) to \(C'\),
and \(j\) to \(D'\).

It is also helpful to note that the tie component \(C\) and \(D\)
are trivial tie components,
since they only contain vertex \(f\) and \(j\) respectively.

\begin{corollary}
  Given a quasi-transitive digraph \(Q\),
  there only exists a unique tie component condensation
  \(f: Q \to T\)
\end{corollary}
\begin{proof}
  Tie component condensation exists by definition,
  tie component condensation unique
  by \cref{the: tie components partition unique}.
\end{proof}

\begin{definition}
  Given a quasi-transitive digraph \(Q\),
  we call the tie component condensation \(f: Q \to T\)
  the tie component condensation on \(Q\).
\end{definition}

\begin{definition}
  For a given quasi-transitive oriented graph \(Q\),
  we call the result of tie component condensation of \(Q\)
  the \keyword{underlying tournament} of \(Q\).
\end{definition}

It is very surprising that every
quasi-transitive oriented graph
can always be condensed into a tournament,
which is one of the most understood family of digraphs.
But just to say that a condensation exists is not surprising enough,
since for every graph, there exists a trivial condensation,
which also always result in a tournament
(a single vertex is, by definition, a tournament).
Therefore, we need to investigate the efficiency of the tie
component condensation.

\begin{definition}
  For two orientated graphs \(G, H\), such that
  \begin{itemize}
    \item \(V(G) = V(H)\),
    \item \(a\) ties \(b\) in \(H\) if and only if
      \(a\) ties \(b\) in \(G\),
  \end{itemize}
  then we say \(G, H\) has the same \keyword{tie structure}.
\end{definition}

Graph with same tie structure means
that if we draw out all the ties from two graph,
the graph formed by the ties are the same.
In other words, these two graphs can only differ
by the orientation of some of their edges.

\begin{figure}
  \centering
  \begin{subfigure}[b]{0.3\linewidth}
    \centering
    \tikz\graph[simple necklace layout, math nodes, node sep=1cm] {
      a -> b -> c;
      b -> d -> e;
      e -> c -> a;
    };
    \caption{graph \(G\)}
    \label{fig: same tie structure example: G}  % chktex 24
  \end{subfigure}
  \begin{subfigure}[b]{0.3\linewidth}
    \centering
    \tikz\graph[simple necklace layout, math nodes, node sep=1cm] {
      a -> b -> c;
      b -> d -> e;
      e <- c <- a;
    };
    \caption{graph \(H\)}
    \label{fig: same tie structure example: H}  % chktex 24
  \end{subfigure}
  \begin{subfigure}[b]{0.3\linewidth}
    \centering
    \tikz\graph[simple necklace layout, math nodes, node sep=1cm] {
      a; b; c; d; e;
      b --[dashed] e --[dashed] a;
      c --[dashed] d;
    };
    \caption{``tie structures'' of \(G, H\)}
    \label{fig: same tie structure example: tie}  % chktex 24
  \end{subfigure}
  \caption{\(G\) and \(H\) has the same tie}
  \label{fig: same tie structure example}  % chktex 24
\end{figure}

See \cref{fig: same tie structure example},
in \cref{fig: same tie structure example: G} and
\cref{fig: same tie structure example: H},
we show two graph \(G\) and \(H\) with the same tie structure,
In \cref{fig: same tie structure example: tie},
we draw out all the ties from the previous two graphs,
and discover that they are the same.
the only differences between these two graphs are that
\(c \to a\), \(e \to c\) in \(G\),
however \(a \to c\), \(c \to e\) in \(H\),
that is, the only differences between \(G\) and \(H\) are
the orientations of these two edges.

\begin{corollary}\label{the: same tie structure same tie path}
  For \(G\) and \(H\) with the same tie structure,
  if there exists a tie path between \(a, b\) in \(G\),
  then there exists a tie path between \(a, b\) in \(H\).
\end{corollary}

\begin{proof}
  since given any two vertices \(p, q \in G\),
  if \(p\) ties \(q\) in \(G\)
  then \(p\) ties \(q\) in \(H\).
  Therefore if there exists a tie path
  \([a, a_0, a_1, \ldots, a_n, b]\) in \(G\),
  then the same tie path exists in \(H\)
\end{proof}

\begin{theorem}\label{the: tie condensation effcient}
  Given a quasi-transitive graph \(Q\)
  and its tie component condensation \(f\),
  consider the set of all the condensation \(f_k: G_k \to T_k\),
  where \(G_k\) has the same tie structure as \(Q\)
  and \(T_k\) is a tournament:
  \(F = \{f_0, f_1, \ldots, f_{n-1}, f_n\} \).
  \(f\) is an efficient condensation in \(F\).
\end{theorem}

\begin{figure}
  \centering
  \tikz\graph[simple necklace layout, math nodes, node sep=1cm] {
    a --[dashed] dot_1[as=\ldots]
    --[dashed] a_p --[dashed] a_{p+1} --[dashed] dot_2[as=\ldots]
    a_m --[dashed] dots_3[as=\ldots] --[dashed] a_{n-1}
    --[dashed] a_n --[dashed] b;
    / [label=left: component of \(a\), draw] //
    {a, dot_1, a_p};
    / [label=left: another component, draw] //
    {a_{p+1}, dot_2};
  };
  \caption{vertices \(a_p\) and \(a_{p+1}\) crosses components}
  \label{fig: efficient proof: cross component}  % chktex 24
\end{figure}

\begin{proof}
  Because \(G_k\) are has the same tie structure as \(Q\),
  therefore all the \(G_k\) has the same number of vertices.

  if there exists another condensation \(f': G' \to T'\)
  such that \(T'\) has more vertices than
  the underlying tournament \(T\) of \(Q\).
  Then there exists
  two vertices \(a, b\) in the same component in \(Q\),
  and in different components in \(G'\),
  because the number of component in the uncondensed graph
  is the same as the number of vertices in the condensed
  graph.

  Because \(a, b\) in the same component in \(Q\),
  there exists a tie path between \(a\) and \(b\) in \(Q\).
  By \cref{the: same tie structure same tie path},
  there exists a tie path between \(a\) and \(b\) in \(G'\).
  Because \(a\) and \(b\) are in different components in \(G'\),
  there exists a point on the tie path between \(a\) and \(b\)
  that ``crosses components'' in \(G'\).

  Formally, denote the tie path between \(a, b\) as:
  \([a, a_0, a_1, \ldots, a_n, b]\),
  then there exists \(a_p\) such that
  \(a_p\) and \(a_{p+1}\) are not in the same component.
  Because \(a_p\) ties \(a_{p+1}\)
  and they are in distinct components,
  then by \cref{the: vertex force image beating},
  \(f'(a)\) ties \(f'(a')\).
  Therefore \(T'\) is not a tournament.
  Contradiction.
\end{proof}

\cref{the: tie condensation effcient} states that
tie component condensation
is not only the most efficient condensation to tournaments
on any quasi-transitive orientated graph,
tie component condensation
is the most efficient in all the condensations defined on
all orientated graphs with the same tie structure.

To put it in other way,
for all the graph with the same tie structure,
the quasi-transitive orientated graphs
are the ones that can be condensed into tournaments
most efficiently.

\section{Kings}\label{sec: quasi-transitive king}

The definition of king states that it can beat every vertex
by one or two steps,
and the definition of quasi-transitive oriented graph states that
if there exists a path of length two (implies beats by 1 or 2 steps)
from one vertex to another vertex, then they are adjacent.

Both definitions are related to ``beats by 1 or 2 steps'',
therefore kings in quasi-transitive oriented graphs
have nice properties.

\begin{lemma}\label{the: king adjacent}
  In a quasi-transitive oriented graph,
  a king is adjacent to every vertex.
\end{lemma}

\begin{proof}
  Given a king \(k\), and another vertex \(v\),
  then there are two possibilities:
  \begin{itemize}
    \item
      Case 1, \(k\) beats \(v\) by one step:
      therefore \(k \to v\), \(k\) and \(v\) are adjacent.
    \item
      Case 2, \(k\) beats \(v\) by two steps:
      there exists vertex \(a\), such that \(k \to a \to v\).
      Then by definition of quasi-transitive oriented graph,
      \(k\) needs to be adjacent to \(v\)
  \end{itemize}
\end{proof}

\begin{lemma}\label{the: king partitions in quasi-transitive}
  In a quasi-transitive oriented graph,
  for any king \(k\), \(\set{D_k, S_k, \set{k}}\)
  partitions the graph.
\end{lemma}

\begin{proof}
  Proof disjoint: pretty obvious.
  \(D_k, S_k\) disjoint because we are working in an oriented graph.
  \(D_k, S_k\) disjoint with \(\set{k}\)
  because a vertex cannot beat itself.

  Proof the union is the whole vertex set:
  by \cref{the: king adjacent}
\end{proof}

\begin{lemma}\label{the: D S of king adjacent in quasi-transitive}
  In a quasi-transitive oriented graph,
  for any king \(k\), \(D_k, S_k\) are adjacent.
\end{lemma}

\begin{proof}
  By definition of quasi-transitive oriented graph,
  because every vertex in \(D_k\) beats \(k\)
  and then beats every vertex in \(S_k\),
  every vertex in \(D_k\) is adjacent to every vertex in \(S_k\).
\end{proof}

\begin{theorem}
  In a quasi-transitive oriented graph,
  if we have a king \(k\), then
  \begin{itemize}
    \item \({D_k, S_k, \set{k}}\) partitions the vertex set.
    \item \(D_k\) and \(S_k\) is adjacent.
  \end{itemize}
\end{theorem}

\begin{figure}
  \centering
  \tikz\graph[simple necklace layout, math nodes, node sep=1.75cm] {
      D_k [draw, circle, minimum size=2cm] ->
      k ->
      S_k [draw, circle, minimum size=2cm];
      ""; % chktex 18
      D_k --[bend left] S_k;
  };
  \caption{the rich structure of a king in a quasi-transitive oriented graph.}
  \label{fig: king in quasi-transitive}  % chktex 24
\end{figure}

See \cref{fig: king in quasi-transitive},
this figure shows the rich structure of a king
in quasi-transitive oriented graph.
We were able to see a similar partition structure mentioned in
\cref{chap: semi-complete digraph},
and \cref{the: graph partition lemma}
(the partition structure in \cref{fig: king in quasi-transitive}
is the same as \(\set{\set{v}, D_v, S_v}\)
for any vertex \(v\) in a tournament~\cite{maurer_king_1980}).

\(D_k\) cannot tie with anything outside of \(D_k\),
and similarly for \(S_k\).
If there is a king in a quasi-transitive oriented graph,
then this graph becomes ``very connected'',
the only places ties can appear are inside the
induced subgraphs of \(D_k\) and \(S_k\).

Since we have deduced nice structures about ties
in quasi-transitive oriented graphs,
it is only logical to combine the property of king
with the property of tie (tie component).

\begin{theorem}\label{the: king in quasi-transitive}
  A vertex \(k\) is a king in a quasi-transitive oriented
  graph if and only if
  \begin{itemize}
    \item \(k\) is in a trivial tie component.
    \item the result of \(k\) after tie component condensation
    is a king in the underlying tournament.
  \end{itemize}
\end{theorem}

\begin{proof}
  By \cref{the: condensation preserves king},
  \(k\) is a king if and only if
  \begin{itemize}
    \item \(k\) is a king in the induced subgraph of its tie component.
    \item the result of \(k\) after tie component condensation
    is a king in the underlying tournament.
  \end{itemize}

  So we need to show \(k\) is a king in
  induced subgraph of its tie component
  if and only if \(k\) is in a trivial tie component.

  Suppose \(k\) is a king in the induced graph of
  its tie component,
  and the fact that the induced subgraph is a quasi-transitive
  oriented graph.
  Because of \cref{the: king adjacent},
  \(k\) is a king in its tie component implies
  \(k\) is adjacent to all vertices in the tie component.
  There can be no tie path in this tie component.
  Therefore \(k\) has to be in a trivial tie component.

  Suppose \(k\) is is in a trivial tie component,
  then \(k\) is a king in the induced graph of
  its tie component.
 Recall that the definition of king says that
  a digraph with one vertex,
  that vertex is a king in the digraph.
\end{proof}

\begin{figure}
\centering
  \begin{subfigure}[b]{0.45\linewidth}
  \centering
    \tikz\graph[simple necklace layout, math nodes, node sep=1cm] {
      a_1 -> {b_1, b_2};
      {b_1, b_2} -> c_1;
      {b_1, b_2} -> c_2;
      {c_1, c_2} -> a_1;
      {b_1, b_2} -> d_1;
      {c_1, c_2} -> d_1;
      a_1 -> d_1;
      / [label=\(a\), draw] // {a_1};
      / [label=left:\(b\), draw] // {b_1, b_2};
      / [label=right:\(c\), draw] // {c_1, c_2};
      / [label=right:\(d\), draw] // {d_1};
    };
    \caption{quasi-transitive oriented graph \(G\).}
  \end{subfigure}
  \begin{subfigure}[b]{0.45\linewidth}
  \centering
    \tikz\graph[simple necklace layout, math nodes, node sep=1cm] {
      a -> b -> c -> a;
      {a, b, c} -> d;
    };
    \caption{the underlying tournament of \(G\).}
  \end{subfigure}
  \caption{a quasi-transitive digraph and its underlying tournament.}
  \label{fig: quasi-transitive king example}  % chktex 24
\end{figure}

In \cref{fig: quasi-transitive king example},
we show a quasi-transitive oriented graph \(G\)
and its underlying tournament.
In the underlying tournament the kings are
\(a, b, c\), but only \(a_1\) can be a king of \(G\),
since vertices \(b\) and \(c\) do not correspond to
trivial tie components.
(tie component \(\set{b_1, b_2}\) is condensed into \(b\),
and tie component \(\set{c_1, c_2}\) is condensed into \(c\)).
Notice, although vertex \(d_1\) is in a trivial tie component,
\(d_1\) is not a king in \(G\),
because in the underlying tournament, \(d\)
(the image of \(d_1\) under tie condensation) is not a king.

This theorem not only gives us a way to identify kings
in a quasi-transitive oriented graph,
but it helps us to construct a quasi-transitive oriented graph
with a certain number of kings.

\begin{figure}
\centering
  \begin{subfigure}[b]{0.45\linewidth}
  \centering
    \tikz\graph[simple necklace layout, math nodes, node sep=1cm] {
      a -> b -> c -> a;
      {a, b, c} -> d
    };
    \caption{start with a tournament with 3 kings.}
  \end{subfigure}
  \begin{subfigure}[b]{0.45\linewidth}
  \centering
    \tikz\graph[simple necklace layout, math nodes, node sep=1cm] {
      a -> {b_1, b_2} -> c -> a;
      b_1 --[dashed] b_2;
      {a, b_1, b_2, c} -> d;
      / [label=left:\(b\), draw] // {b_1, b_2}
    };
    \caption{then change king \(b\) to \(\set{b_1, b_2}\).}  %chktex 44
  \end{subfigure}
  \caption{construct a quasi-transitive oriented graph with 2 kings.}
  \label{fig: quasi-transitive 2 kings}  % chktex 24
\end{figure}

For example, if we want to construct a quasi-transitive
oriented graph with \(k\) kings,
we first start with a quasi-transitive graph with \(k'\) kings,
where \(k' > k\),
and then we change \(k' - k\) kings into
``non-trivial tie components''.
This method enables us to construct
a quasi-transitive oriented graphs
with arbitrary number of kings.
See \cref{fig: quasi-transitive 2 kings},
we start with a tournament with 3 kings \(\set{a, b, c}\),
then we change the king \(b\) into a non-trivial tie component
\(\set{b_1, b_2}\).
Then the only kings left are \(a\) and \(c\),
and we get a quasi-transitive oriented graph with 2 kings.

This way of constructing quasi-transitive oriented graphs
is very useful,
since we can control the number of ties in the graph.
For a tie components with \(n\) vertices,
the number of ties in the component can be any number
between \(n - 1\)
(ties forms a spanning tree of the component)
and \(\frac{n(n+1)}{2}\)
(there exists a tie between every two vertices).
And there is no tie between different tie components.

Another significant usage of \cref{the: king in quasi-transitive}
is to use the properties of kings in tournaments to prove
properties of kings in quasi-transitive oriented graphs.

\begin{definition}
  \(k\) is a \keyword{great king} in a digraph \(G\),
  if and only if \(k\) is the only king in digraph \(G\),
  and \(k\) is not an emperor.
\end{definition}

By \cref{the: one king iff emperor},
\cref{the: if only king then emperor},
a great king does not exists in semi-complete digraphs and
tournaments.

\begin{corollary}
  If there exists a great king in a
  quasi-transitive oriented graph \(G\),
  then there exists at least two ties in \(G\).
\end{corollary}

\begin{proof}
  Assume \(k\) is the great king in \(G\).
  Then there exists a vertex  \(v\) such that \(v \to k\).
  Denote the underlying tournament of \(G\) as \(T\),
  and the tie component condensation of \(G\) as \(f\).

  By \cref{the: vertex force image beating}, \(f(v) \to f(k)\).
  Therefore, by \cref{the: king in quasi-transitive}
  \(f(k)\) is a king, but not an emperor in \(T\).
  Because a tournament never has exactly 2 kings
  and exactly 1 king must be an emperor,
  there exists at least 3 kings in \(T\).
  Because \(k\) is the only king in \(G\),
  there has to exist at least 2 non-trivial tie components.
  Therefore, \(G\) have at least 2 ties.
\end{proof}

\cref{the: king in quasi-transitive}
provides us a way to detect, construct, and understand kings
in quasi-transitive oriented graph using
the properties and constructions of kings in tournaments.


  \chapter{Further Problems}

The existence of a \((n, k, t)\) digraph.
\begin{definition}
  a \((n, k, t)\) digraph is a digraph with \(n\) chickens, \(k\) kings and \(t\) ties.
\end{definition}


What is the property of a king with very low out-degree?

Can we extend these result to infinite graph?

How hard is a condensation?
How many graph only have identity condensation
and trivial condensation on them?

quasi-transitive oriented graph is strong
iff its underlying tournament is strong.
What does this imply?
~\cite{bang-jensen_kings_1998}
~\cite{bangjensen_quasitransitive_1995}




  %%% Post face
  % figure menu
  \listoffigures
  % bibliography
  \printbibliography[heading=bibintoc]
  % \bibliography{bib_file/bib_file}
  % index page
  \cleardoublepage{}
  \printindex


\end{document}
